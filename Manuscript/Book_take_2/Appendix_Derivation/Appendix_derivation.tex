%!TEX root = /Users/matt/Documents/GitRepositories/Dynamic-IO-Modeling/Manuscript/Book_take_2/Heun_Dale_Haney_A_dynamic_approach_to_input_output_modeling.tex
%%%%%%%%%%%%%%%%%%%%% appendix.tex %%%%%%%%%%%%%%%%%%%%%%%%%%%%%%%%%
%
% sample appendix
%
% Use this file as a template for your own input.
%
%%%%%%%%%%%%%%%%%%%%%%%% Springer-Verlag %%%%%%%%%%%%%%%%%%%%%%%%%%

\appendix
%\motto{All's well that ends well}
\chapter{Derivation of Equation~\ref{eq:Xdifference2_inverse}}
\label{app:C} % Always give a unique label
% use \chaptermark{}
% to alter or adjust the chapter heading in the running head

%Use the template \emph{appendix.tex} together with the Springer document class SVMono (monograph-type books) or SVMult (edited books) to style appendix of your book in the Springer layout.


We begin with a restatement of Equation \ref{eq:Xdifference2}.

\begin{equation} \label{eq:Xdifference2_inverse_Proof-1}
	\hat{\vec{X}} - \vec{X}_t^\mathrm{T} = \hat{\vec{X}}(\vec{I} - \vec{A}^\mathrm{T})
\end{equation}

\noindent We take the inverse of both sides of the equation to obtain

\begin{equation} \label{eq:Xdifference2_inverse_Proof-2}
	\left(\hat{\vec{X}} - \vec{X}_t^\mathrm{T}\right)^{-1} = \left(\hat{\vec{X}}(\vec{I} - \vec{A}^\mathrm{T})\right)^{-1}.
\end{equation}

\noindent We now apply the following matrix identity (formula 6.2, pg. 308 from \cite{Zwillinger2011})

%[MIK: THE REFERENCE FOR THIS IS BEYER, WILLIAM H. STANDARD MATHEMATICAL TABLES AND FORMULAE, CRC PRESS, 29TH EDITION, 1991, FORMULA 6.2, P. 308. NEEDS TO BE CITED WITH A PROPER TEX CITATION. --MKH]

\begin{equation} \label{eq:Xdifference2_inverse_Proof-3}
	\left(\vec{A}\vec{B}\vec{C}\right)^{-1} = \vec{C}^{-1} \vec{B}^{-1} \vec{A}^{-1}
\end{equation}

\noindent to the right side of Equation \ref{eq:Xdifference2_inverse_Proof-2} to obtain

\begin{equation} \label{eq:Xdifference2_inverse_Proof-4}
	\left(\hat{\vec{X}} - \vec{X}_t^\mathrm{T}\right)^{-1} = (\vec{I} - \vec{A}^{\mathrm{T}})^{-1} \hat{\vec{X}}^{-1},
\end{equation}

\noindent which is identical to Equation~\ref{eq:Xdifference2_inverse}.



%\section{Section Heading}
%\label{sec:A1}
%% Always give a unique label
%% and use \ref{<label>} for cross-references
%% and \cite{<label>} for bibliographic references
%% use \sectionmark{}
%% to alter or adjust the section heading in the running head
%Instead of simply listing headings of different levels we recommend to let every heading be followed by at least a short passage of text. Furtheron please use the \LaTeX\ automatism for all your cross-references and citations.
%
%
%\subsection{Subsection Heading}
%\label{sec:A2}
%Instead of simply listing headings of different levels we recommend to let every heading be followed by at least a short passage of text. Furtheron please use the \LaTeX\ automatism for all your cross-references and citations as has already been described in Sect.~\ref{sec:A1}.
%
%For multiline equations we recommend to use the \verb|eqnarray| environment.
%\begin{eqnarray}
%\vec{a}\times\vec{b}=\vec{c} \nonumber\\
%\vec{a}\times\vec{b}=\vec{c}
%\label{eq:A01}
%\end{eqnarray}
%
%\subsubsection{Subsubsection Heading}
%Instead of simply listing headings of different levels we recommend to let every heading be followed by at least a short passage of text. Furtheron please use the \LaTeX\ automatism for all your cross-references and citations as has already been described in Sect.~\ref{sec:A2}.
%
%Please note that the first line of text that follows a heading is not indented, whereas the first lines of all subsequent paragraphs are.
%
%% For figures use
%%
%\begin{figure}[t]
%\sidecaption[t]
%%\centering
%% Use the relevant command for your figure-insertion program
%% to insert the figure file.
%% For example, with the option graphics use
%\includegraphics[scale=.65]{figure}
%%
%% If not, use
%%\picplace{5cm}{2cm} % Give the correct figure height and width in cm
%%
%\caption{Please write your figure caption here}
%\label{fig:A1}       % Give a unique label
%\end{figure}
%
%% For tables use
%%
%\begin{table}
%\caption{Please write your table caption here}
%\label{tab:A1}       % Give a unique label
%%
%% For LaTeX tables use
%%
%\begin{tabular}{p{2cm}p{2.4cm}p{2cm}p{4.9cm}}
%\hline\noalign{\smallskip}
%Classes & Subclass & Length & Action Mechanism  \\
%\noalign{\smallskip}\hline\noalign{\smallskip}
%Translation & mRNA$^a$  & 22 (19--25) & Translation repression, mRNA cleavage\\
%Translation & mRNA cleavage & 21 & mRNA cleavage\\
%Translation & mRNA  & 21--22 & mRNA cleavage\\
%Translation & mRNA  & 24--26 & Histone and DNA Modification\\
%\noalign{\smallskip}\hline\noalign{\smallskip}
%\end{tabular}
%$^a$ Table foot note (with superscript)
%\end{table}
%%
