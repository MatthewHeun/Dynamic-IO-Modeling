%!TEX root = ../../Heun_Dale_Haney_A_dynamic_approach_to_input_output_modeling.tex
%%%%%%%%%%%%%%%%%%%%% chapter.tex %%%%%%%%%%%%%%%%%%%%%%%%%%%%%%%%%
%
% sample chapter
%
% Use this file as a template for your own input.
%
%%%%%%%%%%%%%%%%%%%%%%%% Springer-Verlag %%%%%%%%%%%%%%%%%%%%%%%%%%
%\motto{Use the template \emph{chapter.tex} to style the various elements of your chapter content.}
\motto{Where there is no reliable accounting 
and therefore no competent knowledge
of the economic and ecological effects of our lives,
we cannot live lives that are economically
and ecologically responsible. 
It is futile to plead and protest and lobby 
in favor of public ecological responsibility while, 
in virtually every act of our private lives, 
we endorse and support an economic system 
that is by intention, 
and perhaps by necessity, 
ecologically irresponsible.~\emph{\cite[p.~26]{Berry1998}}

\hfill---\emph{Wendell Berry}}


\chapter{Introduction}
% use \chaptermark{}
% to alter or adjust the chapter heading in the running head
\chaptermark{Introduction}
% Always give a unique label
\label{chap:intro}

\abstract*{**** Re-write the abstract. ****
In this chapter we give our motivation for writing this book. 
We outline some of the models and subsequent metaphors 
that have been used to describe the economy---clockwork, 
machine, engine---and suggest a new metaphor---the 
metabolism of an organism.
We give an overview of Leontief input-output methods
and their extension to include energy and material inputs
and waste flows out of the economy.
We then propose a new input-output analysis method,
fitting to the new metaphor of the metabolic economy;
a dynamic accounting framework that includes accumulation of stocks
within economic sectors.}

****	Red queen \& treadmill of production

\begin{quote}
by year 50 the cost of maintaining the capital stock 
has overwhelmed the income from resource extraction, 
so profits are no longer sufficient to keep investment ahead of depreciation. 
The operation quickly shuts down, as the capital stock declines. 
The last and most expensive of the resource stays in the ground; 
it doesn't pay to get it out~\cite[p.60]{Meadows2008}
\end{quote}
****

The Prologue demonstrates that, at least in the U.S., 
we are not accounting for natural assets, 
including the important categories of materials and energy,
in a way that informs meaningful economic discussion or
leads to enlightened natural resource and energy policies.
Whether you think this is a problem depends, in part, on 
whether you think the world's economies are approaching 
	natural resource extraction limits;
whether you think the materials and energy upon which the world currently depends
	can be easily substituted by other, readily-available, inexpensive resources; and
whether you think that the world's economies have exeeded 
	the biosphere's waste assimilation capacity.

There is ample evidence that the world is facing
natural resource extraction limits.
For many commodities, supply is scarce relative to demand, 
with oil being, arguably, the most important. 
Highly voaltile oil prices are an indicator of how very little 
spare oil production capacity exists in the world today. 
A ****\% change in demand following the 2009--2009 recession
led to a ****\% change in oil price! 
**** Need reference here. Probably use EIA data. ****
With oil prices spikes as leading indicators of all but one recession 
since World War II, it is clear that energy (especially oil)
is an extremely important commodity to the world's economy.
[**** Reference Hamilton here ****]
Having to reach yet deeper and still farther 
to produce natural resources (including oil) 
reinforces the observation that 
the world is reaching natural resource extraction limits.

Unfortunately, there are no known technical substitutes for oil 
in many sectors of the economy.
**** Discuss the lack of substitutibility **** 
**** Refer to deWit and Heun thinkpiece ****
**** Refer to Stern, Pelli, the Italian author on (lack of) 
substitutability of renewables for FFs. ****

That some economies have exceeded the assimilative capacity of the biosphere
is evident in China. **** More here about China's pollution problems. ****

If the world is facing natural resource extraction limits
(especially for some forms of energy), 
and if few substitues for fossil fuels (especially oil) exist,
it follows that some form of energy transformation is in our future.
The question is: how will the energy transformation unfold?
We contend that either the energy transformation 
will \emph{happen to} us or we will \emph{plan for} it.
As the Prologue points out, at present, the transition is happening \emph{to} us:
there is very little energy planning occurring worldwide.
Because it is likely that unplanned transitions will be rocky and difficult,
we believe that planning is the preferred way to achieve a smooth transition.

Clearly, effective planning requires knowledge of where things stand today.
We need to know the existing materials and energy structure of our economies. 
That implies that we should know the rates at which 
the world's economies consume raw materials and emit wastes back to the biosphere.
We need to understand the connections between the biosphere and the economies of the world.
In short, we need to be \emph{accounting for the environment}!\footnote{The title 
	of this book has a triple meaning.
	First, we should take the environment ``into \emph{account}''
		when developing our system of national accounts.
		As discussed in the Prologue, at present, we are not.
	Second, we should be maintaining and disseminating
		records or accounts of environmental assets. 
		(In this usage, \emph{account} is a noun
		referring to our endowment of natural resources.)
	Third, we should develop our system of national accounts 
		\emph{for the benefit of} the environment, 
		with the understanding 
		that a healthy environment sustains a healthy economy.
		(In this sense, \emph{accounting} is used as a verb with a telos.)
}

But, we also need to know how materials and energy comsumption 
are likely to evolve in the future.
We should know where materials and embodied energy accumulate 
in economic sectors and infrastructure. 
We should know the material and energy costs for maintaining the
increasingly-complex infrastructure of society.
We should have a sense of the transient and variable nature of economies
and the materials and energy flows that sustain them.

However, the Prologue clearly indicates that we (as a society) 
are not accounting in a way that helps us plan 
for the economic challenges that will accompany 
the impending materials and energy transformation.
All of which leads to a burning question,
one with significant consequences for the future:

\begin{quote}
	How can you maintain a system of national accounts without accounting for natural assets?
\end{quote}

The purpose of this book is to develop a dynamic model to help us
``account for the environment''
with the objective of planning for impending materials and energy transformations.

**** Discuss examples of better information leading to changed behvior here? ****


%%%%%%%%%% Metaphors and Models %%%%%%%%%%
\section{Metaphors and Models}
\label{sec:metaphors_and_models}

%%%%%%%%%%

Before moving ahead with the work of developing our dynamic model,
it is useful to consider how we got to this place. 
How is it that we don't account for the environment when 
considering materials and energy flows into, within, and out of the economy?

Most economics textbooks present the economy as a smoothly-running machine
that is disconnected from the biosphere.
The disconnected machine metaphor is an example of a simplification 
that helps us make sense of the world around us.
Metaphors like the disconnected machine influence us as stories told to ourselves.
They inform the mental and empirical models we construct.
And, as we collect data (via accounting methods) 
to assess the validity of those models,
our perception of the world is molded and shaped
by our accounting, which was informed, in the first place,
by the stories we told ourselves about reality.
Our models tell us what aspects of the world
are important to value 
(in the literal sense of making measurements),
and also, by extension, 
which parts of the world (literally) have no value.
This process has a deeper normative
consequence: the aspects of the world to which our models ascribe
value are \emph{valuable},
and those ascribed no value become \emph{worthless}.
Thus, metaphors inform our thinking about the real world,
but consequently,
they also constrain our ability to frame reality.
We mistake the model-metaphor for reality, and
we interact with reality in the same manner 
as we interact with the objects in our
metaphors.\footnote{This fallacious process is called
	\emph{reification}; the making (\emph{facere}, Latin) real of
	something (\emph{res}, Latin) that is merely an idea.
	Alfred Whitehead refers to this as
	\emph{the fallacy of misplaced concreteness}.\cite{Whitehead2011}}
Classical physics tells us the universe is
\emph{like} clockwork, 
so we begin to interact with the universe
as if it \emph{really were} clockwork.
Then, it becomes easy to collect data that confirms the clockwork model,
because the model tells us which data to collect.

The isolated machine metaphor precludes any sort of connection 
between the economy and the biosphere. 
It tells us that natural resources will always be available, 
and if a particular resource is becomes scarce, substitution will be possible.
It tells us that wastes are unimportant, effectively assuming that the biosphere
has infinite assimilative capacity.
Finally, the isolated machine metaphor tells us that economic forces 
(through prices and the market mechanism) will smoothly guide any transition
to good and just outcomes.
In short, the machine can and will carry on.

If our collective imagination is stuck on a metaphor that tells us 
the economy is disconnected from the biosphere,
none of the challenges cited earlier (natural resource extraction limits,
difficulty of material and energy substitutions, and 
exceeding the biosphere's waste assimilation capacity)
seem important. 
Nor will the suggestion to ``account for the environment'' be convincing.

But, if we hope to better understand the complex, 
messy dynamics of real-world economies,
if we hope to make sense of real-world events, 
if we hope to learn where and how economies can go wrong;
we had better be counting data that informs \emph{dynamic} models
guided by metaphors that tell us \emph{more} than ``the world is an orderly place.''
We need metaphors and models that are
able to cope with rapid transience,
not just ordered stability.


%%%%%%%%%% An Apt Metaphor %%%%%%%%%%
\section{An Apt Metaphor for the Economy}
\label{sec:apt_metaphor}
%%%%%%%%%%

If the isolated machine metaphor is inappropriate, 
what might an apt metaphor for the economy be?
And, if we find an apt metaphor, what types of models would it inform?
Furthermore, what data would the models tell us to collect?
I.e., how should we do our accounting?

To begin the search for an apt metaphor, 
we might first ask the question, 
what characteristics are required for an apt metaphor for the economy?
In our opinion, an apt metaphor should capture the following
aspects of real economies:\footnote{We note that 
	several areas of the literature speak to the items in this list.
	Materials Flow Analysis (MFA) and 
	Economy-Wide Materials Flow Analys (EW-MFA)
	stress the importance of
	material intake by the economy. 
	(See Section~\ref{sec:materials_auto}.)
	The Input-Output method highlights the effects of internal exchanges
	of material and information with economies. 
	(See Chapter~\ref{chap:intensity}.)
	Life-Cycle Assessment (LCA) techniques focus attention 
	on otherwise-neglected wastes. 
	(See Section~\ref{sec:intensity_auto}.)
	Net Energy Analysis (NEA) predicts that energy resource 
	scarcity reduces Energy Return on Investment (EROI)
	and increases energy prices.
	(See Sections~\ref{sec:B_energy} and~\ref{sec:resource_quality_and_irreversibility}.)
	The Energy Input-Output (EI-O) method gives prominence to energetic costs
	for internal material and energy flows.
	(See Chapter~\ref{chap:intensity}.)
	And, thermodynamic control-volume modeling describes
	transient behavior and system transformations.
	(See Chapters~\ref{chap:materials}--\ref{chap:value}.)}

\begin{itemize}
	\item{intakes material and energy from the biosphere,}
	\item{exchanges materials and information internally,}
	\item{discharges wastes to the biosphere,}
	\item{affected by energetic costs,}
	\item{affected non-linearly by scarcity in the face of low substitutibility,}
	\item{enable non-linear dynamics with the potential for structural transformation, and}
	\item{embodies energy in material stocks.}
\end{itemize}

Living metabolisms\footnote{The 
	Greek root of metabolism 
	(\emph{metabol$\bar{e}$}) means ``change.''}
exhibit the characteristics in the list above.
Metabolisms and the organisms they support
withdraw materials and energy from the biosphere, 
exchange materials and energy internally via metabolic processes, 
and discharge wastes to the biosphere.
Metabolisms are affected by energetic costs: 
an organism that obtains less energy than it expends 
is doomed to fail.
Metabolisms struggle mightily to achieve a new equillibrium
when confronted with resource scarcity (e.g., migrations).
Metabolisms enable non-linear structural transformations 
in their host organisms (e.g., metamorphosis, puberty, and evolution).
And, energy acquired by a metabolism is considered to be ``embodied''
in the material stock (cells) of the organism.

The economy is society's metabolism. 


%%%%%%%%%% Apt Models %%%%%%%%%%
\section{An Apt Material, Energy, and Economy Model}
\label{sec:apt_models}
%%%%%%%%%%

As discussed in Section~\ref{sec:metaphors_and_models}, 
metaphors give rise to models.
So, the natural question is: 
``what types of models does the metabolism metaphor inform?''
But, before developing our models, we can ask the question
``what attributes should be present in models informed
by the metabolism metaphor?'' 
In our opinion, apt models should have the following characteristics:

\begin{itemize}
	\item{accounts for flows of materials into, within, and out of the economy,}
	\item{allows accumulation of materials within the economy,}
	\item{accounts for flows of energy into, within, and out of the economy,}
	\item{allows accumulation of energy embodied within the economy,}
	\item{provides metrics that link energy and economic value, and}
	\item{is compatible with existing systems of national accounts.}
\end{itemize}

Transient conservation equations often employed by thermodynamicists 
provide the first four items in the list above.
We will adapt an existing energy analysis method to obtain a 
technique for obtaining metrics that link energy and economic value
(the fifth item in the list above).
Finally, we note that systems of national accounts use the economic sector
as their level of analysis. 
Thus, our models should be also implemented at the economic sector level
so that results from the models we develop can be 
integrated with and compared against existing systems of national accounts.


%%%%%%%%%% What to Count %%%%%%%%%%
\section{What to Count?}
\label{sec:what_to_count}
%%%%%%%%%%

After deciding to utilize transient thermodynamic equations to develop our model, 
the need arises to specify what must be counted. 
The answer to the question ``what should be accounted'' leads to the development
of an accounting framework.
We believe the key to understanding how energy transformations will unfold
involves understanding how 

\begin{itemize}
	\item{materials,}
	\item{energy,}
	\item{embodied energy, and}
	\item{economic value}
\end{itemize}

interact in society's metabolism, the economy.


%%%%%%%%%% Structure %%%%%%%%%%
\section{Structure of the Book}
\label{sec:structure}
%%%%%%%%%%

The list in Section~\ref{sec:what_to_count} 
provides the beginnings of the outline for the remainder of this book.
Part~\ref{part:matter} addresses flows of physical matter and energy
through the economy.
Chapter~\ref{chap:materials} discusses material flows.
Flows of energy are covered in Chapter~\ref{chap:direct_energy}, 
and a rigorous, thermodynamics-based definition of and accounting for 
embodied energy is presented in Chapter~\ref{chap:embodied_energy}.
In Part~\ref{part:values} we turn to non-physical flows through the economy. 
Flows of economic value are discussed in Chapter~\ref{chap:value}.
In Chapter~\ref{chap:intensity} we combine the results from 
Chapters~\ref{chap:embodied_energy} and~\ref{chap:value} to
calculate an important indicator of economic activity:
the energy intensity of economic production.
Part~\ref{part:implications} gives context to the framework developed in
Parts~\ref{part:matter}~and~\ref{part:values}.
Chapter~\ref{chap:implications} draws out some of the direct implications
of the results.
Chapter~\ref{chap:unfinished_business} looks at 
unfinished business: practical, conceptual, and theoretical issues
that arise in the development of this framework.
We finish off the book with a summary in Chapter~\ref{chap:summary}.

Throughout the methodological chapters~(\ref{chap:materials}--\ref{chap:value}),
our accounting framework is developed
through a series of increasingly-disaggregated
models of the economy (Table~\ref{tab:examplesABC}).

\begin{table}
\caption[Examples used throughout this book]{Examples
used throughout this book.}
\begin{center}
  \begin{tabular}{r @{\hspace{2em}} c @{\hspace{2em}} c @{\hspace{2em}} c @{\hspace{2em}} c}
    \toprule
    Example & Sector 0 & Sector 1 & Sector 2 & Sector 3 \\ 
	\midrule
    A & Biosphere	&	Society            & NA         & NA                 \\
    B & Biosphere	&	Final Consumption  & Production & NA                 \\
    C & Biosphere	&	Final Consumption  & Energy     & Goods \& Services  \\
  \bottomrule
  \end{tabular}
\end{center}
\label{tab:examplesABC}
\end{table}
 
In addition, we use the U.S. auto industry 
as a running example for application and discussion.

**** Add more detail to justify the auto industry example. **** 

**** Here are notes from previous email corresondence. ****

The running example of the US auto industry demonstrates that our dynamic model 
can be tied to the System of National Accounts, 
thereby meeting one of the requirements for a dynamic model. 
Our purpose is not to update previous energy intensity calculations.

The US auto industry example shows where data are available 
(e.g., economic value, Ch 5), 
where it is old (e.g., energy intensity, Ch 6), 
and where it has never been available (e.g., accumulated embodied energy, Ch 4).  
The US auto industry is, therefore, 
illustrative of the challenges inherent in obtaining data that would feed the model.

**** The following notes are from the authors telecon on Tue 15 Apr 2014. ****

Historical - Berry and Fels

Economic importance

Social/cultural importance

Large energy consumer (both directly and indirectly)

**** The next paragraph is the original paragraph about the auto industry 
from v1 of the Introduction. ****

We choose the auto industry,
because it remains a large portion of many industrialized economies, 
because is very resource intensive,
because it has been used in the literature~\cite{Bullard:1978vd}
to illustrate input-output analysis methods, 
because its links with energy are obvious,
because its health is sensitive to disruptions in energy supplies, and
because it shows evidence of post-industrial decline (shrinking profit margins, etc.).























%%%%%% Problem		%%%%%%
\section{Problem: Our accounting systems are inadequate}
\label{sec:problem}
%%%%%%

We're not counting materials, energy, and economic value in a way that allows society to (a) understand the connections between energy and the economy and (b) plan for a future in which energy transitions/transformations will occur. Why is this important?

%%%%%% Transformations		%%%%%%
\subsection{The world is facing energy transformations}
\label{sec:transormations}
%%%%%%



%%%%%% Planning		%%%%%%
\subsection{We need to plan for the imminent transformations}
\label{sec:planning}
%%%%%%

Either the transformations will happen to us, or we will plan for them. We believe that planning is preferred.

%%%%%%	Economic structure	%%%%%%
\subsection{Structure of the economy is important}
\label{sec:structure}
%%%%%%

To plan for the coming transformation, we need to know the existing materials and energy structure of our economy. And, we need to know where materials and embodied energy accumulate in economic sectors and infrastructure. In short, we need to understand the connections between energy and the economy.

%%%%%%	Isolated machine metaphor	%%%%%%
\subsection{Materials and energy not accounted}
\label{sec:accounting}
%%%%%%

However, we (as a society) are not accounting the necessary things to help us plan for the economic challenges that will accompany the impending energy transformation. In fact, Congress is preventing the BEA from doing this! (See Prologue.)


%%%%%%	Scarcity	%%%%%%
\subsection{Relative vs. absolute scarcity and depletion!}
\label{sec:scarcity}
%%%%%%

%%%%%% Solution	%%%%%%
\section{Solution: we need to account for materials and energy}
\label{sec:solution}
%%%%%%

Our book provides a dynamic model that shows (a) what should be accounted and (b) how to analyze the data. This is a step along the path of solving the stated problem.

%%%%%% Collective imagination	%%%%%%
\section{Why is our collective imagination stuck on a view of the economy as isolated from the biosphere?}
\label{sec:collective_imagination}
%%%%%%

%%%%%%	Isolated machine metaphor	%%%%%%
\subsection{The isolated economy}
\label{sec:isolated_economy}
%%%%%%

The isolated machine metaphor for the economy tells us that structural transformations will occur smoothly. 
     The isolated machine metaphor leads to some incorrect assumptions:

%%%%%%	Economy not connected to biosphere	%%%%%%
\subsubsection{No connection between the economy and the biosphere}
\label{sec:no_connection}
%%%%%%

%%%%%% Infinite resources	%%%%%%
\subsubsection{Resources are always available}
\label{sec:infinite_resources}
%%%%%%

Resources are always available.  If a particular resource is becomes scarce, substitute.

%%%%%% Infinite waste		%%%%%%
\subsubsection{Infinite assimilative capacity for wastes}
\label{sec:infinite_waste}
%%%%%%

%%%%%% Internal dynamics	%%%%%%
\subsubsection{Only internal dynamics affect the economy}
\label{sec:internal_dynamics}
%%%%%%

%%%%%%	Economic forces	%%%%%%
\subsubsection{Economic forces (through prices and the market mechanism) will smoothly guide transition processes to good and just outcomes}
\label{sec:economic_forces}
%%%%%%


%%%%%% Machine endures	%%%%%%
\subsubsection{The machine can and will carry on}
\label{sec:machine_endures}
%%%%%%

%%%%%% Destructive transformations	%%%%%%
\subsection{Transformations have been rocky to this point. (eg. 1973, 2008)}
\label{sec:transformations}
%%%%%%

%%%%%%	Wrong models	%%%%%%
\subsection{Perhaps our economic models are wrong}
\label{sec:wrong_models}
%%%%%%

But, economic models are informed by metaphors. Perhaps our metaphors are wrong.

%%%%%%	Wrong metaphors		%%%%%%
\subsection{Perhaps our metaphors are wrong}
\label{sec:wrong_metaphors}
%%%%%%


%%%%%%	Importance of metaphors		%%%%%%
\subsection{Importance of metaphors}
\label{sec:importance_metaphors}
%%%%%%

%%%%%%	Requirements	%%%%%%
\section{Attributes for an apt metaphor for the economy}
\label{sec:metaphor_requirements}
%%%%%%

see first column of table below

%%%%%% Inputs from biosphere	%%%%%%
\subsection{Intakes materials and energy from the biosphere}
\label{sec:inputs}
%%%%%%

%%%%%%	Internal exchanges	%%%%%%
\subsection{Exchanges materials and information and dissipates energy internally}
\label{sec:internal_exchange}
%%%%%%

%%%%%%	Discharges wastes	%%%%%%
\subsection{Discharges wastes to the biosphere}
\label{sec:discharges_waste}
%%%%%%

%%%%%% Accounts for scarcity	%%%%%%
\subsection{Accounts for effects of scarcity (non-linearly) in the face of low-substitutability (and uncertainty)}
\label{sec:accounts_for_scarcity}
%%%%%%

%%%%%% Importance of energy	%%%%%%
\subsection{Places importance on energetic costs}
\label{sec:importance_of_energy}
%%%%%%

%%%%%%	Dynamics and structure	%%%%%%
\subsection{Able to address dynamics and structural transformations}
\label{sec:structure}

Able to describe energy embodied in capital stock of economic sectors.

%%%%%%	Vefifiable	%%%%%%
\subsection{Informs models that are verifiable against historical data}
\label{sec:verifiable}
%%%%%%

%%%%%%	Metabolic metaphor %%%%%%
\section{The metabolic economy}
\label{sec:metabolic_economy}
%%%%%%

Perhaps a better metaphor for the economy is a metabolism (Greek root means ``change''). 

%%%%%%	Metabolisms work %%%%%%
\subsection{Metabolism captures the desired attributes}
\label{sec:metabolism_works}
%%%%%%
     
Metabolisms have the characteristics to meet the requirements outlined above.
How can we fill in the metabolism metaphor? What ``ideas ... are lying around''?
How can we fill in the metabolism metaphor? 
There are useful ideas ``lying around'' that can address these important aspects of economies. 
(See middle column of Table~\ref{tab:metabolic_economy}.) 
However, ideas 1–5 above create snapshots of flows. 
None (except 6) provides structural information. 
In other words, 1–5 talk about the blood, not the bones. 
We need to understand flows (blood) and structure (stocks, bones). 
6 helps with that.

{
% Put all of this inside braces to scope the redefinition of arraystretch and tabcolsep
\renewcommand{\arraystretch}{1.5}
\setlength{\tabcolsep}{10pt}

\begin{landscape}
\begin{table}
\label{tab:metabolic_economy}
\caption{Attributes of the real economy that should be captured by our economic models and metaphors 
				and ``useful ideas'' (analytic tools) that can be used to capture these attributes. **** Put a version of this table into the appendix? 
				Decide after it is clear how the introduction is developing. ****}
\begin{tabular}{m{0.1cm}m{7cm}m{3cm}m{6cm}}

\toprule
																													&	
	\textbf{Attribute}																					&
	\textbf{Useful ideas}																				&
	\textbf{Significance}																				\\

\midrule

1																													&	
	Intakes material and energy from the biosphere										&
	MFA, EW-MFA																							&
	Stresses importance of material intake													\\[4pt]
2																													&
	Exchanges materials and information internally										&
	I-O method																								&
	Highlights importance of internal exchanges											\\[4pt]
3																													&
	Discharges wastes to the biosphere															&
	LCA, EIOLCA																								&
	Focuses attention on otherwise-neglected wastes 
		(wastes are systemic, not exceptional)												\\[4pt]
4																													&
	Affected by scarcity (non-linearity) in the face of 
		low-substitutability (and uncertainty)													&
		EROI																										&
		Predicts that energy resource scarcity reduces EROI 
			and increases prices																			\\[4pt]
5																													&
	Affected by energetic costs																		&
	EI-O method																								&
	Gives prominence to energetic costs for internal flows							\\[4pt]
6																													&
	Exhibits non-linear dynamics with 
		the potential for structural transformation.
		Embodies energy in capital stock of economic sectors.						&
	Thermodynamic control volume modeling												&
	Describes transient behavior and transformations.
		Provides ``accounting'' equations.															\\
		
\bottomrule

\end{tabular}
\end{table}
\end{landscape}

}

How can we fill in the metabolism metaphor? There are ideas “lying around” that can address these important aspects of economies. (See middle column of table.) However, ideas 1–5 above create snapshots of flows. None (except 6) provides structural information. In other words, 1–5 talk about the blood, not the bones. We need to understand flows (blood) and structure (stocks, bones). 6 helps with that.


%%%%%%	Analytic tools	%%%%%%
\subsection{Filling in the metabolism metaphor with useful ideas: analytic tools}
\label{sec:analytic_tools}
%%%%%%

What analytical tools and data are at our disposal?
(Each sub-point below ties to the same sub-points in Item 7 above.)

%%%%%%	MFA	%%%%%%
\subsubsection{Material flow analysis (MFA)}
\label{sec:MFA}
%%%%%%

MFA, EW-MFA: Stresses the importance of material intake

%%%%%%	I-O	%%%%%%
\subsubsection{Input-Output (I-O) method}
\label{sec:I-O}
%%%%%%

I-O method: Highlights importance of internal exchanges

%%%%%%	LCA	%%%%%%
\subsubsection{Life cycle analysis (LCA)}
\label{sec:LCA}
%%%%%%

LCA, EIOLCA: Focuses attention on otherwise-neglected wastes

%%%%%%	NEA	%%%%%%
\subsubsection{Net energy analysis}
\label{sec:NEA}
%%%%%%

EROI: Predicts that energy resource scarcity reduces EROI and increases prices

%%%%%%	EI-O method		%%%%%%
\subsubsection{Energy Input-Output (EI-O) method}
\label{sec:EI-O}
%%%%%%

EI-O method: gives prominence to energetic costs for all internal flows

%%%%%%	CV modeling		%%%%%%
\subsubsection{Thermodynamic control volume (CV) modeling}
\label{sec:CV_modeling}
%%%%%%

Thermodynamic control volume modeling: captures transient behavior 
and transformations.  
Provides ``accounting'' equations.

%%%%%%	Historically verifiable 	%%%%%%
\subsection{Apt metaphor should inform models that are verifiable against historical data}
\label{sec:historically_verifiable}
%%%%%%

Furthermore, an apt metaphor should inform models that are verifiable against historical data. We need a way to collect and categorize the information. The SNA is supposed to do that. But, the BEA doesn’t include important information (material flows and accumulation, for example) in the SNA. 

%%%%%%	National Accounts	%%%%%%
\subsubsection{System of National Accounts}
\label{sec:national_accounts}
%%%%%%

System of National Accounts: provides data for analysis and verification purposes.  
Needs to be expanded.

These facts bring us back to:
\begin{enumerate}
	\item our burning question: how can you maintain a system of national accounts without accounting for natural assets? and
	\item stated problem: we're not counting materials, energy, and economic value in a way that allows society to
				\begin{enumerate}
					\item understand the connections between energy and the economy
					\item plan for a future in which energy transitions/transformations will occur.
				\end{enumerate}
\end{enumerate}


%%%%%%	Restate the problem 	%%%%%%
\section{Restate the Problem: we're not counting materials, energy, and value}
\label{sec:restate_problem}
%%%%%%

We’re not counting materials, energy, and value in a way that allows society to: 
\begin{enumerate}
	\item understand the connections between energy and the economy
	\item plan for a future in which energy transitions/transformations will occur.
\end{enumerate} 




%%%%%%	Need		%%%%%%
\section{Need: a new modeling approach}
\label{sec:need}
%%%%%%

Need: A modeling approach that
\begin{enumerate}
	\item brings together the ideas ``lying around,''
	\item has the potential to describe structural transformations of the economy, and 
	\item provides a framework for improved economic and environmental accounting.
\end{enumerate} 


%%%%%%	Approach		%%%%%%
\section{Methodological approach}
\label{sec:approach}
%%%%%%

Our approach to meet the need is to develop a dynamic model by applying rigorous thermodynamics to economies, informed by the ideas lying around and the metabolic metaphor, in a manner that is verifiable against the existing (or expanded) System of National Accounts (SNA).

Where and how should we apply rigorous thermodynamics to develop our dynamic model? 

%%%%%%	Structure		%%%%%%
\subsection{What accounting must be done well to understand the dynamics of economies?}
\label{sec:structure}
%%%%%%

%%%%%%	Materials		%%%%%%
\subsubsection{Materials}
\label{sec:materials}
%%%%%%

Materials (Chapter 2)

%%%%%%	Energy		%%%%%%
\subsubsection{Energy}
\label{sec:energy}
%%%%%%

Energy (Chapter 3)

%%%%%%	Embodied Energy		%%%%%%
\subsubsection{Embodied energy}
\label{sec:embodied_energy}
%%%%%%

Embodied energy (Chapter 4)

%%%%%%	Economic value	%%%%%%
\subsubsection{Economic value}
\label{sec:economic_value}
%%%%%%

Economic value (already done very well by the System of National Accounts, Chapter 5)

%%%%%%	Key metrics	%%%%%%
\subsubsection{Key metrics}
\label{sec:key_metrics}
%%%%%%

Key metrics that arise from this dynamic model: energy intensity and EROI (Chapter 6)

The list above provides the outline for the center of the book. The rest of the outline is:

%%%%%%	Implications	%%%%%%
\subsubsection{Implications from the dynamic model}
\label{sec:implications}
%%%%%%

Implications from the dynamic model (Chapter 7)

%%%%%%	Unfinished business		%%%%%%
\subsubsection{Unfinished business}
\label{sec:unfinished_business}
%%%%%%

Unfinished business (Chapter 8)

%%%%%%	Summary		%%%%%%
\subsubsection{Summary}
\label{sec:summary}
%%%%%%

Summary (Chapter 9)


%%%%%%  Examples  %%%%%%
\subsection{Examples}
%%%%%%

%%%%%%  Economies  %%%%%%
\subsubsection{Economies}
%%%%%%

Indicate that we'll use example economies A--C
throughout the text.


%%%%%%  Auto Industry Example  %%%%%%
\subsubsection{Auto Industry}
%%%%%%

Motivate Auto Industry example thoroughly and clearly. 

The running example of the US auto industry demonstrates that our dynamic model 
can be tied to the System of National Accounts, 
thereby meeting one of the requirements for a dynamic model. 
Our purpose is not to update previous energy intensity calculations.

Bring in some of the verbiage from the email conversations.

Historical - Berry and Fels

Economic importance

Social/cultural importance

Large energy consumer (both directly and indirectly)




\bibliographystyle{unsrt}
\bibliography{../../Metabolic}


% Always give a unique label
% and use \ref{<label>} for cross-references
% and \cite{<label>} for bibliographic references
% use \sectionmark{}
% to alter or adjust the section heading in the running head
%% Instead of simply listing headings of different levels we recommend to let every heading be followed by at least a short passage of text. Furtheron please use the \LaTeX\ automatism for all your cross-references and citations.

%% Please note that the first line of text that follows a heading is not indented, whereas the first lines of all sequent paragraphs are.

%% Use the standard \verb|equation| environment to typeset your equations, e.g.
%
%% \begin{equation}
%% a \times b = c\;,
%% \end{equation}
%
%% however, for multiline equations we recommend to use the \verb|eqnarray|
%% environment\footnote{In physics texts please activate the class option \texttt{vecphys} to depict your vectors in \textbf{\itshape boldface-italic} type - as is customary for a wide range of physical jects.}.
%% \begin{eqnarray}
%% a \times b = c \nonumber\\
%% \vec{a} \cdot \vec{b}=\vec{c}
%% \label{eq:01}
%% \end{eqnarray}

%% \section{section Heading}
%% \label{sec:2}
%% Instead of simply listing headings of different levels we recommend to let every heading be followed by at least a short passage of text. Furtheron please use the \LaTeX\ automatism for all your cross-references\index{cross-references} and citations\index{citations} as has already been described in Sect.~\ref{sec:2}.

%% \begin{quotation}
%% Please do not use quotation marks when quoting texts! Simply use the \verb|quotation| environment -- it will automatically render Springer's preferred layout.
%% \end{quotation}


%% \section{section Heading}
%% Instead of simply listing headings of different levels we recommend to let every heading be followed by at least a short passage of text. Furtheron please use the \LaTeX\ automatism for all your cross-references and citations as has already been described in Sect.~\ref{sec:2}, see also Fig.~\ref{fig:1}\footnote{If you copy text passages, figures, or tables from other works, you must obtain \textit{permission} from the copyright holder (usually the original publisher). Please enclose the signed permission with the manucript. The sources\index{permission to print} must be acknowledged either in the captions, as footnotes or in a separate section of the book.}

%% Please note that the first line of text that follows a heading is not indented, whereas the first lines of all sequent paragraphs are.

% For figures use
%
%% \begin{figure}[b]
%% \sidecaption
% Use the relevant command for your figure-insertion program
% to insert the figure file.
% For example, with the option graphics use
%% \includegraphics[scale=.65]{figure}
%
% If not, use
%\picplace{5cm}{2cm} % Give the correct figure height and width in cm
%
%% \caption{If the width of the figure is less than 7.8 cm use the \texttt{sidecapion} command to flush the caption on the left side of the page. If the figure is positioned at the top of the page, align the sidecaption with the top of the figure -- to achieve this you simply need to use the optional argument \texttt{[t]} with the \texttt{sidecaption} command}
%% \label{fig:1}       % Give a unique label
%% \end{figure}


%% \paragraph{Paragraph Heading} %
%% Instead of simply listing headings of different levels we recommend to let every heading be followed by at least a short passage of text. Furtheron please use the \LaTeX\ automatism for all your cross-references and citations as has already been described in Sect.~\ref{sec:2}.

%% Please note that the first line of text that follows a heading is not indented, whereas the first lines of all sequent paragraphs are.

%% For typesetting numbered lists we recommend to use the \verb|enumerate| environment -- it will automatically render Springer's preferred layout.

%% \begin{enumerate}
%% \item{Livelihood and survival mobility are oftentimes coutcomes of uneven socioeconomic development.}
%% \begin{enumerate}
%% \item{Livelihood and survival mobility are oftentimes coutcomes of uneven socioeconomic development.}
%% \item{Livelihood and survival mobility are oftentimes coutcomes of uneven socioeconomic development.}
%% \end{enumerate}
%% \item{Livelihood and survival mobility are oftentimes coutcomes of uneven socioeconomic development.}
%% \end{enumerate}


%% \paragraph{paragraph Heading} In order to avoid simply listing headings of different levels we recommend to let every heading be followed by at least a short passage of text. Use the \LaTeX\ automatism for all your cross-references and citations as has already been described in Sect.~\ref{sec:2}, see also Fig.~\ref{fig:2}.

%% Please note that the first line of text that follows a heading is not indented, whereas the first lines of all sequent paragraphs are.

%% For unnumbered list we recommend to use the \verb|itemize| environment -- it will automatically render Springer's preferred layout.

%% \begin{itemize}
%% \item{Livelihood and survival mobility are oftentimes coutcomes of uneven socioeconomic development, cf. Table~\ref{tab:1}.}
%% \begin{itemize}
%% \item{Livelihood and survival mobility are oftentimes coutcomes of uneven socioeconomic development.}
%% \item{Livelihood and survival mobility are oftentimes coutcomes of uneven socioeconomic development.}
%% \end{itemize}
%% \item{Livelihood and survival mobility are oftentimes coutcomes of uneven socioeconomic development.}
%% \end{itemize}

%% \begin{figure}[t]
%% \sidecaption[t]
% Use the relevant command for your figure-insertion program
% to insert the figure file.
% For example, with the option graphics use
%% \includegraphics[scale=.65]{figure}
%
% If not, use
%\picplace{5cm}{2cm} % Give the correct figure height and width in cm
%
%% \caption{Please write your figure caption here}
%% \label{fig:2}       % Give a unique label
%% \end{figure}

%% \runinhead{Run-in Heading Boldface Version} Use the \LaTeX\ automatism for all your cross-references and citations as has already been described in Sect.~\ref{sec:2}.

%% \runinhead{Run-in Heading Italic Version} Use the \LaTeX\ automatism for all your cross-refer\-ences and citations as has already been described in Sect.~\ref{sec:2}\index{paragraph}.
% Use the \index{} command to code your index words
%
% For tables use
%
%% \begin{table}
%% \caption{Please write your table caption here}
%% \label{tab:1}       % Give a unique label
%
% For LaTeX tables use
%
%% \begin{tabular}{p{2cm}p{2.4cm}p{2cm}p{4.9cm}}
%% \hline\noalign{\smallskip}
%% Classes & class & Length & Action Mechanism  \\
%% \noalign{\smallskip}\svhline\noalign{\smallskip}
%% Translation & mRNA$^a$  & 22 (19--25) & Translation repression, mRNA cleavage\\
%% Translation & mRNA cleavage & 21 & mRNA cleavage\\
%% Translation & mRNA  & 21--22 & mRNA cleavage\\
%%Translation & mRNA  & 24--26 & Histone and DNA Modification\\
%%\noalign{\smallskip}\hline\noalign{\smallskip}
%%\end{tabular}
%%$^a$ Table foot note (with superscript)
%%\end{table}
%
%% \section{Section Heading}
%%\label{sec:3}
% Always give a unique label
% and use \ref{<label>} for cross-references
% and \cite{<label>} for bibliographic references
% use \sectionmark{}
% to alter or adjust the section heading in the running head
%% Instead of simply listing headings of different levels we recommend to let every heading be followed by at least a short passage of text. Furtheron please use the \LaTeX\ automatism for all your cross-references and citations as has already been described in Sect.~\ref{sec:2}.

%% Please note that the first line of text that follows a heading is not indented, whereas the first lines of all sequent paragraphs are.

%%If you want to list definitions or the like we recommend to use the Springer-enhanced \verb|description| environment -- it will automatically render Springer's preferred layout.

%%\begin{description}[Type 1]
%%\item[Type 1]{That addresses central themes pertainng to migration, health, and disease. In Sect.~\ref{sec:1}, Wilson discusses the role of human migration in infectious disease distributions and patterns.}
%%\item[Type 2]{That addresses central themes pertainng to migration, health, and disease. In Sect.~\ref{sec:2}, Wilson discusses the role of human migration in infectious disease distributions and patterns.}
%%\end{description}

%%\section{section Heading} %
%% In order to avoid simply listing headings of different levels we recommend to let every heading be followed by at least a short passage of text. Use the \LaTeX\ automatism for all your cross-references and citations citations as has already been described in Sect.~\ref{sec:2}.

%% Please note that the first line of text that follows a heading is not indented, whereas the first lines of all sequent paragraphs are.

%% \begin{svgraybox}
%% If you want to emphasize complete paragraphs of texts we recommend to use the newly defined Springer class option \verb|graybox| and the newly defined environment \verb|svgraybox|. This will produce a 15 percent screened box 'behind' your text.

%% If you want to emphasize complete paragraphs of texts we recommend to use the newly defined Springer class option and environment \verb|svgraybox|. This will produce a 15 percent screened box 'behind' your text.
%% \end{svgraybox}


%% \section{section Heading}
%%Instead of simply listing headings of different levels we recommend to let every heading be followed by at least a short passage of text. Furtheron please use the \LaTeX\ automatism for all your cross-references and citations as has already been described in Sect.~\ref{sec:2}.

%% Please note that the first line of text that follows a heading is not indented, whereas the first lines of all sequent paragraphs are.

%% \begin{theorem}
%% Theorem text goes here.
%% \end{theorem}
%
% or
%
%% \begin{definition}
%% Definition text goes here.
%% \end{definition}

%% \begin{proof}
%\smartqed
%% Proof text goes here.
%% \qed
%% \end{proof}

%%\paragraph{Paragraph Heading} %
%% Instead of simply listing headings of different levels we recommend to let every heading be followed by at least a short passage of text. Furtheron please use the \LaTeX\ automatism for all your cross-references and citations as has already been described in Sect.~\ref{sec:2}.

%% Note that the first line of text that follows a heading is not indented, whereas the first lines of all subsequent paragraphs are.
%
% For built-in environments use
%
%%\begin{theorem}
%%Theorem text goes here.
%%\end{theorem}
%
%%\begin{definition}
%%Definition text goes here.
%%\end{definition}
%
%%\begin{proof}
%%\smartqed
%% Proof text goes here.
%%\qed
%%\end{proof}
%
%% \begin{acknowledgement}
%% If you want to include acknowledgments of assistance and the like at the end of an individual chapter please use the \verb|acknowledgement| environment -- it will automatically render Springer's preferred layout.
%% \end{acknowledgement}
%
%% \section*{Appendix}
%% \addcontentsline{toc}{section}{Appendix}
%
%% When placed at the end of a chapter or contribution (as opposed to at the end of the book), the numbering of tables, figures, and equations in the appendix section continues on from that in the main text. Hence please \textit{do not} use the \verb|appendix| command when writing an appendix at the end of your chapter or contribution. If there is only one the appendix is designated ``Appendix'', or ``Appendix 1'', or ``Appendix 2'', etc. if there is more than one.

%% \begin{equation}
%% a \times b = c
%% \end{equation}
% Problems or Exercises should be sorted chapterwise
%% \section*{Problems}
%% \addcontentsline{toc}{section}{Problems}
%
% Use the following environment.
% Don't forget to label each problem;
% the label is needed for the solutions' environment
%% \begin{prob}
%% \label{prob1}
%% A given problem or Excercise is described here. The
%% problem is described here. The problem is described here.
%% \end{prob}

%% \begin{prob}
%% \label{prob2}
%% \textbf{Problem Heading}\\
%% (a) The first part of the problem is described here.\\
%% (b) The second part of the problem is described here.
%% \end{prob}


