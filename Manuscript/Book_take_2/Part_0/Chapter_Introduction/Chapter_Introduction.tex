%!TEX root = ../../Heun_Dale_Haney_A_dynamic_approach_to_input_output_modeling.tex
%%%%%%%%%%%%%%%%%%%%% chapter.tex %%%%%%%%%%%%%%%%%%%%%%%%%%%%%%%%%
%
% sample chapter
%
% Use this file as a template for your own input.
%
%%%%%%%%%%%%%%%%%%%%%%%% Springer-Verlag %%%%%%%%%%%%%%%%%%%%%%%%%%
%\motto{Use the template \emph{chapter.tex} to style the various elements of your chapter content.}
\motto{Need a motto.~\emph{\cite[p.~26]{Berry1998}}

\hfill---\emph{Wendell Berry}}


%%%%%%%%%%%%%%%%%%%%%%%%%%%%%%%%%%
%%%%%%%%%% Introduction %%%%%%%%%%
%%%%%%%%%%%%%%%%%%%%%%%%%%%%%%%%%%
\chapter{Introduction: The end of an era}
% Always give a unique label
\label{chap:intro}
% use \chaptermark{}
% to alter or adjust the chapter heading in the running head
\chaptermark{Introduction}
%%%%%%%%%%%%%%%%%%%%%%%%%%%%%%%%%%
%%%%%%%%%%%%%%%%%%%%%%%%%%%%%%%%%%
%%%%%%%%%%%%%%%%%%%%%%%%%%%%%%%%%%


%% \abstract{Each chapter should be preceded by an abstract (10--15 lines long) that summarizes the content. The abstract will appear \textit{online} at \url{www.SpringerLink.com} and be available with unrestricted access. This allows unregistered users to read the abstract as a teaser for the complete chapter. As a general rule the abstracts will not appear in the printed version of your book unless it is the style of your particular book or that of the series to which your book belongs.\newline\indent
%% Please use the 'starred' version of the new Springer \texttt{abstract} command for typesetting the text of the online abstracts (cf. source file of this chapter template \texttt{abstract}) and include them with the source files of your manuscript. Use the plain \texttt{abstract} command if the abstract is also to appear in the printed version of the book.}

%% Use the template \emph{chapter.tex} together with the Springer document class SVMono (monograph-type books) or SVMult (edited books) to style the various elements of your chapter content in the Springer layout.

\abstract*{**** Re-write the abstract. ****
In this chapter we give our motivation for writing this book. 
We outline some of the models and subsequent metaphors 
that have been used to describe the economy---clockwork, 
machine, engine---and suggest a new metaphor---the 
metabolism of an organism.
We give an overview of Leontief Input-Output methods
and their extension to include energy and material inputs
and waste flows out of the economy.
We then propose a new Input-Output analysis method,
fitting to the new metaphor of the metabolic economy;
a dynamic accounting framework that includes accumulation of stocks
within economic sectors.}




%%%%%%%%%% Growth has stalled %%%%%%%%%%
\section{[BRH] Economic growth (growth rate of GDP) has stalled for mature (OECD, developed?) economies}
\label{sec:growth_has_slowed}
%%%%%%%%%%

In Chapter~\ref{chap:intro}.

\begin{figure}[!ht]
\centering\
\includegraphics[width=\linewidth]{Part_0/Chapter_Introduction/images/GDPPC.pdf}
\caption[The traditional model]{Source: Authors' calculations using 
data obtained from World Bank databank (Indicator NY.GDP.PCAP.KD.ZG accessed August 1, 2014.)
Source note: ``Annual percentage growth rate of GDP per capita based on constant local currency. Aggregates are based on constant 2005 U.S. dollars. GDP per capita is gross domestic product divided by midyear population. GDP at purchaser's prices is the sum of gross value added by all resident producers in the economy plus any product taxes and minus any subsidies not included in the value of the products. It is calculated without making deductions for depreciation of fabricated assets or for depletion and degradation of natural resources.''}
\label{fig:gdppc}
\end{figure}


%%%%%%%%%% Stalled growth is a problem %%%%%%%%%%
\section{[BRH] Stalling economic growth is a problem}
\label{sec:stall_is_a_problem}
%%%%%%%%%%

In Chapter~\ref{chap:intro}.


%%%%%%%%%% Endogenous factors %%%%%%%%%%
\section{[BRH] Endogenous factors}
\label{sec:endogenous_factors}
%%%%%%%%%%

In Chapter~\ref{chap:intro}.


%%%%%%%%%% Exogenous factors %%%%%%%%%%
\section{[MKH] Exogenous, biophysical factors}
\label{sec:exogenous_factors}
%%%%%%%%%%

As mentioned in Section~\ref{sec:stall_is_a_problem}, 
very little of the discourse 
about mature economy slowdown 
in mainstream economic circles
involves biophysical factors.%
	\footnote{
	In this context, we are using the term ``biophysical factors''
	to indicate any factor related to 
	the extraction, transport, processing, manipulation, and disposal 
	of the physical (as opposed to financial) manifestation 
	of any material or energy resource in the economy.
	}
Mainstream economics considers biophysical factors
to be \emph{exogenous} to the economy.%
	\footnote{
	Of course, mainstream economics discusses \emph{prices}
	of raw materials, goods, and services. 
	And, to the extent that biophysical factors affect prices,
	it could be said that mainstream economic discussions involve
	biophysical factors.
	However, biophysical factors are rarely acknowledged as causal 
	for establishing the prices of goods and services and the raw materials 
	of which they are comprised.
	}

Arguably, the most important (but not the only) biophysical factor 
vis-\`{a}-vis the economy is energy.
If we are to understand how exogenous factors cause economic slowdown
and, conversely, drive economic growth,
we would do well to understand how energy operates in the economy.
Thus, we first discuss the correlation 
between energy consumption and economic activity 
(Section~\ref{sec:energy-economy_coupling}).
Then, we show how economic demands for energy and materials
are related to important stocks 
of raw materials and energy carriers in the biosphere 
(Section~\ref{sec:stall_non-renewable_stocks})
and the stocks of manufactured capital in the economy
(Section~\ref{sec:stall_capital_stock}).


%+++++++++ Energy-economy coupling ++++++++++
\subsection{Coupling between energy and the economy}
\label{sec:energy-economy_coupling}
%+++++++++

All manufactured goods are made and services provided
from raw materials that have been
manipulated, processed, transported, or otherwise transformed by energy.
Indeed, energy consumption and economy activity are highly correlated,
as Cleveland, et.\ al.\ showed in 1984. 
(See Figure~\ref{fig:Cleveland1984}.)

\begin{figure}[!ht]
\centering\
\includegraphics[width=\linewidth]{Part_0/Chapter_Introduction/images/Cleveland1984.pdf}
\caption[Energy and economic activity]{The famous graph from Cleveland, et.\ al.\
\cite{Cleveland:1984aa} showing the strong correlation 
between energy and economic activity from 1890 to 1982.
**** Need to obtain permission to use this graph? ****}
\label{fig:Cleveland1984}
\end{figure}

Because of the high correlation between energy consumption and economic activity,
it stands to reason that energy shortage relative to demand will hinder economic activity.
Of course, there are degrees of shortage. 
In extreme cases, goods become hard to find 
as observed in the US during 1970s oil crisis.
(See Figure~\ref{fig:gas_shortage}.)

\begin{figure}[!ht]
\centering\
\includegraphics[width=\linewidth]{Part_0/Chapter_Introduction/images/gas_shortage_1973.jpg}
\caption[Gasoline shortage]{Gasoline scarcity (in the vernacular sense) in 1973.
**** We probably don't need to obtain permission to use this photograph, because
it is from the US national archives.
\url{http://arcweb.archives.gov/arc/action/ExternalIdSearch?id=548053}
\url{https://www.flickr.com/photos/usnationalarchives/4272321708/in/set-72157623204210352/}}
****
\label{fig:gas_shortage}
\end{figure}

In mild cases, and in the absence of price controls, 
shortage of any good relative to demand leads to increasing prices,
even when goods remain available.
An oil price spike in the 2003--2009 timeframe
was caused by demand increasing faster than production.

In both cases (1970s and 2000s), 
significant slowing of economic activity (recessions)
followed the oil shortages and prices spikes.
These were not isolated cases.
Hamilton noted that 
10 of the 11 US postwar recessions 
involved the same pattern.\cite[p.~45]{Hamilton:2013vc}
It is clear that 
there is a correlation between energy consumption and economic activity.

But, what are the dynamics that cause economic slowdowns 
to follow energy price spikes?
When prices rise faster than the cost of production, 
the profit motive should, according to economic theory, induce 
new firms to enter the market and
established firms to increase production.
However, the timing of supply and demand events is crucial.
If firms don't increase production to meet demand, 
prices will remain elevated.
In the absence of increased production, falling demand will 
bring prices back to earth.

In terms of energy, and oil in particular, 
the \emph{rate} at which production can be increased 
is of the utmost importance, and
there are physical and technological limits. 
For example, increasing worldwide oil production by, say, 20\% would involve
finding additional oil deposits, drilling additional wells, installing new pumps,
and expanding transport and delivery infrastructure worldwide.
Perhaps that could be accomplished in a 5-year timeframe, 
but it would be physically and technologically impossible in a 3-month timeframe.
In, say, a 2-year timeframe, it might be physically and technologically possible
to increase oil production by 20\%, 
but it would be extremely costly to do so,
and the profit motive would evaporate.
Thus, temporal considerations 
layered upon physical and technological constraints 
become the ties that bind the financial to the biophysical.
Put another way, temporal considerations are the point at which the economy 
becomes coupled to the biosphere.

% These dynamics help to explain several aspects of the links between energy and the economy.
% First, an inability to increase energy production to meet demand
% implies a constraint on economic activity.
% Second, rapid price spikes
% (indicating quickly-increasing scarcity due to either rapidly increasing demand or
% rapidly falling availability)
% provide a more-severe economic impact.

Recently, a few authors have found that \emph{energy cost share} 
is an explanatory variable for energy-economy dynamics.
Energy cost share ($f_E$) is defined, for any period of time, as
%
\begin{equation}
	f_E \equiv \frac{\displaystyle\sum_i p_i Q_i}{GDP} \; ,
\end{equation}
%
where 
the subscript $i$ indicates types of energy 
(electricity, gasoline, natural gas, etc.),
$p$ indicates the price of energy,
$Q$ indicates the quantity of energy purchased within the economy, and
$GDP$ is gross domestic product.

To our knowledge, 
Bashmakov~\cite{Bashmakov:2007ek} was the first to 
identify a long-term sustainable range for energy cost share
in mature economies.
He also showed that developed economies 
can sustain high total energy cost share ($f_E$) 
for a short period of time 
(possibly 2--3 years) 
before recessionary pressures 
destroy energy demand,%
	\footnote{
	Note that ``destruction of energy demand'' 
	is accomplished through recession
	in the short run.
	}
stimulate energy efficiency,%
	\footnote{
	Like increasing oil production, 
	increasing energy efficiency also has 
	physical and technological limits.
	Increased energy efficiency is a medium- to long-term process. 
	}
reduce energy prices, 
and return total energy cost share to its long-term sustainable range.
On the other hand, reduction of total energy cost share below 
a lower bound provides economic stimulus, 
increases energy demand, 
provides upward pressure on energy prices, 
and returns energy cost share to its long-term sustainable range.
Bashmakov speculates that 
``energy affordability thresholds and behavioral constants'' 
are responsible for the stable range of energy cost share 
over many decades.\cite[p.~3585]{Bashmakov:2007ek} 
 
The long-term stable range for economy-wide energy cost share 
is 8--10\% for the US and 9--11\% for the OECD. 
The stable and narrow range of energy cost share 
for final consumers in the U.S. is 4--5\% and in the OECD is 4.5--5.5\%.
The oil cost share threshold that correlates with US recessions 
is about 5.5\%.\cite{Murphy:2011jh}

The picture emerging from this research shows that 
the cost share of energy in the economy
(and, perhaps more narrowly, oil cost share)
is an important factor in stimulating or restraining economic growth,
despite its small value (typically, less than 10\%).%
	\footnote{
	Embarking on an economic growth path
	appears to reduce the energy cost share in an economy from very high values
	(indicating that nearly all economic activity is focused on procuring energy)
	to small values that remain within a stable range.
	For example, Sweden's energy cost share has stabilized at 12\% since 1970,
	although it was nearly 100\% in 1800.\cite{Stern:2012ey}.
	}
It appears that the economy-biosphere system has 
a built-in feedback mechanism that 
enforces alignment between biophysical limits and the economy.

This may be somewhat surprising to mainstream economics, 
which ascribes importance 
based on \emph{financial} information, 
not \emph{physical} information. 
Indeed, the cost share of energy in mature economies is low, 
and viewing energy as relatively unimportant is justified if
one's view of ``importance'' is limited to financial information only.
But, many have noted that the physical importance of energy to the economy 
far exceeds its cost share.\cite{Ayres:2013aa}
And, as discussed above, because the biophysical world is coupled 
to the economy through temporal considerations, 
the physical importance of energy far exceeds its financial importance.
Eventually, tight coupling between energy and the economy 
has economic impacts that are difficult to see (when you're focused on financials only)
but impossible to ignore (when recessions destroy economic value).


%+++++++++ Stall related to non-renewable stocks ++++++++++
\subsection{[MKH] Stall is related to non-renewable stocks}
\label{sec:stall_non-renewable_stocks}
%+++++++++

Given the tight coupling between the biosphyical world and the economy,
especially regarding energy,
discussed in Section~\ref{sec:energy-economy_coupling} above,
it is prudent to consider the economic role that important biosphere stocks
of materials and energy play in the economy.

The Best First Principle~\cite{Cleveland:2008aa}
indicates that the economy will extract the easiest-to-obtain 
stocks of mineral and energy resources first.
``Best'' can be measured in several ways, usually related to cost.
For example, inexpensive-to-obtain West Texas crude oil was extracted
before expensive-to-obtain offshare oil. 
Surface deposits of gold and diamonds are exhausted before subsurface
veins and pipes are exploited.
High-purity mineral deposits are exploited before low-purity deposits.
As a result, it becomes more ``difficult'' to continually increase
extraction rates as time proceeds.
Historical oil production trends reflect these realities.
Through time, the annual rate of increase of oil production
has declined from 7.8\% to 0.7\% (Figure~\ref{fig:oil_production}).

\begin{figure}[!ht]
\centering\
\includegraphics[width=\linewidth]{Part_0/Chapter_Introduction/images/growth-in-world-oil-supply.png}
\caption[World Oil Supply]{Slowing growth in world oil supply.
\url{http://ourfiniteworld.com/2013/10/02/our-oil-problems-are-not-over/}
**** Becky--can you obtain this data and plot it similarly? ****
}
\label{fig:oil_production}
\end{figure}

It is important to realize that it takes energy to make energy available to society.
Oil production requires energy for the ongoing
operation of pumps, 
transportation of crude to the refinery,
refinement of crude to useable products, and 
transportation of refined products to consumers and firms.
In addition, it take energy to manufacture the wells and pumps, 
tankers and pipelines, and
refineries used in oil production processes.

Application of the Best First Principle to the energy production process 
indicates that it will take more energy 
to make energy available to society as time proceeds.
The metric that measures the energy impacts of the Best First Principle is 
Energy Return on Investment (EROI), which is defined as 
%
\begin{equation}
	EROI \equiv \frac{Q_p}{Q_i} \; ,
\end{equation}
%
where $Q_p$ is the rate of energy made available to society
and $Q_i$ is the rate of energy consumed in the energy production process.
As time proceeds, the Best First Principle suggests that
$EROI$ will decline.
Indeed, that is true.
$EROI$ for oil has declined 
from a value of 100 in the 1930s~\cite[p.~781]{Cleveland:2005uy} 
to around 20 today.\cite[Fig.~2]{Hall:2014aa}
Furthermore, declining oil $EROI$ has economic impacts.
Both Heun and de~Wit~\cite{Heun:2012ek} and King and Hall~\cite{King:2011go}
show that declining $EROI$ correlates with higher prices for oil, 
because declining $EROI$ provides upward pressure on 
production costs, and therefore, prices
as time proceeds.

Technological advances can bring about some improvements 
in the race against the Best First Principle.
For example, the recent shale oil boom has increased US oil production
significantly. 
But, Figure~\ref{fig:US_oil_production} shows that increased US production
is coming from so-called ``tight'' oil (which includes shale oil), 
which has lower $EROI$ than crude oil.
Furthermore, today's ``tight'' oil is more expensive to produce 
than the crude of yesteryear.
Consequently, oil prices must remain high for 
shale production to continue into the foreseeable future. 
Unfortunately, Section~\ref{sec:energy-economy_coupling} indicates that
high energy prices can lead to high energy cost share in the economy
and recessionary pressure.

\begin{figure}[!ht]
\centering\
\includegraphics[width=\linewidth]{Part_0/Chapter_Introduction/images/us-crude-oil-production-including-tight-oil.png}
\caption[US oil production]{US oil production.
\url{http://ourfiniteworld.com/2014/07/23/world-oil-production-at-3312014-where-are-we-headed/}
**** Becky--can you obtain this data and plot it similarly? ****
}
\label{fig:US_oil_production}
\end{figure}

The fact that scarcity of crude oil provides incentives for shale oil production
appears, at first glace, to be a good thing.
Substitution of a plentiful raw material for a scarce one 
can be seen as evidence that the economy is working as it should.
But, the benefits of shale oil are modest, at best, when the
high financial and energy costs of production are considered.

We see this, especially, when examining the financial situation of oil producers.
Figure~\ref{fig:oil_company_free_cash_flow} 
shows that despite the recent increase in oil production
and continued high prices, 
the free cash flow of independent oil producers is negative.
It remains to be seen how independent producers can continue advancing 
production without free cash flow to cover capital expenditures.
One possible cure is higher oil prices.
But, again, we saw 
in Section~\ref{sec:energy-economy_coupling}
that high energy cost share 
(which could be caused by high energy prices)
provides recessionary pressure.

\begin{figure}[!ht]
\centering\
\includegraphics[width=\linewidth]{Part_0/Chapter_Introduction/images/Cash-Flow.jpg}
\caption[Oil company free cash flow]{Oil company free cash flow.
\url{http://blogs.platts.com/2014/07/30/peak-oil-forecasts/}
**** Becky--can you obtain this data and plot it similarly? ****
}
\label{fig:oil_company_free_cash_flow}
\end{figure}

All of this comes about simply because it is 
more physically ``difficult,'' and, as a consequence, 
more financially expensive, 
to extract oil today than it was 10, 20, 30, and 100 years ago.
It is more difficult to obtain oil today because we have depleted
the stock of easy-to-obtain crude oil from the biosphere.
And, the remaining stock is lower quality (e.g., shale)
or further away (e.g., deeper offshore).

Using oil as our example, we observe that 
stocks of natural capital, especially energy resources,
have significant economic implications.
Both the declining \emph{quantity} and 
the diminishing \emph{quality} of the remaining non-renewable stocks 
are contributing to the growth slowdown in mature economies.


**** [bonepile?] The implication is that solving the resource efficiency problem
is the key to continued growth. [Reference Thomas Friedman here,
\url{http://www.nytimes.com/2012/03/04/opinion/sunday/friedman-take-the-subway.html?hp&_r=0}]
****


%+++++++++ Stall related to capital stock ++++++++++
\subsection{Stall is related to capital stock}
\label{sec:stall_capital_stock}
%+++++++++

In Chapter~\ref{chap:intro}.


%%%%%%%%%% Policy focused on flows %%%%%%%%%%
\section{[BRH] Policy solutions focus on flows}
\label{sec:policy_flows}
%%%%%%%%%%

In Chapter~\ref{chap:intro}.


%%%%%%%%%% Consumption-driven policies are unsustainable %%%%%%%%%%
\section{[MKH] Consumption-driven policies are unsustainable}
\label{sec:consumption_unsustainable}
%%%%%%%%%%

In Chapter~\ref{chap:intro}.


%%%%%%%%%% Don't understand real economy %%%%%%%%%%
\section{[MKH] We do not fully understand how the real economy operates}
\label{sec:dont_understand_real_economy}
%%%%%%%%%%

In Chapter~\ref{chap:intro}.


%%%%%%%%%% Prescriptions worse than disease %%%%%%%%%%
\section{[MKH] Prescriptions are worse than the disease}
\label{sec:prescriptions_disease}
%%%%%%%%%%

In Chapter~\ref{chap:intro}.


%%%%%%%%%% Change needed %%%%%%%%%%
\section{[MKH] Change is needed}
\label{sec:change_needed}
%%%%%%%%%%

In Chapter~\ref{chap:intro}.








\bibliographystyle{unsrt}
\bibliography{../../Metabolic}


% Always give a unique label
% and use \ref{<label>} for cross-references
% and \cite{<label>} for bibliographic references
% use \sectionmark{}
% to alter or adjust the section heading in the running head
%% Instead of simply listing headings of different levels we recommend to let every heading be followed by at least a short passage of text. Furtheron please use the \LaTeX\ automatism for all your cross-references and citations.

%% Please note that the first line of text that follows a heading is not indented, whereas the first lines of all sequent paragraphs are.

%% Use the standard \verb|equation| environment to typeset your equations, e.g.
%
%% \begin{equation}
%% a \times b = c\;,
%% \end{equation}
%
%% however, for multiline equations we recommend to use the \verb|eqnarray|
%% environment\footnote{In physics texts please activate the class option \texttt{vecphys} to depict your vectors in \textbf{\itshape boldface-italic} type - as is customary for a wide range of physical jects.}.
%% \begin{eqnarray}
%% a \times b = c \nonumber\\
%% \vec{a} \cdot \vec{b}=\vec{c}
%% \label{eq:01}
%% \end{eqnarray}

%% \section{section Heading}
%% \label{sec:2}
%% Instead of simply listing headings of different levels we recommend to let every heading be followed by at least a short passage of text. Furtheron please use the \LaTeX\ automatism for all your cross-references\index{cross-references} and citations\index{citations} as has already been described in Sect.~\ref{sec:2}.

%% \begin{quotation}
%% Please do not use quotation marks when quoting texts! Simply use the \verb|quotation| environment -- it will automatically render Springer's preferred layout.
%% \end{quotation}


%% \section{section Heading}
%% Instead of simply listing headings of different levels we recommend to let every heading be followed by at least a short passage of text. Furtheron please use the \LaTeX\ automatism for all your cross-references and citations as has already been described in Sect.~\ref{sec:2}, see also Fig.~\ref{fig:1}\footnote{If you copy text passages, figures, or tables from other works, you must obtain \textit{permission} from the copyright holder (usually the original publisher). Please enclose the signed permission with the manucript. The sources\index{permission to print} must be acknowledged either in the captions, as footnotes or in a separate section of the book.}

%% Please note that the first line of text that follows a heading is not indented, whereas the first lines of all sequent paragraphs are.

% For figures use
%
%% \begin{figure}[b]
%% \sidecaption
% Use the relevant command for your figure-insertion program
% to insert the figure file.
% For example, with the option graphics use
%% \includegraphics[scale=.65]{figure}
%
% If not, use
%\picplace{5cm}{2cm} % Give the correct figure height and width in cm
%
%% \caption{If the width of the figure is less than 7.8 cm use the \texttt{sidecapion} command to flush the caption on the left side of the page. If the figure is positioned at the top of the page, align the sidecaption with the top of the figure -- to achieve this you simply need to use the optional argument \texttt{[t]} with the \texttt{sidecaption} command}
%% \label{fig:1}       % Give a unique label
%% \end{figure}


%% \paragraph{Paragraph Heading} %
%% Instead of simply listing headings of different levels we recommend to let every heading be followed by at least a short passage of text. Furtheron please use the \LaTeX\ automatism for all your cross-references and citations as has already been described in Sect.~\ref{sec:2}.

%% Please note that the first line of text that follows a heading is not indented, whereas the first lines of all sequent paragraphs are.

%% For typesetting numbered lists we recommend to use the \verb|enumerate| environment -- it will automatically render Springer's preferred layout.

%% \begin{enumerate}
%% \item{Livelihood and survival mobility are oftentimes coutcomes of uneven socioeconomic development.}
%% \begin{enumerate}
%% \item{Livelihood and survival mobility are oftentimes coutcomes of uneven socioeconomic development.}
%% \item{Livelihood and survival mobility are oftentimes coutcomes of uneven socioeconomic development.}
%% \end{enumerate}
%% \item{Livelihood and survival mobility are oftentimes coutcomes of uneven socioeconomic development.}
%% \end{enumerate}


%% \paragraph{paragraph Heading} In order to avoid simply listing headings of different levels we recommend to let every heading be followed by at least a short passage of text. Use the \LaTeX\ automatism for all your cross-references and citations as has already been described in Sect.~\ref{sec:2}, see also Fig.~\ref{fig:2}.

%% Please note that the first line of text that follows a heading is not indented, whereas the first lines of all sequent paragraphs are.

%% For unnumbered list we recommend to use the \verb|itemize| environment -- it will automatically render Springer's preferred layout.

%% \begin{itemize}
%% \item{Livelihood and survival mobility are oftentimes coutcomes of uneven socioeconomic development, cf. Table~\ref{tab:1}.}
%% \begin{itemize}
%% \item{Livelihood and survival mobility are oftentimes coutcomes of uneven socioeconomic development.}
%% \item{Livelihood and survival mobility are oftentimes coutcomes of uneven socioeconomic development.}
%% \end{itemize}
%% \item{Livelihood and survival mobility are oftentimes coutcomes of uneven socioeconomic development.}
%% \end{itemize}

%% \begin{figure}[t]
%% \sidecaption[t]
% Use the relevant command for your figure-insertion program
% to insert the figure file.
% For example, with the option graphics use
%% \includegraphics[scale=.65]{figure}
%
% If not, use
%\picplace{5cm}{2cm} % Give the correct figure height and width in cm
%
%% \caption{Please write your figure caption here}
%% \label{fig:2}       % Give a unique label
%% \end{figure}

%% \runinhead{Run-in Heading Boldface Version} Use the \LaTeX\ automatism for all your cross-references and citations as has already been described in Sect.~\ref{sec:2}.

%% \runinhead{Run-in Heading Italic Version} Use the \LaTeX\ automatism for all your cross-refer\-ences and citations as has already been described in Sect.~\ref{sec:2}\index{paragraph}.
% Use the \index{} command to code your index words
%
% For tables use
%
%% \begin{table}
%% \caption{Please write your table caption here}
%% \label{tab:1}       % Give a unique label
%
% For LaTeX tables use
%
%% \begin{tabular}{p{2cm}p{2.4cm}p{2cm}p{4.9cm}}
%% \hline\noalign{\smallskip}
%% Classes & class & Length & Action Mechanism  \\
%% \noalign{\smallskip}\svhline\noalign{\smallskip}
%% Translation & mRNA$^a$  & 22 (19--25) & Translation repression, mRNA cleavage\\
%% Translation & mRNA cleavage & 21 & mRNA cleavage\\
%% Translation & mRNA  & 21--22 & mRNA cleavage\\
%%Translation & mRNA  & 24--26 & Histone and DNA Modification\\
%%\noalign{\smallskip}\hline\noalign{\smallskip}
%%\end{tabular}
%%$^a$ Table foot note (with superscript)
%%\end{table}
%
%% \section{Section Heading}
%%\label{sec:3}
% Always give a unique label
% and use \ref{<label>} for cross-references
% and \cite{<label>} for bibliographic references
% use \sectionmark{}
% to alter or adjust the section heading in the running head
%% Instead of simply listing headings of different levels we recommend to let every heading be followed by at least a short passage of text. Furtheron please use the \LaTeX\ automatism for all your cross-references and citations as has already been described in Sect.~\ref{sec:2}.

%% Please note that the first line of text that follows a heading is not indented, whereas the first lines of all sequent paragraphs are.

%%If you want to list definitions or the like we recommend to use the Springer-enhanced \verb|description| environment -- it will automatically render Springer's preferred layout.

%%\begin{description}[Type 1]
%%\item[Type 1]{That addresses central themes pertainng to migration, health, and disease. In Sect.~\ref{sec:1}, Wilson discusses the role of human migration in infectious disease distributions and patterns.}
%%\item[Type 2]{That addresses central themes pertainng to migration, health, and disease. In Sect.~\ref{sec:2}, Wilson discusses the role of human migration in infectious disease distributions and patterns.}
%%\end{description}

%%\section{section Heading} %
%% In order to avoid simply listing headings of different levels we recommend to let every heading be followed by at least a short passage of text. Use the \LaTeX\ automatism for all your cross-references and citations citations as has already been described in Sect.~\ref{sec:2}.

%% Please note that the first line of text that follows a heading is not indented, whereas the first lines of all sequent paragraphs are.

%% \begin{svgraybox}
%% If you want to emphasize complete paragraphs of texts we recommend to use the newly defined Springer class option \verb|graybox| and the newly defined environment \verb|svgraybox|. This will produce a 15 percent screened box 'behind' your text.

%% If you want to emphasize complete paragraphs of texts we recommend to use the newly defined Springer class option and environment \verb|svgraybox|. This will produce a 15 percent screened box 'behind' your text.
%% \end{svgraybox}


%% \section{section Heading}
%%Instead of simply listing headings of different levels we recommend to let every heading be followed by at least a short passage of text. Furtheron please use the \LaTeX\ automatism for all your cross-references and citations as has already been described in Sect.~\ref{sec:2}.

%% Please note that the first line of text that follows a heading is not indented, whereas the first lines of all sequent paragraphs are.

%% \begin{theorem}
%% Theorem text goes here.
%% \end{theorem}
%
% or
%
%% \begin{definition}
%% Definition text goes here.
%% \end{definition}

%% \begin{proof}
%\smartqed
%% Proof text goes here.
%% \qed
%% \end{proof}

%%\paragraph{Paragraph Heading} %
%% Instead of simply listing headings of different levels we recommend to let every heading be followed by at least a short passage of text. Furtheron please use the \LaTeX\ automatism for all your cross-references and citations as has already been described in Sect.~\ref{sec:2}.

%% Note that the first line of text that follows a heading is not indented, whereas the first lines of all subsequent paragraphs are.
%
% For built-in environments use
%
%%\begin{theorem}
%%Theorem text goes here.
%%\end{theorem}
%
%%\begin{definition}
%%Definition text goes here.
%%\end{definition}
%
%%\begin{proof}
%%\smartqed
%% Proof text goes here.
%%\qed
%%\end{proof}
%
%% \begin{acknowledgement}
%% If you want to include acknowledgments of assistance and the like at the end of an individual chapter please use the \verb|acknowledgement| environment -- it will automatically render Springer's preferred layout.
%% \end{acknowledgement}
%
%% \section*{Appendix}
%% \addcontentsline{toc}{section}{Appendix}
%
%% When placed at the end of a chapter or contribution (as opposed to at the end of the book), the numbering of tables, figures, and equations in the appendix section continues on from that in the main text. Hence please \textit{do not} use the \verb|appendix| command when writing an appendix at the end of your chapter or contribution. If there is only one the appendix is designated ``Appendix'', or ``Appendix 1'', or ``Appendix 2'', etc. if there is more than one.

%% \begin{equation}
%% a \times b = c
%% \end{equation}
% Problems or Exercises should be sorted chapterwise
%% \section*{Problems}
%% \addcontentsline{toc}{section}{Problems}
%
% Use the following environment.
% Don't forget to label each problem;
% the label is needed for the solutions' environment
%% \begin{prob}
%% \label{prob1}
%% A given problem or Excercise is described here. The
%% problem is described here. The problem is described here.
%% \end{prob}

%% \begin{prob}
%% \label{prob2}
%% \textbf{Problem Heading}\\
%% (a) The first part of the problem is described here.\\
%% (b) The second part of the problem is described here.
%% \end{prob}


