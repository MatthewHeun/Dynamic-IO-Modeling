%!TEX root = ../../Heun_Dale_Haney_A_dynamic_approach_to_input_output_modeling.tex
%%%%%%%%%%%%%%%%%%%%% chapter.tex %%%%%%%%%%%%%%%%%%%%%%%%%%%%%%%%%
%
% sample chapter
%
% Use this file as a template for your own input.
%
%%%%%%%%%%%%%%%%%%%%%%%% Springer-Verlag %%%%%%%%%%%%%%%%%%%%%%%%%%
%\motto{Use the template \emph{chapter.tex} to style the various elements of your chapter content.}
\motto{The economic light is brightest under the lamppost of the market, 
but neither drunks nor statisticians should confine their search there. 
In extending the accounts, 
we must endeavor to find dimly lit information outside our old boundaries of search, 
particularly when the activities are of great value 
to the nation.~\emph{\cite[p.~23]{Nordhaus1999a}}

\hfill---\emph{William Nordhaus}}

% There are several commands here that change things dramatically.
% So, we put these inside braces to restrict the scope
% to only this area.
{
	% Tell bibTeX to not include "Chapter" at the start of the chapter
	\renewcommand{\chaptername}{}

	% Tell bibTeX to not include the chapter number (in this case "10")
	% at the start of the chapter.
	\renewcommand{\thechapter}{}

	\chapter{Preface}

}

% use \chaptermark{}
% to alter or adjust the chapter heading in the running head
\chaptermark{Preface}
% Always give a unique label
\label{chap:preface}

% \abstract{} % No abstract for the prologue.

%% \abstract{Each chapter should be preceded by an abstract (10--15 lines long) that summarizes the content. The abstract will appear \textit{online} at \url{www.SpringerLink.com} and be available with unrestricted access. This allows unregistered users to read the abstract as a teaser for the complete chapter. As a general rule the abstracts will not appear in the printed version of your book unless it is the style of your particular book or that of the series to which your book belongs.\newline\indent
%% Please use the 'starred' version of the new Springer \texttt{abstract} command for typesetting the text of the online abstracts (cf. source file of this chapter template \texttt{abstract}) and include them with the source files of your manuscript. Use the plain \texttt{abstract} command if the abstract is also to appear in the printed version of the book.}

%% Use the template \emph{chapter.tex} together with the Springer document class SVMono (monograph-type books) or SVMult (edited books) to style the various elements of your chapter content in the Springer layout.

The vast majority of physical scientists are concerned that 
quality of life may very well decrease in the future. 
Limits to the amounts and rates at which resources can be extracted, 
the rate that wastes (including anthropogenic carbon emissions) 
can be assimilated by the biosphere, and
the rate at which human ingenuity can substitute 
for depletion of natural capital are already being encountered. 
The future health and viability of all economies is at risk.\cite{IPCC2013} 
Conversely, the vast majority of economists and policy-makers predict 
that quality of life into the future will continue to improve. 
Economists point out that standards of living have increased steadily over time,
and living standards for even the poorest nations 
are ``accelerating markedly.''~\cite{Malik:2013aa} 
GDP per capita is expected to
grow continuously and quality of life will continue to improve. 
Graphs that extrapolate GDP per capita into the future show increasing
trends under even the most pessimistic assumptions.~\cite[p. 170]{Malik:2013aa} 
The OECD, for example, forecasts an average global
growth rate in GDP of approximately 2\%/year 
for the next several decades.\cite[Table A.1]{OECD2012}

There is a stark contrast between these two visions of the future, 
because the two groups (scientists and economists) focus on 
different parts of the economy: 
scientists observe that the planet's natural capital is dwindling, 
which should lead to a declining quality of life.
While, conversely, economists observe that the stock of manufactured capital is growing, 
and growing increasingly efficient, which
should continue to provide an increasing level of income and quality of life. 

The differences between scientists and economists revolve around 
the understanding and role of capital.
Economists understand that manufactured capital is a primary ``factor of production,'' 
but they overlook the role of natural capital. 
In contrast, physical scientists often focus on natural capital but ignore manufactured capital. 
In fact, natural capital is required to produce, operate, and maintain manufactured capital. 
And, a direct correlation exists between the efficiency of manufactured capital
and economic quality of life. 
Both types of capital are important parts of the production process, 
because both forms of capital provide wealth that generates income. 
Furthermore, both natural and manufactured capital depreciate over time as they are used.
Manufactured capital ``wears out,'' thereby reducing production capacity. 
And, we ``use up'' natural capital when forests are cut
down, fossil fuels are depleted, clean air and water are polluted, 
and wetlands are degraded. 
As natural capital dwindles, the future
capacity for income generation also dwindles. 
Such depletion of natural capital is the primary concern of many scientists, 
and it should be concerning to everyone!

Systems of National Accounts (SNAs) gather, evaluate, and disseminate 
economic statistics concerning income and assets. 
The focus of national accounting centers on income. 
Over the last two hundred years, the stock of manufactured capital has
dramatically increased, and had continued to grow increasingly efficient. 
Human ingenuity has provided replacements for much of the
natural capital that has been used up. 
This has possibly lulled nations into believing that they 
do not need to manage their portfolio of capital assets. 
Natural capital is not included in SNAs. 
Even manufactured capital is given little ``shelf space.'' 
Instead, the spotlight shines on GDP as the measure of a nation’s economic condition. 
This predilection results in SNAs that collect and analyze a trove of data to
produce a robust \emph{income statement} for the economy (GDP)
yet mostly ignore the data needed to produce a similarly rigorous
\emph{balance sheet} that tracks the value 
of a nations' wealth (manufactured and natural capital). 
Without a complete national balance sheet alongside an income statement, 
policy-makers can unwittingly draw down a nation’s wealth (natural capital) 
to generate today’s income (GDP). 
In so doing, future income is put at risk. 

The UN \emph{Inclusive Wealth Report 2012}~\cite{IWR2012} 
accounts for all forms of productive capital (natural, human, and manufactured). 
The \emph{Inclusive Wealth Report} shows that, in fact, 
a nation’s wealth can decline even as its GDP grows. 
Between 1990--2008 GDP per capita for the US grew on average 1.8\%, 
while the nation's inclusive wealth grew at only 0.7\% per year. 
For other nations, inclusive wealth declined. 
Saudi Arabia's GDP per capita grew at 0.4\% per year, 
while its inclusive wealth declined at a rate of 1.1\% per year. 
In an extreme case, Nigeria's GDP per capita grew at 2.5\% per year, 
while its wealth declined at a rate of 1.8\% per year.~\cite[p. 44]{IWR2012}  
Indeed, Saudi Arabia, Russia, Venezuela, South Africa, and Nigeria 
all appear to be consuming their wealth (and that of future generations) 
to support the consumption of the current generation.

Given the above, we contend that a nation needs both 
a balance sheet \emph{and} 
an income statement 
to ensure its own sustainability. 
Nations must monitor and manage not only the goods and services it produces, 
but the stocks of capital that are used to produce them. 
Many questions are unanswerable without both:

\begin{enumerate}
    \item{How is fossil fuel dependency embedded in the interlocking fabric of the economy?} 
    \item{How will industries that are dependent on coal, oil, 
 			and other forms of non-renewable energy transition 
			to renewable forms of energy?} 
    \item{How will industries that are dependent 
 			on the industries that are dependent on non-renewable energy 
			be affected by the transition?} 
\end{enumerate}

We're not the first to point out the importance of accounting 
for both manufactured and natural capital. 
The Brundtland Commission (1983--1987) recognized the need 
to devise rigorous methods for integrating environmental dimensions 
into national balance sheets and income statements.

\begin{quote}
	The process of economic development must be more soundly 
	based upon the realities of the stock of capital that sustains it. 
	This is rarely done in either developed or developing countries. 
	For example, income from forestry operations is conventionally measured 
	in terms of the value of timber and other products extracted, 
	minus the costs of extraction. 
	The costs of regenerating the forest are not taken into account, 
	unless money is actually spent on such work. 
	Thus figuring profits from logging rarely takes full account 
	of the losses in future revenue incurred through degradation of the forest. 
	Similar incomplete accounting occurs in the exploitation 
	of other natural resources, 
	especially in the case of resources that are not capitalized 
	in enterprise or national accounts: air, water, and soil. 
	In all countries, rich or poor, 
	economic development must take full account 
	in its measurements of growth of the improvement or 
	deterioration in the stock of natural resources.\cite[Chapter 2, para 36]{brundtland1987}
\end{quote}

In response to the call by the Brundtland Commission, 
economist Peter Bartelmus headed the effort at the UN Statistics Division 
to develop the 
System for Integrated Environmental and Economic Accounting~(SEEA).\cite{Bartelmus1991} 
This set of ``satellite accounts'' operates alongside and dovetails with 
the international standard of System of National Accounts framework for measuring GDP. 
The UN published the Handbook for the SEEA in 1993.\cite{UNSEEA1993} 
The Philippines implemented SEEA in its national accounts 
at first as part of a 1993 pilot study, 
and then permanently.\cite{uno1998, PhilippinesSEEAWeb}
SEEA is currently in its third revision, 
but has been implemented 
in only a few countries to date.\cite{UNSEEAWeb,PhilippinesSEEAWeb} 

Shortly after the publication 
of the UN integrated environmental-economic accounting methodology, 
the US Bureau of Economic Analysis (BEA) began development of a similar
Integrated Environmental-Economic System of Accounts (IEESA) framework. 
The motivation, methodology, and first set of data tables were published 
in April, 1994.\cite{BEA1994a} 
These accounts provided a range of values 
for the the stocks of subsoil mineral assets in the nation’s portfolio. 

Unfortunately, the progress toward environmental accounting 
in the US came to a screeching halt as soon as the tables were published. 
The US Congress responded swiftly and negatively 
to the publication of the first set of IEESA tables. 
Congress explicitly restricted the BEA from spending any more 
of their budget on developing or extending 
the integrated environmental and economic accounting methodology. 
In fact, the BEA was directed 
to turn its entire IEESA budget over to panel 
of members of the National Academy of Sciences 
to evaluate the BEA’s objectivity and methodology 
related to this work and submit a report to Congress.

\begin{quote}
The conferees understand that there has been considerable debate 
over the years as to the objectivity, methodology, and applicability 
of ``Integrated Environmental-Economic Accounting'' or ``Green GDP.''
The conferees understand that the Department 
has completed the development of Phase I of this initiative. 
The conferees believe that an independent review, 
by an external organization such as the National Academy of Sciences, 
should be conducted to analyze 
the proposed objectivity, methodology, and application of environmental accounting. 
The conferees expect BEA to use \$400,000 under this account 
to fund this independent study, as suggested by the House report. 
The conferees expect BEA to suspend development 
of Phase II of this initiative 
until the review has been completed and the results have been submitted 
to the Committees on Appropriations of the House and the Senate, 
as well as the appropriate authorizing committees.\emph{FY 1995 HR 103-708}
\end{quote}

William Nordhaus chaired the panel of scientists 
to evaluate whether the BEA should extend 
the national income and product accounts to include 
``assets and production activities associated 
with natural resources and the environment.''~\cite[p. 2]{ Nordhaus1999a} 
In 1999, the panel submitted its comprehensive report 
to Congress strongly recommending that the BEA be authorized 
to continue producing the environmental-economic national accounts.\cite{ Nordhaus1999a}

The report illuminates the need for a nation 
to keep ``comprehensive economic accounts'' that ``provide a complete reckoning of
economic activity, whether it takes place inside or outside 
the boundary of the marketplace.''~\cite[p. 29]{ Nordhaus1999a}
The data are used by states, local governments, businesses and investors alike 
to make sound economic decisions, the report emphasizes. 
Should the timber from an old growth forest be harvested?
Using the data that are limited to income generating transactions only, the answer is yes. 
The harvesting of timber adds directly to national income. 
The value of the flow of ``hunting, fishing, and other forms of nonmarket forest recreation'' 
services foregone that likely outweighs the value of the harvested timber 
is not available to be part of the decision.\cite[p. 30]{ Nordhaus1999a}

Although the BEA is no longer expressly prohibited 
from producing environmental-economic accounts, 
the IEESA accounts are not likely to be resumed. 
If the BEA was not able to receive political backing 
to undertake environmental-economic accounting 
under President Bill Clinton, Vice-President Al Gore, 
and two democratically-controlled chambers of Congress, 
it is unlikely the BEA will attempt environmental-economic accounts again 
without a specific mandate from the administration. 

It is a rare nation that will voluntarily consume less today 
to save for future generations. 
However, a nation that has a balance sheet as well as 
an income statement can observe when it is consuming 
the future generations' productive capital 
to produce the current generation's income. 
This is information that business leaders and policy-makers sould want to have. 
This book highlights the need for these data and 
provides a framework for using them in a way relevant for policy-makers.

In the ancient fable, 
six blind men observe six different parts 
of an elephant and draw different conclusions 
about what is in front of them. 
Similarly, scientists and economists observe 
different parts of the economy and draw strikingly different conclusions 
about the future. 
We contend that both the scientists and the economists 
need to take off their blinders and 
appreciate that capital in all forms\footnote{Natural, 
	manufactured, human, social, and financial.} 
is necessary to generate the services an economy requires. 
These two perspectives must be brought together 
to understand the potential futures we are facing. 
This book is our attempt to do so.

\bibliographystyle{unsrt}
\bibliography{../../Metabolic}


% Always give a unique label
% and use \ref{<label>} for cross-references
% and \cite{<label>} for bibliographic references
% use \sectionmark{}
% to alter or adjust the section heading in the running head
%% Instead of simply listing headings of different levels we recommend to let every heading be followed by at least a short passage of text. Furtheron please use the \LaTeX\ automatism for all your cross-references and citations.

%% Please note that the first line of text that follows a heading is not indented, whereas the first lines of all sequent paragraphs are.

%% Use the standard \verb|equation| environment to typeset your equations, e.g.
%
%% \begin{equation}
%% a \times b = c\;,
%% \end{equation}
%
%% however, for multiline equations we recommend to use the \verb|eqnarray|
%% environment\footnote{In physics texts please activate the class option \texttt{vecphys} to depict your vectors in \textbf{\itshape boldface-italic} type - as is customary for a wide range of physical jects.}.
%% \begin{eqnarray}
%% a \times b = c \nonumber\\
%% \vec{a} \cdot \vec{b}=\vec{c}
%% \label{eq:01}
%% \end{eqnarray}

%% \section{section Heading}
%% \label{sec:2}
%% Instead of simply listing headings of different levels we recommend to let every heading be followed by at least a short passage of text. Furtheron please use the \LaTeX\ automatism for all your cross-references\index{cross-references} and citations\index{citations} as has already been described in Sect.~\ref{sec:2}.

%% \begin{quotation}
%% Please do not use quotation marks when quoting texts! Simply use the \verb|quotation| environment -- it will automatically render Springer's preferred layout.
%% \end{quotation}


%% \section{section Heading}
%% Instead of simply listing headings of different levels we recommend to let every heading be followed by at least a short passage of text. Furtheron please use the \LaTeX\ automatism for all your cross-references and citations as has already been described in Sect.~\ref{sec:2}, see also Fig.~\ref{fig:1}\footnote{If you copy text passages, figures, or tables from other works, you must obtain \textit{permission} from the copyright holder (usually the original publisher). Please enclose the signed permission with the manucript. The sources\index{permission to print} must be acknowledged either in the captions, as footnotes or in a separate section of the book.}

%% Please note that the first line of text that follows a heading is not indented, whereas the first lines of all sequent paragraphs are.

% For figures use
%
%% \begin{figure}[b]
%% \sidecaption
% Use the relevant command for your figure-insertion program
% to insert the figure file.
% For example, with the option graphics use
%% \includegraphics[scale=.65]{figure}
%
% If not, use
%\picplace{5cm}{2cm} % Give the correct figure height and width in cm
%
%% \caption{If the width of the figure is less than 7.8 cm use the \texttt{sidecapion} command to flush the caption on the left side of the page. If the figure is positioned at the top of the page, align the sidecaption with the top of the figure -- to achieve this you simply need to use the optional argument \texttt{[t]} with the \texttt{sidecaption} command}
%% \label{fig:1}       % Give a unique label
%% \end{figure}


%% \paragraph{Paragraph Heading} %
%% Instead of simply listing headings of different levels we recommend to let every heading be followed by at least a short passage of text. Furtheron please use the \LaTeX\ automatism for all your cross-references and citations as has already been described in Sect.~\ref{sec:2}.

%% Please note that the first line of text that follows a heading is not indented, whereas the first lines of all sequent paragraphs are.

%% For typesetting numbered lists we recommend to use the \verb|enumerate| environment -- it will automatically render Springer's preferred layout.

%% \begin{enumerate}
%% \item{Livelihood and survival mobility are oftentimes coutcomes of uneven socioeconomic development.}
%% \begin{enumerate}
%% \item{Livelihood and survival mobility are oftentimes coutcomes of uneven socioeconomic development.}
%% \item{Livelihood and survival mobility are oftentimes coutcomes of uneven socioeconomic development.}
%% \end{enumerate}
%% \item{Livelihood and survival mobility are oftentimes coutcomes of uneven socioeconomic development.}
%% \end{enumerate}


%% \paragraph{paragraph Heading} In order to avoid simply listing headings of different levels we recommend to let every heading be followed by at least a short passage of text. Use the \LaTeX\ automatism for all your cross-references and citations as has already been described in Sect.~\ref{sec:2}, see also Fig.~\ref{fig:2}.

%% Please note that the first line of text that follows a heading is not indented, whereas the first lines of all sequent paragraphs are.

%% For unnumbered list we recommend to use the \verb|itemize| environment -- it will automatically render Springer's preferred layout.

%% \begin{itemize}
%% \item{Livelihood and survival mobility are oftentimes coutcomes of uneven socioeconomic development, cf. Table~\ref{tab:1}.}
%% \begin{itemize}
%% \item{Livelihood and survival mobility are oftentimes coutcomes of uneven socioeconomic development.}
%% \item{Livelihood and survival mobility are oftentimes coutcomes of uneven socioeconomic development.}
%% \end{itemize}
%% \item{Livelihood and survival mobility are oftentimes coutcomes of uneven socioeconomic development.}
%% \end{itemize}

%% \begin{figure}[t]
%% \sidecaption[t]
% Use the relevant command for your figure-insertion program
% to insert the figure file.
% For example, with the option graphics use
%% \includegraphics[scale=.65]{figure}
%
% If not, use
%\picplace{5cm}{2cm} % Give the correct figure height and width in cm
%
%% \caption{Please write your figure caption here}
%% \label{fig:2}       % Give a unique label
%% \end{figure}

%% \runinhead{Run-in Heading Boldface Version} Use the \LaTeX\ automatism for all your cross-references and citations as has already been described in Sect.~\ref{sec:2}.

%% \runinhead{Run-in Heading Italic Version} Use the \LaTeX\ automatism for all your cross-refer\-ences and citations as has already been described in Sect.~\ref{sec:2}\index{paragraph}.
% Use the \index{} command to code your index words
%
% For tables use
%
%% \begin{table}
%% \caption{Please write your table caption here}
%% \label{tab:1}       % Give a unique label
%
% For LaTeX tables use
%
%% \begin{tabular}{p{2cm}p{2.4cm}p{2cm}p{4.9cm}}
%% \hline\noalign{\smallskip}
%% Classes & class & Length & Action Mechanism  \\
%% \noalign{\smallskip}\svhline\noalign{\smallskip}
%% Translation & mRNA$^a$  & 22 (19--25) & Translation repression, mRNA cleavage\\
%% Translation & mRNA cleavage & 21 & mRNA cleavage\\
%% Translation & mRNA  & 21--22 & mRNA cleavage\\
%%Translation & mRNA  & 24--26 & Histone and DNA Modification\\
%%\noalign{\smallskip}\hline\noalign{\smallskip}
%%\end{tabular}
%%$^a$ Table foot note (with superscript)
%%\end{table}
%
%% \section{Section Heading}
%%\label{sec:3}
% Always give a unique label
% and use \ref{<label>} for cross-references
% and \cite{<label>} for bibliographic references
% use \sectionmark{}
% to alter or adjust the section heading in the running head
%% Instead of simply listing headings of different levels we recommend to let every heading be followed by at least a short passage of text. Furtheron please use the \LaTeX\ automatism for all your cross-references and citations as has already been described in Sect.~\ref{sec:2}.

%% Please note that the first line of text that follows a heading is not indented, whereas the first lines of all sequent paragraphs are.

%%If you want to list definitions or the like we recommend to use the Springer-enhanced \verb|description| environment -- it will automatically render Springer's preferred layout.

%%\begin{description}[Type 1]
%%\item[Type 1]{That addresses central themes pertainng to migration, health, and disease. In Sect.~\ref{sec:1}, Wilson discusses the role of human migration in infectious disease distributions and patterns.}
%%\item[Type 2]{That addresses central themes pertainng to migration, health, and disease. In Sect.~\ref{sec:2}, Wilson discusses the role of human migration in infectious disease distributions and patterns.}
%%\end{description}

%%\section{section Heading} %
%% In order to avoid simply listing headings of different levels we recommend to let every heading be followed by at least a short passage of text. Use the \LaTeX\ automatism for all your cross-references and citations citations as has already been described in Sect.~\ref{sec:2}.

%% Please note that the first line of text that follows a heading is not indented, whereas the first lines of all sequent paragraphs are.

%% \begin{svgraybox}
%% If you want to emphasize complete paragraphs of texts we recommend to use the newly defined Springer class option \verb|graybox| and the newly defined environment \verb|svgraybox|. This will produce a 15 percent screened box 'behind' your text.

%% If you want to emphasize complete paragraphs of texts we recommend to use the newly defined Springer class option and environment \verb|svgraybox|. This will produce a 15 percent screened box 'behind' your text.
%% \end{svgraybox}


%% \section{section Heading}
%%Instead of simply listing headings of different levels we recommend to let every heading be followed by at least a short passage of text. Furtheron please use the \LaTeX\ automatism for all your cross-references and citations as has already been described in Sect.~\ref{sec:2}.

%% Please note that the first line of text that follows a heading is not indented, whereas the first lines of all sequent paragraphs are.

%% \begin{theorem}
%% Theorem text goes here.
%% \end{theorem}
%
% or
%
%% \begin{definition}
%% Definition text goes here.
%% \end{definition}

%% \begin{proof}
%\smartqed
%% Proof text goes here.
%% \qed
%% \end{proof}

%%\paragraph{Paragraph Heading} %
%% Instead of simply listing headings of different levels we recommend to let every heading be followed by at least a short passage of text. Furtheron please use the \LaTeX\ automatism for all your cross-references and citations as has already been described in Sect.~\ref{sec:2}.

%% Note that the first line of text that follows a heading is not indented, whereas the first lines of all subsequent paragraphs are.
%
% For built-in environments use
%
%%\begin{theorem}
%%Theorem text goes here.
%%\end{theorem}
%
%%\begin{definition}
%%Definition text goes here.
%%\end{definition}
%
%%\begin{proof}
%%\smartqed
%% Proof text goes here.
%%\qed
%%\end{proof}
%
%% \begin{acknowledgement}
%% If you want to include acknowledgments of assistance and the like at the end of an individual chapter please use the \verb|acknowledgement| environment -- it will automatically render Springer's preferred layout.
%% \end{acknowledgement}
%
%% \section*{Appendix}
%% \addcontentsline{toc}{section}{Appendix}
%
%% When placed at the end of a chapter or contribution (as opposed to at the end of the book), the numbering of tables, figures, and equations in the appendix section continues on from that in the main text. Hence please \textit{do not} use the \verb|appendix| command when writing an appendix at the end of your chapter or contribution. If there is only one the appendix is designated ``Appendix'', or ``Appendix 1'', or ``Appendix 2'', etc. if there is more than one.

%% \begin{equation}
%% a \times b = c
%% \end{equation}
% Problems or Exercises should be sorted chapterwise
%% \section*{Problems}
%% \addcontentsline{toc}{section}{Problems}
%
% Use the following environment.
% Don't forget to label each problem;
% the label is needed for the solutions' environment
%% \begin{prob}
%% \label{prob1}
%% A given problem or Excercise is described here. The
%% problem is described here. The problem is described here.
%% \end{prob}

%% \begin{prob}
%% \label{prob2}
%% \textbf{Problem Heading}\\
%% (a) The first part of the problem is described here.\\
%% (b) The second part of the problem is described here.
%% \end{prob}


