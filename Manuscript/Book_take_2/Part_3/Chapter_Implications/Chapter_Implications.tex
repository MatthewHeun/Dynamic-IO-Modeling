%!TEX root = ../../Heun_Dale_Haney_A_dynamic_approach_to_input_output_modeling.tex
%%%%%%%%%%%%%%%%%%%%% chapter.tex %%%%%%%%%%%%%%%%%%%%%%%%%%%%%%%%%
%
% sample chapter
%
% Use this file as a template for your own input.
%
%%%%%%%%%%%%%%%%%%%%%%%% Springer-Verlag %%%%%%%%%%%%%%%%%%%%%%%%%%
%\motto{Use the template \emph{chapter.tex} to style the various elements of your chapter content.}

%%%%%%%%%%%%%%%%%%%%%%%%%%%%%%%%%%
%%%%%%%%%% Implications %%%%%%%%%%
%%%%%%%%%%%%%%%%%%%%%%%%%%%%%%%%%%
\chapter{Implications}
% Always give a unique label
\label{chap:implications}
% use \chaptermark{} to alter or adjust the chapter heading in the running head
\chaptermark{Implications}
%%%%%%%%%%%%%%%%%%%%%%%%%%%%%%%%%%
%%%%%%%%%%%%%%%%%%%%%%%%%%%%%%%%%%
%%%%%%%%%%%%%%%%%%%%%%%%%%%%%%%%%%

\abstract*{[NEED TO ADD ABSTRACT HERE]}

%% \abstract{Each chapter should be preceded by an abstract (10--15 lines long) that summarizes the content. The abstract will appear \textit{online} at \url{www.SpringerLink.com} and be available with unrestricted access. This allows unregistered users to read the abstract as a teaser for the complete chapter. As a general rule the abstracts will not appear in the printed version of your book unless it is the style of your particular book or that of the series to which your book belongs.\newline\indent
%% Please use the 'starred' version of the new Springer \texttt{abstract} command for typesetting the text of the online abstracts (cf. source file of this chapter template \texttt{abstract}) and include them with the source files of your manuscript. Use the plain \texttt{abstract} command if the abstract is also to appear in the printed version of the book.}

%% Use the template \emph{chapter.tex} together with the Springer document class SVMono (monograph-type books) or SVMult (edited books) to style the various elements of your chapter content in the Springer layout.


Several implications can be drawn from the detailed development 
of the I-O method in Chapters~\ref{chap:materials}--\ref{chap:intensity}
that includes energy input from society to the economy,
embodied energy accumulation, and depreciation.


%%%%%%%%%% Implications for the I-O method %%%%%%%%%%
\section{Implications for the I-O method}
\label{sec:Implications_for_IO}
%%%%%%%%%%

The first set of implications are for the I-O method itself;
specifically, for the process of estimating
the energy intensity of economic output ($\bm{\varepsilon}$).


%%%%%%%%%% I-O implications: estimating epsilon %%%%%%%%%%
\subsection{Estimating $\bm{\varepsilon}$}
\label{sec:estimating_epsilon-implications_chapter}
%%%%%%%%%%

Extension of the Leontief\index{Leontief} 
Input-Output method\index{input-output method}
for energy analysis has allowed energy analysts to estimate 
the energy intensity\index{energy intensity}
of economic products ($\bm{\varepsilon}$). 
As discussed in Section~\ref{sec:Value_Methodology},
we do not take this important result as a license
to declare an intrinsic ``energy theory of value.''
\index{theory of value!energy}
Rather, we belive that energy intensity is an 
important and useful metric that can assess 
the energy performance of economies,
even within the prevailing subjective theory of value
\index{theory of value!subjective}
that underlies modern economics. 
Thus, it is important to consider the assumptions behind
the literature's presentation of the I-O method 
for estimating the energy intensity economic output.

In this manuscript, Equation~\ref{eq:epsilon_leontief_depreciation_simplification} 
provides a means of estimating the energy intensity of economic sectors:

\begin{equation}
	\bm{\varepsilon} 
	= {(\vec{I} - \vec{A}^{\mathrm{T}})}^{-1}\hat{\vec{X}}^{-1}
		\left[\vec{E}_{0} 
				+ \vec{T}_{1} 
				- \left. \frac{\mathrm{d}\vec{B}}{\mathrm{d}t} \right|_{\mathrm{other}}
		\right]. \tag{\ref{eq:epsilon_leontief_depreciation_simplification}}
\end{equation}

\noindent{}The I-O literature~\cite{Bullard1975,Casler1984}, 
on the other hand, 
writes Equation~\ref{eq:epsilon_leontief_depreciation_simplification} 
as\footnote{For a discussion of other, more-subtle, differences
between the energy intensity equations in the literature
and this manuscript, see Appendix~\ref{chap:Casler}.}

\begin{equation} \label{eq:epsilon_leontief_with_A_literature}
	\bm{\varepsilon} 
	= {(\vec{I} - \vec{A}^{\mathrm{T}})}^{-1}
	\hat{\vec{X}}^{-1}
	\vec{E}_{0}.
\end{equation}

The differences between Equations~\ref{eq:epsilon_leontief_depreciation_simplification}
and~\ref{eq:epsilon_leontief_with_A_literature} are obvious. 
The literature neglects
energy input from society ($\vec{T}_{1}$)
and accumulation of embodied energy in the economy,
exclusive of physical depreciation 
$\left( \left. \frac{\mathrm{d}\vec{B}}{\mathrm{d}t} \right|_{\mathrm{other}} \right)$,
when estimating the energy intensity ($\bm{\varepsilon}$) 
of economic sectors.\footnote{To be precise, 
the literature effectively assumes
$
	\vec{T}_{1}
	- \left. \frac{\mathrm{d}\vec{B}}{\mathrm{d}t} \right|_{\mathrm{other}}
	= \vec{0}
$}
In other words, energy analysts using the input-output method
have, to date, and perhaps unwittingly, assumed 
a developed-world ($\vec{T}_{1} \ll \vec{E}_{0}$), 
steady state economy\index{steady-state economy}
$\left( \left. \frac{\mathrm{d}\mathrm{\vec{B}}}{\mathrm{d}t} \right|_{\mathrm{other}} 
= \vec{0} \right)$.

The following subsections discuss the assumptions made by the I-O literature.


%+++++++++ Energy input from society ++++++++++
\subsubsection{Negligible energy input from society $\left( \vec{T}_{1} = 0 \right)$}
%+++++++++

Energy input from society to the economy ($\vec{T}_{1}$)
is ``muscle work'' supplied by working humans 
and draught animals.\cite{Ayres:2003ec,Ayres:2010ug,Warr:2012cg} 

**** Mik asks: One question I have is:
Does this value include all of the upstream energy needed to support labor, 
or simply the upstream energy required to produce the food that fuels the labor?
We should spell out exactly what is included via this framework. 
This is important because, in NEA, labor is often used 
as a HUGE lever to produce whatever result is desired.
****

For developed economies, muscle work is a small fraction
of the energy input from fossil fuels ($\vec{E}_{0}$),
so neglecting $\vec{T}_{1}$ causes negligible error when
estimating energy intensity ($\bm{\varepsilon}$) by
Equation~\ref{eq:epsilon_leontief_with_A_literature}.
However, for some agrarian\index{economy!agrarian} 
and developing economies\index{economy!developing}, 
where $\vec{T}_{1}$ and $\vec{E}_{0}$ 
could be on the same order of magnitude,
neglecting $\vec{T}_{1}$ could cause errors
in estimates of $\bm{\varepsilon}$.
To the extent that $\vec{T}_{1}$ 
is significant relative to $\vec{E}_{0}$,
Equation~\ref{eq:epsilon_leontief_with_A_literature}
will underpredict the energy intensity of the economy.

Accurate estimation of the energy intensity of economic output ($\bm{\varepsilon}$)
requires independent knowledge of the rate at which society supplies
energy to the economy ($\vec{T}_{1}$). 
Ayres and Warr have estimated human and animal muscle work
input to the economy for a few developed countries.\cite{Ayres:2010ug}
We recommend that more of this work be done 
in the future for many more countries.


%+++++++++ Accumulation of embodied energy ++++++++++
\subsubsection{Negligible accumulation of embodied energy
$\left( \left. \frac{\mathrm{d}\vec{B}}{\mathrm{d}t} \right|_{\mathrm{other}} = 0 \right)$}
%+++++++++

As discussed in detail below (Section~\ref{sec:implications_for_development}),
the accumulation of embodied energy 
$\left( \frac{\mathrm{d}\vec{B}}{\mathrm{d}t} \right)$ 
in society and the economy
can be considered a marker of economic ``development.''

**** 
Mik says: Daly talks about this in terms 
of Odum's B/P vs P/B ratios for growing vs.\ steady systems.
****

Equation~\ref{eq:epsilon_leontief_depreciation_simplification} 
shows that embodied energy accumulation within economic sectors 
exclusive of depreciation
$\left( \left. \frac{\mathrm{d}\vec{B}}{\mathrm{d}t} \right|_{\mathrm{other}} \right)$ 
decreases the energy intensity of products ($\bm{\varepsilon}$),

**** Mik says: I think we should limit discussion of ``energy intensity'' 
to activities, so here maybe something like,
``decreases the energy intensity of manufacturing goods and services\ldots''
****

**** Matt responds: I understand the sentiment, 
but we'd be departing from the literature, I think. 
The literature definitely discusses the energy intensity
of the \emph{products} of a sector. 
For example, Bullard (1978), p. 268 says:
``The data and methodologies described in this report 
permit calculation of five types of energy `embodied' in a particular goods or service.''
Again, on p. 272, ``We now have most of the data necessary 
to calculate the energy intensity of cars using process analysis.''
****

because incoming energy is embodied within capital stock rather than products.
Comparison with Equation~\ref{eq:epsilon_leontief_with_A_literature}
shows that the method in the literature will overestimate the energy intensity
of economic products ($\bm{\varepsilon}$) when 
$\left. \frac{\mathrm{d}\vec{B}}{\mathrm{d}t} \right|_{\mathrm{other}} > 0$.

Rapidly growing economies (as measured by GDP), such as China and India today,
are expected to have rather large positive values of 
$\left. \frac{\mathrm{d}\vec{B}}{\mathrm{d}t} \right|_{\mathrm{other}}$,
while slowly-growing, industrialized economies, 
such as the United States and the United Kingdom,
are expected to have rather smaller values of 
$\left. \frac{\mathrm{d}\vec{B}}{\mathrm{d}t} \right|_{\mathrm{other}}$.
Applying Equation~\ref{eq:epsilon_leontief_with_A_literature} 
(i.e., assuming a steady-state economy\index{economy!steady-state})
will tend to overestimate the energy intensity 
of economic products for rapidly growing economies such as China and India.

Accurate estimation of the energy intensity of economic output ($\bm{\varepsilon}$)
requires independent knowledge of the rate 
at which embodied energy accumulates in the economy 
$\left( \left. \frac{\mathrm{d}\vec{B}}{\mathrm{d}t} \right|_{\mathrm{other}} \right)$. 
To our knowledge, there are no examples of estimating the accumulation 
of embodied energy in the economy. 
Further work focused on estimating the relative magnitudes of 
$\left( \left. \frac{\mathrm{d}\vec{B}}{\mathrm{d}t} \right|_{\mathrm{other}} \right)$
and $\vec{T}_{1}$ will benefit energy analysts who utilize the I-O method.

A second set of implications for the I-O method 
comes from one of the ways that 
the energy intensity vector ($\bm{\varepsilon}$)
is often used: to estimate energy demand from the biosphere ($\vec{E}_{0}$).


%%%%%%%%%% I-O implications: estimating E %%%%%%%%%%
\subsection{Estimating $\vec{E}_{0}$}
%%%%%%%%%%

The assumptions of Equation~\ref{eq:epsilon_leontief_with_A_literature}
may cause another challenge for energy analysts. 
As discussed in Sections~\ref{sec:estimating_epsilon-intensity_chapter} 
and~\ref{sec:estimating_epsilon-implications_chapter} above, 
the I-O method can be used to estimate energy intensities 
for each sector of the economy ($\bm{\varepsilon}$). 
With $\bm{\varepsilon}$ values in hand,
and assuming that $\bm{\varepsilon}$ is constant with respect to time,
energy analysts can estimate changes in energy demand 
from the biosphere ($\vec{E}_{0}$) 
as the output of economic sectors ($\hat{\vec{X}}$) 
increases or decreases by solving 
Equation~\ref{eq:epsilon_leontief_with_A_literature} 
for $\vec{E}_{0}$:

\begin{equation} \label{eq:Leontief_lit_solved_for_E}
	\vec{E}_{0} 
	= \hat{\vec{X}}(\vec{I} - \vec{A}^{\mathrm{T}})\bm{\varepsilon}.
\end{equation}

When Equation~\ref{eq:epsilon_leontief_with_A_literature}
is modified to account for accumulation of embodied energy 
in the economy 
$\left. \frac{\mathrm{d}\vec{B}}{\mathrm{d}t} \right|_{\mathrm{other}}$
and energy supplied by society to the economy ($\vec{T}_{1}$),
we see that the energy demands ($\vec{E}$) must be calculated differently. 
Solving Equation~\ref{eq:epsilon_leontief_depreciation_simplification} 
for $\vec{E}_{0}$ gives 

\begin{equation} \label{eq:Leontief_solved_for_E_with_embodied_depreciation}
	\vec{E}_{0} 
	= \hat{\vec{X}}
		(\vec{I} - \vec{A}^{\mathrm{T}})
		\bm{\varepsilon} 
	+ \left. \frac{\mathrm{d}\vec{B}}{\mathrm{d}t} \right|_{\mathrm{other}}
	- \vec{T}_{1}.
\end{equation}

\noindent{}Comparing Equations~\ref{eq:Leontief_lit_solved_for_E} 
and~\ref{eq:Leontief_solved_for_E_with_embodied_depreciation}, 
shows that to the extent that embodied energy accumulation,
exclusive of physical depreciation,
$\left. \frac{\mathrm{d}\vec{B}}{\mathrm{d}t} \right|_{\mathrm{other}}$
is non-zero, estimates of energy demand ($\vec{E}_{0}$) using 
Equation~\ref{eq:Leontief_lit_solved_for_E} are too low. 
And, if energy input from society to the economy ($\vec{T}_{1}$) is significant,
estimates of energy demand ($\vec{E}_{0}$) using 
Equation~\ref{eq:Leontief_lit_solved_for_E} are too high. 

At this time, the relative magnitudes of $\vec{E}_{0}$,
$\left. \frac{\mathrm{d}\vec{B}}{\mathrm{d}t} \right|_{\mathrm{other}}$,
and $\vec{T}_{1}$ are unknown. 
Further work to clarify these magnitudes will be beneficial
for energy analysts who employ the I-O method.


******** Matt stopped here. *********


%%%%%%%%%% Implications %%%%%%%%%%
\section{Implications for economic ``development''}
\label{sec:implications_for_development}
%%%%%%%%%%

[IT WOULD BE GOOD TO HAVE A COMPARISON BETWEEN $\frac{\mathrm{d}\vec{B}}{\mathrm{d}t}$ 
AND STANDARD METRIC OF DEVELOPMENT, 
I.E. GDP WHICH I GUESS WOULD BE SOMETHING LIKE $\sum_{i}\dot{X}_{i}$. 
WE CAN CERTAINLY ENVISION SITUATIONS WHERE $\sum_{i}\dot{X}_{i}$ IS INCREASING AND]

One consequence of economic ``progress'' or ``development'' is that embodied energy accumulates 
in economic sectors and society. 
In fact, accumulation of embodied energy in economic sectors and society 
could be considered a \emph{proxy} for development. 
This proxy for development is overly materialistic, one-dimensional, and reductionist, 
but alternatives such as GDP can be similarly criticized. 
In fact, GDP could continue to increase whilst accumulation 
of embodied energy or value actually decreased.

Figure~\ref{fig:total_energy_flows_2S} shows that energy extraction 
from the Earth is what ultimately drives development 
as measured by the accumulation of embodied energy in the economy and society. 
Development occurs over time. 
If embodied energy is the measure, 
development can be expressed as 
the integral of $\frac{\mathrm{d}\vec{B}}{\mathrm{d}t}$ for economic sectors

***************** Need to edit the following equations for index changes. ************

\begin{equation} \label{eq:Dev_Integral_Economy}
	\vec{B}(t) 
	= \vec{B}(0) 
	+ \int_{t=0}^{t=t} \frac{\mathrm{d}\vec{B}}{\mathrm{d}t}\mathrm{d}t,
\end{equation}

\noindent{}or, using Equation~\ref{eq:compare_demand_and_accumulation}, 
as the integral of $\frac{\mathrm{d}B_{2}}{\mathrm{d}t}$ for society,

\begin{equation} \label{eq:Dev_Integral_Society}
	B_{2}(t) 
	= B_{2}(0) 
	+ \int_{t=0}^{t=t} \frac{\mathrm{d}B_{2}}{\mathrm{d}t}\mathrm{d}t 
	= B_{2}(0) 
	+ \int_{t=0}^{t=t} (Y_{\dot{T}} 
	- \gamma_{2}B_{2} 
	- \dot{Q}_{21})\mathrm{d}t.
\end{equation}

%[I ADDED IN A $\dot{Q}_{21}$ TERM HERE - MD]

Using embodied energy is obviously an incomplete measure of development. 
We might also use $X(t) = X(0) + \int\frac{dX}{dt}dt$. 
In fact, B and X are two complimentary factors to the economic process. 
For capital, B, to be useful, we need direct energy, 
E (to run the capital) and economic value, X (i.e.\ currency). 
Therefore each of these factors are necessary, but insufficient.

Table~\ref{table:embodied_energy_accumulation_factors} 
describes some of the dynamics that can be observed from 
Equation~\ref{eq:dB_dt_leontief_with_A}. 
It is quite possible that, especially for regions like the U.S. and Western Europe, 
the rate of embodied energy accumulation in the economy 
$\left(\frac{\mathrm{d}\vec{B}}{\mathrm{d}t}\right)$ 
will be small relative to the rate of energy extraction 
from the Earth ($\vec{E}$). 
On the other hand, in rapidly developing countries, 
like China or India, the rate of embodied energy accumulation 
in the economy may be significantly higher than in a developed economy.

\begin{table}
\caption[Factors affecting the accumulation rate of embodied energy in the economy.]{Factors from Equation~\ref{eq:dB_dt_leontief_with_A} 
affecting the rate of embodied energy accumulation in the economy.}
\begin{center}
  \begin{tabular}{c @{\hspace{2em}} l}
    \toprule
    Term & Implication \\ 
	\midrule
    $\hat{\vec{X}}$ & As economic output increases, $\frac{\mathrm{d}\vec{B}}{\mathrm{d}t}$ goes up (as will $\vec{E})$  \\
    $\vec{A}$ & As input-ouput ratios increase, $\frac{\mathrm{d}\vec{B}}{\mathrm{d}t}$ goes up  \\
    $\varepsilon$ & As the energy intensity of the economy increases, $\frac{\mathrm{d}\vec{B}}{\mathrm{d}t}$ goes up  \\ 
   $ \vec{E}$ & As the rate of energy flow from the Earth increases, $\frac{\mathrm{d}\vec{B}}{\mathrm{d}t}$ goes up  \\ 
    $\hat{\vec{\gamma}}$ & As the depreciation rate increases, $\frac{\mathrm{d}\vec{B}}{\mathrm{d}t}$ goes down  \\ 
    $\vec{B}$ & As the embodied energy in the economy increases, $\frac{\mathrm{d}\vec{B}}{\mathrm{d}t}$ goes down  \\ \bottomrule
  \end{tabular}
\end{center}
\label{table:embodied_energy_accumulation_factors}
\end{table}

The behavior of $\vec{B}$ with $\frac{\mathrm{d}\vec{B}}{\mathrm{d}t}$ 
is vitally important. 
A developed economy has significantly higher embodied energy ($\vec{B}$) 
than a developing economy, and, thus, 
the outflow rate of embodied energy 
due to depreciation ($\hat{\vec{\gamma}}\vec{B}$) will be higher. 
As increasingly large amounts of energy are embodied in the economy, 
increasingly large energy extraction rates ($\vec{E}$) 
are required to offset depreciation ($\hat{\vec{\gamma}}\vec{B}$) 
and maintain positive growth $\left(\frac{\mathrm{d}\vec{B}}{\mathrm{d}t} > 0\right)$ 
in the sectors of the economy. 
Depreciation may also be, temporarily, 
offset by increasing energy efficiency, i.e.\ by decreasing energy intensity, $\varepsilon$.

In a similar manner, 
Equation~\ref{eq:compare_demand_and_accumulation} 
indicates that maintaining a positive rate of societal development 
$\left(\frac{\mathrm{d}B_{2}}{\mathrm{d}t} > 0\right)$ 
requires ever increasing embodied energy input rates 
to society ($Y_{\dot{T}}$) as the society ``develops.'' 
This mechanism provides a natural restraint
to the continued growth of physical economies.


%%%%%%%%%% Implications %%%%%%%%%%
\section{Implications for recycling, reuse, and dematerialization}
%%%%%%%%%%

Dematerialization is the idea that economic activity can be unlinked 
from material or energy demands (UNEP, 2011) 
******** Get real reference ********. 
One of the primary methods for dematerializing an economy 
is reuse and recycling of materials. 
The impact of recycling can be seen in the I-O formulation 
only when depreciation and accumulation are included. 

One effect of recycling is to reduce the magnitude 
of the disposal rate 
($\hat{\vec{\gamma}}$). 
Equation~\ref{eq:dB_dt_leontief_with_A} indicates that 
recycling of material in an economy, 
thereby reducing $\hat{\vec{\gamma}}$, 
will slow the effect of depreciation 
$\left(\hat{\vec{\gamma}}\vec{B}\right)$ 
and put upward pressure on growth 
$\left(\frac{\mathrm{d}\vec{B}}{\mathrm{d}t}\right)$. 

Recycling has a mixed effect on energy demand ($\vec{E}$). 
Because recycled material displaces newly-produced material 
in the economy and society, 
recycling will tend to reduce energy demand ($\vec{E}$). 
Equation~\ref{eq:dB_dt_leontief_with_A} indicates that 
this displacement effect will put downward pressure on growth 
$\left(\frac{\mathrm{d}B}{\mathrm{d}t}\right)$. 
However, recycling processes require energy to operate, 
thereby increasing energy demand ($\vec{E}$). 
Equation~\ref{eq:dB_dt_leontief_with_A} indicates that 
additional energy demand will put upward pressure on growth 
$\left(\frac{\mathrm{d}B}{\mathrm{d}t}\right)$. 

If recycling produces a net reduction in energy demand ($\vec{E}$), 
that is if the effect of displaced production dominates over the effect 
of energy consumed in recycling processes, 
the upward pressure on growth $\left(\frac{\mathrm{d}B}{\mathrm{d}t}\right)$ 
from decrease in $\hat{\vec{\gamma}}$ and 
the downward pressure on growth from net reduction of $\vec{E}$ 
offset each other, 
the growth rate $\left(\frac{\mathrm{d}B}{\mathrm{d}t}\right)$ 
will remain near zero, 
and total embodied energy $(\vec{B})$ will remain constant. 
In that scenario, dematerialization can develop: 
reduced material and energy input ($\vec{E}$) can be accompanied by 
no change in growth $\left(\frac{\mathrm{d}\vec{B}}{\mathrm{d}t}\right)$.


%%%%%%%%%% Implications %%%%%%%%%%
\section{Comparison to a Steady-state Economy}
%%%%%%%%%%

****** Finish this section. 
In terms of what a SSE would look like in the I-O framework, 
at first blush, I would think that dB/dt = 0 is one aspect.  
Also, with no growth, inflow rates = depreciation rates.  
The larger that B is for any society, the larger E must be (to overcome depreciation).  
To minimize E, hyper-recycling is probably useful.  
Those are at least a place to start. ******

****** In our discussion, 
we also addressed the attempts at SSE from point of view of society. 
In order to achieve this goal \emph{without} recycling, 
the goods and services sector should have to increase extraction to offset decreasing ore grade, 
the energy sector should have to increase extraction of energy 
to allow increasing extraction (unless efficiency could make up the gap: unlikely) 
in which case the SSE would be violated from these two and from the POV of the earth.
******


\bibliographystyle{unsrt}
\bibliography{../../EROI_review_v2}


% Always give a unique label
% and use \ref{<label>} for cross-references
% and \cite{<label>} for bibliographic references
% use \sectionmark{}
% to alter or adjust the section heading in the running head
%% Instead of simply listing headings of different levels we recommend to let every heading be followed by at least a short passage of text. Furtheron please use the \LaTeX\ automatism for all your cross-references and citations.

%% Please note that the first line of text that follows a heading is not indented, whereas the first lines of all subsequent paragraphs are.

%% Use the standard \verb|equation| environment to typeset your equations, e.g.
%
%% \begin{equation}
%% a \times b = c\;,
%% \end{equation}
%
%% however, for multiline equations we recommend to use the \verb|eqnarray|
%% environment\footnote{In physics texts please activate the class option \texttt{vecphys} to depict your vectors in \textbf{\itshape boldface-italic} type - as is customary for a wide range of physical subjects.}.
%% \begin{eqnarray}
%% a \times b = c \nonumber\\
%% \vec{a} \cdot \vec{b}=\vec{c}
%% \label{eq:01}
%% \end{eqnarray}

%% \subsection{Subsection Heading}
%% \label{subsec:2}
%% Instead of simply listing headings of different levels we recommend to let every heading be followed by at least a short passage of text. Furtheron please use the \LaTeX\ automatism for all your cross-references\index{cross-references} and citations\index{citations} as has already been described in Sect.~\ref{sec:2}.

%% \begin{quotation}
%% Please do not use quotation marks when quoting texts! Simply use the \verb|quotation| environment -- it will automatically render Springer's preferred layout.
%% \end{quotation}


%% \subsubsection{Subsubsection Heading}
%% Instead of simply listing headings of different levels we recommend to let every heading be followed by at least a short passage of text. Furtheron please use the \LaTeX\ automatism for all your cross-references and citations as has already been described in Sect.~\ref{subsec:2}, see also Fig.~\ref{fig:1}\footnote{If you copy text passages, figures, or tables from other works, you must obtain \textit{permission} from the copyright holder (usually the original publisher). Please enclose the signed permission with the manucript. The sources\index{permission to print} must be acknowledged either in the captions, as footnotes or in a separate section of the book.}

%% Please note that the first line of text that follows a heading is not indented, whereas the first lines of all subsequent paragraphs are.

% For figures use
%
%% \begin{figure}[b]
%% \sidecaption
% Use the relevant command for your figure-insertion program
% to insert the figure file.
% For example, with the option graphics use
%% \includegraphics[scale=.65]{figure}
%
% If not, use
%\picplace{5cm}{2cm} % Give the correct figure height and width in cm
%
%% \caption{If the width of the figure is less than 7.8 cm use the \texttt{sidecapion} command to flush the caption on the left side of the page. If the figure is positioned at the top of the page, align the sidecaption with the top of the figure -- to achieve this you simply need to use the optional argument \texttt{[t]} with the \texttt{sidecaption} command}
%% \label{fig:1}       % Give a unique label
%% \end{figure}


%% \paragraph{Paragraph Heading} %
%% Instead of simply listing headings of different levels we recommend to let every heading be followed by at least a short passage of text. Furtheron please use the \LaTeX\ automatism for all your cross-references and citations as has already been described in Sect.~\ref{sec:2}.

%% Please note that the first line of text that follows a heading is not indented, whereas the first lines of all subsequent paragraphs are.

%% For typesetting numbered lists we recommend to use the \verb|enumerate| environment -- it will automatically render Springer's preferred layout.

%% \begin{enumerate}
%% \item{Livelihood and survival mobility are oftentimes coutcomes of uneven socioeconomic development.}
%% \begin{enumerate}
%% \item{Livelihood and survival mobility are oftentimes coutcomes of uneven socioeconomic development.}
%% \item{Livelihood and survival mobility are oftentimes coutcomes of uneven socioeconomic development.}
%% \end{enumerate}
%% \item{Livelihood and survival mobility are oftentimes coutcomes of uneven socioeconomic development.}
%% \end{enumerate}


%% \subparagraph{Subparagraph Heading} In order to avoid simply listing headings of different levels we recommend to let every heading be followed by at least a short passage of text. Use the \LaTeX\ automatism for all your cross-references and citations as has already been described in Sect.~\ref{sec:2}, see also Fig.~\ref{fig:2}.

%% Please note that the first line of text that follows a heading is not indented, whereas the first lines of all subsequent paragraphs are.

%% For unnumbered list we recommend to use the \verb|itemize| environment -- it will automatically render Springer's preferred layout.

%% \begin{itemize}
%% \item{Livelihood and survival mobility are oftentimes coutcomes of uneven socioeconomic development, cf. Table~\ref{tab:1}.}
%% \begin{itemize}
%% \item{Livelihood and survival mobility are oftentimes coutcomes of uneven socioeconomic development.}
%% \item{Livelihood and survival mobility are oftentimes coutcomes of uneven socioeconomic development.}
%% \end{itemize}
%% \item{Livelihood and survival mobility are oftentimes coutcomes of uneven socioeconomic development.}
%% \end{itemize}

%% \begin{figure}[t]
%% \sidecaption[t]
% Use the relevant command for your figure-insertion program
% to insert the figure file.
% For example, with the option graphics use
%% \includegraphics[scale=.65]{figure}
%
% If not, use
%\picplace{5cm}{2cm} % Give the correct figure height and width in cm
%
%% \caption{Please write your figure caption here}
%% \label{fig:2}       % Give a unique label
%% \end{figure}

%% \runinhead{Run-in Heading Boldface Version} Use the \LaTeX\ automatism for all your cross-references and citations as has already been described in Sect.~\ref{sec:2}.

%% \subruninhead{Run-in Heading Italic Version} Use the \LaTeX\ automatism for all your cross-refer\-ences and citations as has already been described in Sect.~\ref{sec:2}\index{paragraph}.
% Use the \index{} command to code your index words
%
% For tables use
%
%% \begin{table}
%% \caption{Please write your table caption here}
%% \label{tab:1}       % Give a unique label
%
% For LaTeX tables use
%
%% \begin{tabular}{p{2cm}p{2.4cm}p{2cm}p{4.9cm}}
%% \hline\noalign{\smallskip}
%% Classes & Subclass & Length & Action Mechanism  \\
%% \noalign{\smallskip}\svhline\noalign{\smallskip}
%% Translation & mRNA$^a$  & 22 (19--25) & Translation repression, mRNA cleavage\\
%% Translation & mRNA cleavage & 21 & mRNA cleavage\\
%% Translation & mRNA  & 21--22 & mRNA cleavage\\
%%Translation & mRNA  & 24--26 & Histone and DNA Modification\\
%%\noalign{\smallskip}\hline\noalign{\smallskip}
%%\end{tabular}
%%$^a$ Table foot note (with superscript)
%%\end{table}
%
%% \section{Section Heading}
%%\label{sec:3}
% Always give a unique label
% and use \ref{<label>} for cross-references
% and \cite{<label>} for bibliographic references
% use \sectionmark{}
% to alter or adjust the section heading in the running head
%% Instead of simply listing headings of different levels we recommend to let every heading be followed by at least a short passage of text. Furtheron please use the \LaTeX\ automatism for all your cross-references and citations as has already been described in Sect.~\ref{sec:2}.

%% Please note that the first line of text that follows a heading is not indented, whereas the first lines of all subsequent paragraphs are.

%%If you want to list definitions or the like we recommend to use the Springer-enhanced \verb|description| environment -- it will automatically render Springer's preferred layout.

%%\begin{description}[Type 1]
%%\item[Type 1]{That addresses central themes pertainng to migration, health, and disease. In Sect.~\ref{sec:1}, Wilson discusses the role of human migration in infectious disease distributions and patterns.}
%%\item[Type 2]{That addresses central themes pertainng to migration, health, and disease. In Sect.~\ref{subsec:2}, Wilson discusses the role of human migration in infectious disease distributions and patterns.}
%%\end{description}

%%\subsection{Subsection Heading} %
%% In order to avoid simply listing headings of different levels we recommend to let every heading be followed by at least a short passage of text. Use the \LaTeX\ automatism for all your cross-references and citations citations as has already been described in Sect.~\ref{sec:2}.

%% Please note that the first line of text that follows a heading is not indented, whereas the first lines of all subsequent paragraphs are.

%% \begin{svgraybox}
%% If you want to emphasize complete paragraphs of texts we recommend to use the newly defined Springer class option \verb|graybox| and the newly defined environment \verb|svgraybox|. This will produce a 15 percent screened box 'behind' your text.

%% If you want to emphasize complete paragraphs of texts we recommend to use the newly defined Springer class option and environment \verb|svgraybox|. This will produce a 15 percent screened box 'behind' your text.
%% \end{svgraybox}


%% \subsubsection{Subsubsection Heading}
%%Instead of simply listing headings of different levels we recommend to let every heading be followed by at least a short passage of text. Furtheron please use the \LaTeX\ automatism for all your cross-references and citations as has already been described in Sect.~\ref{sec:2}.

%% Please note that the first line of text that follows a heading is not indented, whereas the first lines of all subsequent paragraphs are.

%% \begin{theorem}
%% Theorem text goes here.
%% \end{theorem}
%
% or
%
%% \begin{definition}
%% Definition text goes here.
%% \end{definition}

%% \begin{proof}
%\smartqed
%% Proof text goes here.
%% \qed
%% \end{proof}

%%\paragraph{Paragraph Heading} %
%% Instead of simply listing headings of different levels we recommend to let every heading be followed by at least a short passage of text. Furtheron please use the \LaTeX\ automatism for all your cross-references and citations as has already been described in Sect.~\ref{sec:2}.

%% Note that the first line of text that follows a heading is not indented, whereas the first lines of all subsequent paragraphs are.
%
% For built-in environments use
%
%%\begin{theorem}
%%Theorem text goes here.
%%\end{theorem}
%
%%\begin{definition}
%%Definition text goes here.
%%\end{definition}
%
%%\begin{proof}
%%\smartqed
%% Proof text goes here.
%%\qed
%%\end{proof}
%
%% \begin{acknowledgement}
%% If you want to include acknowledgments of assistance and the like at the end of an individual chapter please use the \verb|acknowledgement| environment -- it will automatically render Springer's preferred layout.
%% \end{acknowledgement}
%
%% \section*{Appendix}
%% \addcontentsline{toc}{section}{Appendix}
%
%% When placed at the end of a chapter or contribution (as opposed to at the end of the book), the numbering of tables, figures, and equations in the appendix section continues on from that in the main text. Hence please \textit{do not} use the \verb|appendix| command when writing an appendix at the end of your chapter or contribution. If there is only one the appendix is designated ``Appendix'', or ``Appendix 1'', or ``Appendix 2'', etc. if there is more than one.

%% \begin{equation}
%% a \times b = c
%% \end{equation}
% Problems or Exercises should be sorted chapterwise
%% \section*{Problems}
%% \addcontentsline{toc}{section}{Problems}
%
% Use the following environment.
% Don't forget to label each problem;
% the label is needed for the solutions' environment
%% \begin{prob}
%% \label{prob1}
%% A given problem or Excercise is described here. The
%% problem is described here. The problem is described here.
%% \end{prob}

%% \begin{prob}
%% \label{prob2}
%% \textbf{Problem Heading}\\
%% (a) The first part of the problem is described here.\\
%% (b) The second part of the problem is described here.
%% \end{prob}


