%!TEX root = ../../Heun_Dale_Haney_A_dynamic_approach_to_input_output_modeling.tex
%%%%%%%%%%%%%%%%%%%%% chapter.tex %%%%%%%%%%%%%%%%%%%%%%%%%%%%%%%%%
%
% sample chapter
%
% Use this file as a template for your own input.
%
%%%%%%%%%%%%%%%%%%%%%%%% Springer-Verlag %%%%%%%%%%%%%%%%%%%%%%%%%%
%\motto{Use the template \emph{chapter.tex} to style the various elements of your chapter content.}
\motto{The economy is a wholly owned subsidiary of the environment, not the reverse.
**** Need citation. ****

\hfill---\emph{Herman E. Daly}}

%%%%%%%%%%%%%%%%%%%%%%%%%%%%%%%%%%
%%%%%%%%%% Implications %%%%%%%%%%
%%%%%%%%%%%%%%%%%%%%%%%%%%%%%%%%%%
\chapter{Implications}
% Always give a unique label
\label{chap:implications}
% use \chaptermark{} to alter or adjust the chapter heading in the running head
\chaptermark{Implications}
%%%%%%%%%%%%%%%%%%%%%%%%%%%%%%%%%%
%%%%%%%%%%%%%%%%%%%%%%%%%%%%%%%%%%
%%%%%%%%%%%%%%%%%%%%%%%%%%%%%%%%%%

\abstract*{[NEED TO ADD ABSTRACT HERE]}

%% \abstract{Each chapter should be preceded by an abstract (10--15 lines long) that summarizes the content. The abstract will appear \textit{online} at \url{www.SpringerLink.com} and be available with unrestricted access. This allows unregistered users to read the abstract as a teaser for the complete chapter. As a general rule the abstracts will not appear in the printed version of your book unless it is the style of your particular book or that of the series to which your book belongs.\newline\indent
%% Please use the 'starred' version of the new Springer \texttt{abstract} command for typesetting the text of the online abstracts (cf. source file of this chapter template \texttt{abstract}) and include them with the source files of your manuscript. Use the plain \texttt{abstract} command if the abstract is also to appear in the printed version of the book.}

%% Use the template \emph{chapter.tex} together with the Springer document class SVMono (monograph-type books) or SVMult (edited books) to style the various elements of your chapter content in the Springer layout.


Several implications can be drawn from the detailed development 
of our framework for materials and energy accounting 
(in Chapters~\ref{chap:materials}--\ref{chap:intensity}).
In the sections below, we discuss 
implications for the I-O method itself,
implications for economic ``development,''
implications for recycling, reuse, and dematerialization, and
comparisons with the idea of a steady-state economy.
We begin by examining the I-O method itself 
through the lens of our framework.


%%%%%%%%%% Implications for the I-O method %%%%%%%%%%
\section{Implications for the I-O method}
\label{sec:Implications_for_IO}
%%%%%%%%%%

Extension of the Leontief\index{Leontief} 
Input-Output method\index{input-output method}
for energy analysis has allowed energy analysts to estimate 
the energy intensity\index{energy intensity}
of economic products~($\boldsymbol{\varepsilon}$). 
As discussed in Section~\ref{sec:Value_Methodology},
we do not take the ability to estimate energy intensity as a license
to declare an intrinsic ``energy theory of value.''
\index{theory of value!energy}
Rather, we belive that energy intensity~($\boldsymbol{\varepsilon}$) is an 
important and useful metric that can assess 
the energy performance of economies,
even within the prevailing subjective theory of value\index{theory of value!subjective}
that underlies modern economics.
It is important to consider the assumptions behind
the literature's presentation of the energy I-O method 
for estimating the energy intensity of economic output
before drawing implications from our framework 
for the energy I-O method.

As we investigate, we will use the following
coordinates of analysis:
product-based vs.\ physical accounting frameworks,
whether capital stock is included in the accounting framework, and
whether energy input from society to the economy is included.
(See Figure~\ref{fig:coords_of_analysis}.)
We will end with our suggestion for how best to estimate
$\boldsymbol{\varepsilon}$ within
a materials, energy, and value accounting system.

\begin{figure}[!ht]
\centering{}
\includegraphics[width=0.8\linewidth]{Part_3/Chapter_Implications/Images/Grid.pdf}
\caption[Coordinates of analysis for implications for the I-O method]{Coordinates of analysis for implications for the I-O method.}
\label{fig:coords_of_analysis}
\end{figure}


%+++++++++ I-O implications: product-based vs. physical ++++++++++
\subsection{Product-based vs.\ physical approaches}
\label{sec:prod_vs_physical}
%+++++++++

The distinction between product-focused and physical
accounting frameworks is located in the columns
of Figure~\ref{fig:coords_of_analysis}.
A physical accounting framework strictly follows materials 
through the economy. 
Embodied energy is accounted with the material stock or material flow
in which it resides.
For example, energy embodied within wastes ($\vec{B}_{waste}$) 
is not assigned to economic products. 
Rather, the energy embodied in wastes flows from sectors 
to the biosphere \emph{with the waste material}.
In contrast, a product-focused accounting framework assigns 
energy embodied in wastes to the products of the sector.
Both product-based and physical accounting frameworks
assign direct energy ($\dot{E}$) consumed by each sector
to the products of each sector.

Equation~\ref{eq:B_prod_physical_excluding_K} 
describes the outflow of embodied energy from sector $j$, 
exclusive of capital stock, 
for a physical accounting system 
that neglects capital stock (upper-right quadrant of Figure~\ref{fig:coords_of_analysis}).\footnote{Equation~\ref{eq:B_prod_physical_excluding_K} 
	is used for illustrative purposes only. 
	A physical accounting framework would necessarily 
	include both flows and stocks of capital.
	Thus, the upper-right quadrant of Figure~\ref{fig:coords_of_analysis}
	(physical accounting framework that neglects capital)
	is labeled as nonsensical.}

\begin{equation} \label{eq:B_prod_physical_excluding_K}
	\dot{B}_{j}^{'}
	= \sum\limits_{i=1}^{n} \dot{B}_{ij}^{'} 
	- \dot{B}_{j,waste}
	+ \dot{Q}_{j0}
\end{equation}

\noindent{}Variables written with a ``prime''
(e.g.~$\dot{B}_{j}^{'}$) indicate definitions and terms that include only 
resource~($\dot{R}$) and short-lived~($\dot{S}$) flows 
and exclude capital flows~($\dot{K}$) and stock~($K$).
The term~$\dot{B}_{j,waste}$ represents the energy embodied 
within wasted resource~($\dot{R}_{j0}$) 
and short-lived~($\dot{S}_{j0}$) material flows.
The term is subtracted, because waste material flows are not assigned 
to the product~($\dot{B}_{j}^{'}$) in a physical accounting system.

In contrast, Equation~\ref{eq:B_prod_product_excluding_K} describes the outflow 
of embodied energy from sector~$j$,
exclusive of capital stock ($\dot{B}_{j}^{'}$)
for a product-focused accounting framework 
(upper left quadrant of Figure~\ref{fig:coords_of_analysis}).

\begin{equation} \label{eq:B_prod_product_excluding_K}
	\dot{B}_{j}^{'}
	= \sum\limits_{i=1}^{n} \dot{B}_{ij}^{'} 
	+ \dot{Q}_{j0}
\end{equation}

\noindent{}Notice that Equation~\ref{eq:B_prod_product_excluding_K} 
does not subtract the energy embodied in waste resource and short-lived material
flows ($\dot{B}_{j,waste}$) on the right side of the equation, 
because product-focused accounting systems assign energy embodied in wastes
to products.


%+++++++++ I-O implications: Capital stock ++++++++++
\subsection{Capital stock}
\label{sec:capital_stock}
%+++++++++

The rows of Figure~\ref{fig:coords_of_analysis}
represent the role of capital stock in an accounting framework.
During the earliest years of the energy-focused I-O method 
(prior to the mid-1970s) capital was ignored, 
both as a flow and as a stock.
In essence, the state of the art was located in the upper-left quadrant
of Figure~\ref{fig:coords_of_analysis}.
In time, Kirkpatrick~\cite{Kirkpatrick:1974te}, 
Bullard and Herendeen~\cite{Bullard-III:1975aa},
and Casler~\cite{Casler:1983uy} attempted to 
include inflows of capital stock in a product-focused
accounting framework, thereby moving the state of the art 
to the lower-left quadrant of Figure~\ref{fig:coords_of_analysis}.

We agree with this move, because of the many ways in which
capital stock is important for economies.
To explain the importance of capital stock, 
we can relu upon the ecosystem work of Eugene Odum
and rely upon Herman Daly to construct bridges 
between ecosystems and economies.

In 1969, Odum outlined a number of 
defining characteristics of both \emph{developmental},
i.e.\ growing,
and \emph{mature} ecosystems in terms of key
properties of the system.\cite{Odum1969}
Ecosystems cannot
grow indefinitely in their (photosynthetic)~production~($P$)
due mainly to the necessity of increasing maintenance
demands as the amount of (biomass) capital
stock~($B$) increases.
Eventually, all production is used in this manner
and growth ceases 
$\left(\frac{\mathrm{d}P}{\mathrm{d}t} = 0\right)$.

In the early stages of ecosystem development,
the the energy production per unit of biomass stock ($\frac{P}{B}$)
is high.
As the ecosystem approaches maturity,
this ratio decreases.
Put another way,
the biomass stock (maintained) per unit of energy produced
(the inverse ratio $\frac{B}{P}$)
starts low and asymptotically increases to a maximum
when growth (in both $P$ and $B$) has ceased. 
The value of $\frac{B}{P}$ at the asymptote may be high or low\footnote{The
	value of $\frac{B}{P}$ at maturity (and the time taken to reach it)
	``may vary not only with different climatic 
	and physiographic situations but also with
	different ecosystem attributes in the same physical 
	environment.''~\cite[p.263]{Odum1969}}
and may therefore be considered a measure of 
the ``efficiency'' to which the ecosystem applies
energy production toward
the goal of maintaining biomass stock.
 
Turning back to economies,
Daly has, in our view,
correctly applied this concept to societal patterns
of economic consumption.\cite{Daly1995}
Our framework analogously suggests that
as capital stock~($\vec{B}_{K}$) increases,
an increasing flow of energy supply~($\vec{E}_{0}$)
will be needed to maintain that stock.\footnote{Today's economies 
	(and economic models and economic assumptions) are still focused
	on the objective of growth.
	If energy supply rates ($\vec{E}_{0}$) are constrained,
	these dynamics provide a possible reason for the difficulty
	of maintaining high levels of economic growth
	in mature economies.
	Eventually, we believe,
	we must learn to maximize the $\frac{B}{P}$
	ratios of our economies 
	$\left(\frac{\vec{B}_{K}}{\vec{E}_{0}}\right)$.}
Thus, it is important to account for capital stock in 
a material, energy, and value accounting framework.

To see the effect of the move from the upper-left to the lower-left
quadrant of Figure~\ref{fig:coords_of_analysis}, 
it is important to understand clearly both the assumptions and data that were used.
Energy analysts in the mid-1970s were utilizing the BEA I-O tables,
which do not include captial flows. 
Thus, this early literature implicitly assumes that 

\begin{equation} \label{eq:a_prime_lit}
	a_{ij}^{'} 
	\equiv \frac{\dot{X}_{\dot{R}_{ij}} + \dot{X}_{\dot{S}_{ij}}}
				{\dot{X}_{\dot{R}_{j}} + \dot{X}_{\dot{S}_{j}}}.
\end{equation}

\noindent{}Comparison between Equations~\ref{eq:aij_def_expanded}
and~\ref{eq:a_prime_lit}
highlights the fact that the early literature neglects flows of capital stock
($\dot{X}_{\dot{K}}$).
Thus, the the input-output matrix in the early literature
($\vec{A^{'}}$) is

\begin{equation} \label{eq:A_matrix_def_literature}
	\vec{A}^{'} 
	=
	\begin{bmatrix}
		a_{22}^{'} & a_{23}^{'}	\\
		a_{32}^{'} & a_{33}^{'}	
	\end{bmatrix}.
\end{equation}

\noindent{}Furthermore, the early literature implicitly defines 
energy intensity as

\begin{equation}
	\varepsilon_{j}^{'} 
	= \frac{\dot{T}_{j}^{'}}{\dot{X}_{j}^{'}},
\end{equation}

\noindent{}with

\begin{equation}
	\dot{T}_{j}^{'} \equiv \dot{T}_{\dot{R}_{j}} + \dot{T}_{\dot{S}_{j}}
\end{equation}

\noindent{}and

\begin{equation}
	\dot{X}_{j}^{'} \equiv \dot{X}_{\dot{R}_{j}} + \dot{X}_{\dot{S}_{j}}.
\end{equation}

\noindent{}The above equations implicitly ignore both value and energy embodied within
flows of capital.
In contrast, our framework (Equation~\ref{eq:X_hat_matrix_def})
includes capital flows explicitly:

\begin{equation}
	\hat{\vec{X}} = \hat{\vec{X}}_{\dot{R}} + \hat{\vec{X}}_{\dot{S}} + \hat{\vec{X}}_{\dot{K}}.
\end{equation}

The implicit assumptions of the early energy I-O literature 
are consistent with the upper-left quadrant
of Figure~\ref{fig:coords_of_analysis}. 
The energy intensity equation
found in most of the early literature is

\begin{equation} \label{eq:intensity_upper_left}
	\boldsymbol{\varepsilon}^{'}
	= {\left( \vec{I} - {\vec{A}^{'}}^{\mathrm{T}} \right)}^{-1}
		{\left( \hat{\vec{X}}^{'}  \right)}^{-1}
		\vec{E}_{0}.
\end{equation}

Bullard and Herendeen~\cite{Bullard-III:1975aa}, 
following Kirkpatrick~\cite{Kirkpatrick:1974te},
added flows of capital stock as inputs 
to each sector~\cite[Figure~5]{Bullard-III:1975aa},
and, in so doing, changed Equation~\ref{eq:intensity_upper_left}
to Equation~\ref{eq:intensity_lower_left}.

\begin{equation} \label{eq:intensity_lower_left}
	\boldsymbol{\varepsilon}^{'}
	= {\left( \vec{I} - {\vec{A}^{'}}^{\mathrm{T}} \right)}^{-1}
		{\left( \hat{\vec{X}}^{'}  \right)}^{-1}
		\left[ \vec{E}_{0}
			  + 0.5 \sum\limits_{i=1}^{n} \dot{B}_{\dot{K}_{ij}} \right]
\end{equation}

\noindent{}They counted embodied energy 
from incoming capital stock only if it was used
for replacement.\cite[p.~488]{Bullard-III:1975aa}
Consequently, they did not count incoming energy embodied 
in capital if the incoming capital was used 
to increase the stock of capital within a sector.
In fact, Bullard and Herendeen's
product-focused accounting system
did not include an embodied energy stock 
for each economic sector ($\vec{B}$).
They assumed that half of the incoming capital 
went toward replacement, leading to the 0.5 coefficient 
in Equation~\ref{eq:intensity_lower_left}.
We note that there was no attempt to redefine $\vec{A}^{'}$
or $\boldsymbol{\varepsilon}^{'}$ to include flows of capital stock.
These early researchers moved from the upper-left quadrant
to the lower-left quadrant of Figure~\ref{fig:coords_of_analysis}.
And, Equation~\ref{eq:intensity_lower_left} represents 
a partial step toward developing 
a method for estimating energy intensity ($\boldsymbol{\varepsilon}$)
that fully accounts for capital stock.

As stated above, we agree with Kirkpatrick~\cite{Kirkpatrick:1974te}, 
Bullard and Herendeen~\cite{Bullard-III:1975aa},
and Casler~\cite{Casler:1983uy} that capital stock
is important and should be included in an accounting framework
(i.e., we should be on the lower half of Figure~\ref{fig:coords_of_analysis}).
But, we recommend that inclusion of capital stock should
be done within a \emph{physical} accounting framework, 
i.e.\ we should move from the lower-left to the lower-right
quadrant of Figure~\ref{fig:coords_of_analysis}.
Specifically, capital should be included 
(a) as a stock that can accumulate and (b) on all outflow streams,
not just inflow streams.
Furthermore, capital should be included in the definitions 
of the input-output matrix ($\vec{A}$) and 
energy intensity ($\boldsymbol{\varepsilon}$).

Our recommendation is informed by the work of 
Odum~\cite{Odum1969} and Daly~\cite{Daly1995}
and is based on the belief that
accounting for stocks of capital is important for developing a coherent view
of the structure of an economy. 
Stocks of capital are essential to the production process:
without machines and factories, cars cannot be produced. 
Thus, the buildup of capital stock (and associated embodied energy) 
within economic sectors is an essential aspect of industrialization.
Carefully tracking (on a physical, as opposed to financial, basis) 
capital stock is essential for understanding the network effects of 
upstream energy demand as new industries and products arise 
(e.g., electric vehicles). 
Finally, maintenance of captial stock becomes an important 
driver of demand, especially in mature economies.

In a physical accounting system that includes capital stock
(lower-right quadrant of Figure~\ref{fig:coords_of_analysis}), 
energy embodied within accumulated capital stock
is not assigned to products ($\vec{P}$); 
rather, accumulated embodied energy is assigned to a stock of embodied energy
for each sector ($\vec{B}_{K}$).
And, the stock of embodied energy ($\vec{B}_{K}$)
can be depreciated.

A physical accounting framework that fully includes capital stock
(lower-right quadrant of Figure~\ref{fig:coords_of_analysis}) 
is described by Equation~\ref{eq:intensity_lower_right_no_society}. 

\begin{equation} \label{eq:intensity_lower_right_no_society}
	\boldsymbol{\varepsilon}
	= {\left( \vec{I} - {\vec{A}}^{\mathrm{T}} \right)}^{-1}
		{\left( \hat{\vec{X}}  \right)}^{-1}
		\left[ \vec{E}_{0}
			  - \frac{\mathrm{d}\vec{B}_{K}}{\mathrm{d}t}
			  - \vec{B}_{waste}
			  - \hat{\boldsymbol{\gamma}}_{B} \vec{B}_{K} \right].
\end{equation}

\noindent{}Differences between Equation~\ref{eq:intensity_lower_right_no_society}
and Equation~\ref{eq:intensity_lower_left} include:

\begin{itemize}
	\item{$\boldsymbol{\varepsilon}$ appears rather than $\boldsymbol{\varepsilon}^{'}$,
	because both embodied energy and value of capital is included 
	in the definition of energy intensity,}
	\item{Equation~\ref{eq:intensity_lower_right_no_society} does not have
	the $\sum\limits_{i=1}^{n} \dot{B}_{ij}$ term, 
	because embodied energy flows are already included in $\vec{A}$,}
	\item{$\hat{\vec{X}}$ appears rather than $\hat{\vec{X}}^{'}$,
	because energy embodied in product streams is include in $\hat{\vec{X}}$,}
	\item{Equation~\ref{eq:intensity_lower_right_no_society} subtracts
	accumulation $\left( \frac{\mathrm{d}\vec{B}_{K}}{\mathrm{d}t} \right)$
	and depreciation ($\hat{\boldsymbol{\gamma}}_{B} \vec{B}_{K}$)
	of energy embodied in capital stock,
	because energy embodied in the stock of capital ($K$)
	for a sector
	is not assigned to products of the sector, and}
	\item{Equation~\ref{eq:intensity_lower_right_no_society} subtracts $\vec{B}_{waste}$,
	because energy embodied in waste products is not assigned to 
	products of the sector.}
\end{itemize}

There are two topics related to capital stock that are worthy of consideration:
waste flows and an accounting equation for capital stock.


%--------- I-O implications: Capital stock: Waste ----------
\subsection{Waste flows}
\label{sec:waste_flows}
%---------

We are unaware of any estimates of the energy embodied in wasted
material in an economy ($\vec{B}_{waste}$).  
But, it may be possible to develop a metric for the resource material effiency of an 
economic sector ($\eta_{\dot{R}}$)\nomenclature[h]{$\eta_{\dot{R}}$}{resource efficiency [kg/kg]} 
such that

\begin{equation} \label{eq:manufacturing_effiency}
	\eta_{\dot{R},j}
	\equiv \frac{\dot{P}_{j}}{\sum\limits_{i=1}^{n} \dot{R}_{ij}}.
\end{equation}

\noindent{}With the above definition, 
the scrap rate for resources could be expressed as
$(1~-~\eta_{\dot{R}}) \sum\limits_{i=1}^{n} \dot{R}_{ij}$.
Allwood et.\ al.~\cite[p. 193]{allwood2012sustainable} 
used a process-based approach to manufacturing efficiencies
for metals used in manufacturing. 
The data are summarized in Table~\ref{tab:scrap_rates}.

\begin{table}
\caption[Manufacturing efficiencies for selected goods]{Manufacturing efficiencies (Equation~\ref{eq:manufacturing_effiency})
for selected manufactured goods.\cite{allwood2012sustainable}}
\begin{center}
\begin{tabular} {r @{\hspace{2em}} l}
	\toprule
	Product & Manufacturing Efficiency [\%] \\
	\midrule
	Steel I-beam             & 90 \\
	Car Door Panel           & 50 \\
	Aluminium Drink Can      & 50 \\
	Aircraft Wing Skin Panel & 10 \\
	\bottomrule
\end{tabular}
\end{center}
\label{tab:scrap_rates}
\end{table}

Furthermore, one could assume that the rate
of short-lived materials ($\dot{S}$) used by a sector could be given as a 
fraction of the resource ($\dot{R}$) use rate such that:

\begin{equation}
	\rho_{\dot{S},j}
	\equiv \frac{\dot{S}_{j0}}{\sum\limits_{i=1}^{n} \dot{R}_{ij}}
	= \frac{\sum\limits_{i=1}^{n} \dot{S}_{ij}}{\sum\limits_{i=1}^{n} \dot{R}_{ij}}.
\end{equation} \nomenclature[r]{$\rho_{\dot{S}}$}{ratio of short-lived material
flow to resource inputs [kg/kg]}

With the above definitions, the waste resource rate from an economic sector
can be given as

\begin{equation}
	\dot{R}_{j0} + \dot{S}_{j0}
	= (1 - \eta_{\dot{R}_{j}} + \rho_{\dot{S},j}) \sum\limits_{i=1}^{n} \dot{R}_{ij}.
\end{equation}

\noindent{}The embodied energy in the waste materials would need to be estimated
from the embodied energy of the incoming resource and short-lived material flows as

\begin{equation}
	\dot{B}_{waste,j}
	= \dot{B}_{\dot{R}_{j0}} + \dot{B}_{\dot{S}_{j0}}.
\end{equation}


%--------- I-O implications: Capital stock: capital accounting equation ----------
\subsection{Simplification via capital stock accounting equation}
\label{sec:capital_accounting}
%---------

A possible simplification to Equation~\ref{eq:intensity_lower_right_no_society}
can be obtained from a control volume around the stock of capital in sector $j$:

\begin{equation}
	\frac{\mathrm{d}B_{K_{j}}}{\mathrm{d}t}
	= \sum\limits_{i=1}^{n} \dot{B}_{\dot{K}_{ij}} 
	- \gamma_{B,j} B_{K_{j}}.
\end{equation}

We can express the incoming energy embodied in capital ($\sum_{i=1}^{n} \dot{B}_{ij}$)
as a fraction ($\alpha_{j}$) of the capital stock ($B_{K_{j}}$) as

\begin{equation}
	\alpha_{B,j}
	\equiv 
	\frac{\sum\limits_{i=1}^{n} \dot{B}_{\dot{K}_{ij}}} {B_{K_{j}}}
\end{equation}

\noindent{}for $j \in [2,n]$ and the vector $\boldsymbol{\alpha}_{B}$ as

\begin{equation}
	\vec{\boldsymbol{\alpha}_{B}}
	\equiv
	\begin{Bmatrix}
		\alpha_{B,2}	\\
		\alpha_{B,3}
	\end{Bmatrix}.
\end{equation}

\noindent{}Together with the Kronecker delta ($\delta_{ij}$), we can write

\begin{equation}
	\hat{\boldsymbol{\alpha}}_{B}
	\equiv
	\delta_{ij} \boldsymbol{\alpha}_{B}
	=
	\begin{bmatrix}
		\alpha_{B,2}	& 0           \\
		0               & \alpha_{B,3}  \\
	\end{bmatrix}.
\end{equation}

\noindent{}Thus, the embodied energy accounting equation around
the stock of capital in the economy can be written in matrix form as

\begin{equation}
	\frac{\mathrm{d}\vec{B}_{K}}{\mathrm{d}t}
	= \hat{\boldsymbol{\alpha}}_{B} \vec{B}_{K}
	- \hat{\boldsymbol{\gamma}}_{B} \vec{B}_{K}.
\end{equation}

\noindent{}Rearranging slightly gives

\begin{equation} \label{eq:alpha_cap_stock}
	\hat{\boldsymbol{\alpha}}_{B} \vec{B}_{K}
	= \frac{\mathrm{d}\vec{B}_{K}}{\mathrm{d}t}
	+ \hat{\boldsymbol{\gamma}}_{B} \vec{B}_{K},
\end{equation}

\noindent{}which says that incoming capital ($\hat{\boldsymbol{\alpha}}_{B} \vec{B}_{K}$)
can be used to either
increase the stock of capital in the economy 
$\left( \frac{\mathrm{d}\vec{B}_{K}}{\mathrm{d}t} \right)$
or overcome depreciation ($\hat{\boldsymbol{\gamma}}_{B} \vec{B}_{K}$).
Substituting Equation~\ref{eq:alpha_cap_stock} into
Equation~\ref{eq:epsilon_leontief_with_A} gives

\begin{equation} \label{eq:epsilon_leontief_with_A_alpha}
	\boldsymbol{\varepsilon} 
	= {(\vec{I} - \vec{A}^{\mathrm{T}})}^{-1}\hat{\vec{X}}^{-1}
		\left[\vec{E}_{0} 
				+ \vec{T}_{1} 
				- \hat{\boldsymbol{\alpha}}_{B} \vec{B}_{K}
				- \vec{B}_{waste}
		\right].
\end{equation}


%+++++++++ I-O implications: Energy input from society ++++++++++
\subsection{Energy input from society}
\label{sec:energy_from_society}
%+++++++++

In Sections~\ref{sec:prod_vs_physical}
and~\ref{sec:capital_stock} above,
we implicitly assumed that society 
(final consumption, Sector 1 in our model)
contributes negligible energy to the economy.
Thus, all vectors and matrices in Equation~\ref{eq:intensity_lower_right_no_society}
involve Sectors~2--n, but not Sector~1.

Energy input from society to the economy ($\vec{T}_{1}$)
is ``muscle work'' supplied by working humans 
and draught animals.\cite{Ayres:2003ec,Ayres:2010ug,Warr:2012cg} 
This muscle work term ($\vec{T}_{1}$) should include
all upstream energy required to make the labor available.\footnote{It is important
	to note that $\vec{T}_{1}$ should include all upstream energy,
	because at this point in the development of our framework,
	we are assuming that Final Consumption (Sector 1) is exogenous to the economy 
	(Sectors 2--n), 
	and upstream energy consumption needs to be included manually.
	However, in Section~\ref{sec:what_is_endogenous}, we show that Final Consumption
	can be endogenized.
	Once endogenized, the energy intensity of Final Consumption ($\varepsilon_{1}$) 
	will automatically include the upstream energy required to make labor available.
	(See Appendix~\ref{chap:infinite_series}.)

	It is important to note, too, that labor can have very high energy intensity, 
	because $\varepsilon_{1}$ includes the energy required to supply food and transport
	to workers.} 
Equation~\ref{eq:epsilon_leontief_with_A} 
adds the effect of energy input from society to the economy,
effectively moving from the top half to the lower half of the lower-right quadrant
in Figure~\ref{fig:coords_of_analysis}.

\begin{equation}
	\boldsymbol{\varepsilon} 
	= {(\vec{I} - \vec{A}^{\mathrm{T}})}^{-1}\hat{\vec{X}}^{-1}
		\left[\vec{E}_{0} 
				+ \vec{T}_{1} 
				- \frac{\mathrm{d}\vec{B}_{K}}{\mathrm{d}t} 
				- \vec{B}_{waste}
				- \hat{\boldsymbol{\gamma}}_{B}\vec{B}_{K}
		\right].\tag{\ref{eq:epsilon_leontief_with_A}}
\end{equation}

For industrialized economies, the direct energy component ($\vec{E}_{1}$) 
of muscle work ($\vec{T}_{1}$)
is likely to provide only a small fraction
of the energy input from fossil fuels ($\vec{E}_{0}$).
But, the embodied energy of the muscle work ($\vec{B}_{1}$) is likely to be large.
For agrarian\index{economy!agrarian} 
and developing economies\index{economy!developing}, 
$\vec{T}_{1}$ and $\vec{E}_{0}$ 
could be on the same order of magnitude.
For both industrial and agrarian economies,
neglecting $\vec{T}_{1}$ could cause errors
in estimates of $\boldsymbol{\varepsilon}$.
To the extent that $\vec{T}_{1}$ 
is significant relative to $\vec{E}_{0}$,
neglecting $\vec{T}_{1}$
will underpredict the energy intensity of economic output.


%+++++++++ I-O implications: Recommendation ++++++++++
\subsection{Recommendation}
\label{sec:I-O_recommendation}
%+++++++++

Sections~\ref{sec:prod_vs_physical}--\ref{sec:energy_from_society} 
discussed three factors that affect the form of the
energy intensity equation: 
product-focused vs.\ physical accounting frameworks,
whether capital stock is included, and
whether energy input from society is included.
The three factors are summarized in Figure~\ref{fig:coords_of_analysis}.

At this point, it is instructive to look back at the 
product-focused vs.\ physical discussion in Section~\ref{sec:prod_vs_physical}.
We understand the argument for including capital stock in a product-focused
accounting framework (lower-left quadrant of Figure~\ref{fig:coords_of_analysis}):
capital stock and waste exist 
solely due to product demand, 
therefore energy embodied in capital and waste should be assigned to products. 
However, a product-focused framework that includes capital stock (lower-left quadrant of
Figure~\ref{fig:coords_of_analysis})
masks structural aspects of economies
that we believe are essential to fully understanding how and why energy flows 
through economies, namely the accumulation of capital
and associated energy embodied within sectors.

The metabolic metaphor provides guidance here. 
If we were to create a model of an organism that neglects 
tissues that accumulate embodied energy,
the organism (in the model) has nothing with which to 
absorb, process, waste, or otherwise exchange
material with the biosphere.
The organism doesn't physically exist (in the model)!
Neglecting to account for the stock of capital (and its embodied energy) 
is tantamount to assuming that economic production occurs out of nothing!
Accounting for capital stock is essential.

For our framework, we chose a physical accounting approach
(which puts us in the right column of Figure~\ref{fig:coords_of_analysis}).
We chose the physical approach primarily because of our belief that 
capital is an important aspect of economies,
and the physical accounting approach
properly includes a stock of capital for each sector of the economy.
Product-based accounting frameworks mask crucial aspects 
of why and how energy flows through economies will be masked.
We acknowledge that the choice of a physical accounting framework necessitates
careful tracking of capital flows (and associated embodied energy)
through the economy. 

Finally, we suggest that accounting for energy input from society
to the economy is important, 
and we need to be in the lower half of the bottom-right quadrant
of Figure~\ref{fig:coords_of_analysis}.
So, the state of the art has moved from the nascent energy I-O literature
located in the upper-left quadrant of Figure~\ref{fig:coords_of_analysis}
as represented by Equation~\ref{eq:intensity_upper_left}
through the lower-left quadrant of Figure~\ref{fig:coords_of_analysis}
as represented by Equation~\ref{eq:intensity_lower_left}
to the lower half of the bottom-right quadrant 
of Figure~\ref{fig:coords_of_analysis}
as represented by Equation~\ref{eq:epsilon_leontief_with_A_alpha}.

The implication of this detailed development of a framework for
material, energy, and value accounting on the energy I-O method is 
some suggested enhancements to the energy I-O method, 
including

\begin{itemize}
	\item{conversion to a physical accounting framework such as the one we propose herein,}
	\item{physical (as opposed to financial) tracking 
	of accumulated capital stock within economic sectors,}
	\item{redefinition of $\vec{A}$ and $\boldsymbol{\varepsilon}$ to include
	embodied energy on both inflows and outflows of material, and}
	\item{use of Equation~\ref{eq:epsilon_leontief_with_A_alpha} instead of
	Equations~\ref{eq:intensity_upper_left} or~\ref{eq:intensity_lower_left}
	for estimating energy intensity ($\boldsymbol{\varepsilon}$)
	of economic sectors within an economy.}
\end{itemize}
 

%%%%%%%%%% Implications %%%%%%%%%%
\section{Implications for economic ``development''}
\label{sec:implications_for_development}
%%%%%%%%%%

**** Becky, write a paragraph here about alternative measures of development. 
Discuss Daly's Genuine Progress Indicator (which includes natural capital, Mik to verify)
and Gross National Happiness and other. ****

Economic ``development''\footnote{We choose to use the word ``development'' to
describe expanding economies, despite significant misgivings about the term. 
The unambiguously positive connotations of the words ``development'' and ``developed''
fail to capture the nuances of travel along economic development paths:
there are so many ways in which life experience in ``developed'' countries 
is both better and worse than life in ``developing'' countries.
We hope to convey our misgivings by surrounding these words with quotation marks in this text.}
is usually ``measured''
by Gross Domestic Product\index{Gross Domestic Product} (GDP).
With reference to Figure~\ref{fig:C_value}, $GDP$ is calculated by

\begin{equation} \label{eq:GDP_def}
	GDP
	= \sum\limits_{j=2}^{n} \dot{X}_{j}
\end{equation}

\noindent{}where $n$ is the number of sectors in the economy.
Equation~\ref{eq:GDP_def} clearly shows that 
$GDP$ is a \emph{flow} of value in units of \$/year.

As a given economy moves along its ``development'' path,
sectors within the economy accumulate capital stock 
($K$, typically expressed in units of dollars)
and associated embodied energy 
($B_{K}$, expressed in units of joules).
If we turn this around, 
accumulation of embodied energy in economic sectors and society 
could be considered a \emph{proxy} for ``development.''\footnote{Embodied energy 
as a proxy for development may be overly focused on capital stock, 
therefore one-dimensional, and reductive, 
but GDP and other measures are open to similar criticism.}
Equation~\ref{eq:Dev_Integral_Economy} indicates how accumulated
embodied energy in the capital stock 
of an economy ($\vec{B}_{K}$) could be calculated:

\begin{equation} \label{eq:Dev_Integral_Economy}
	\vec{B}_{K}(t) 
	= \vec{B}_{K}(0) 
	+ \int_{t=0}^{t=t} \frac{\mathrm{d}\vec{B}_{K}}{\mathrm{d}t}\mathrm{d}t,
\end{equation}

\noindent{}where $\vec{B}_{K}$ is given by Equation~\ref{eq:B_vec_def}.
Equation~\ref{eq:Dev_Integral_Economy} clearly shows
that energy embodied energy in capital stock ($\vec{B}_{K}$) 
is a \emph{stock} (in units of joules), not a flow.

The behavior of $\vec{B}_{K}$ with respect to
$\left. \frac{\mathrm{d}\vec{B}_{K}}{\mathrm{d}t} \right|_{\mathrm{other}}$ 
is vitally important. 
As an economy transitions from agrarian to industrialized, 
its capital stock ($K$) and associated embodied energy ($B_{K}$)
grows ever larger. 
The outflow of depreciated capital stock and its associated embodied energy 
will occur at a faster rate, too.
As increasingly large amounts of energy are embodied 
in the capital stock of an economy ($B_{K}$), 
Equation~\ref{eq:matrix_leontief} shows that
increasingly large energy extraction rates ($\vec{E}_{0}$) 
are required to maintain capital stock 
in the sectors of the economy
to offset the effects of depreciation 
($\hat{\boldsymbol{\gamma}}_{K} \vec{B}_{K}$),
assuming that $\frac{\mathrm{d}\vec{B}_{K}}{\mathrm{d}t} \ge 0$ 
is desired. 

During a period of rapid industrialization and infrastructure build-out, 
we expect both $GDP$ and energy embodied in the economy ($\vec{B}_{K}$)
to increase.
But, there is no guarantee that $GDP$ and $\vec{B}_{K}$ move 
in the same direction at all times.
Industrialized economies may experience $GDP$ growth while 
the stock of embodied energy in the economy ($\vec{B}_{K}$) remains nearly constant,
because the economy is running circles to overcome the effects of depreciation.
There can be a time lag between movements of GDP and $\vec{B}_{K}$, too.
At the beginning of an economic downturn (defined as prolonged GDP reduction),
capital stock and associated embodied energy ($\vec{B}_{K}$) will remain approximately constant:
GDP moves but $\vec{B}_{K}$ doesn't.
But as the GDP decline continues, 
maintenance flows for capital stock will be reduced
and depreciation will overtake maintenance leading to a decline in $\vec{B}_{K}$. 
If economic decline is associated with significant external 
infrastructure investment (such as occurred in post-colonial Africa),
GDP may decrease 
while the stock of energy embodied in the economy ($\vec{B}_{K}$) increases 
due to foreign aid focused on infrastructure enhancement.
**** We may want a better example here.
Does foreign aid inflows count toward GDP\@?
Did this actually occur for any country in post-colonial Africa? 
--MKH ****

A second possible measure of economic ``development'' is another \emph{stock}, 
wealth:

\begin{equation} \label{eq:Dev_Integral_Wealth}
	X_{j}(t) 
	= X_{j}(0) 
	+ \int_{t=0}^{t=t} \frac{\mathrm{d}X_{j}}{\mathrm{d}t}\mathrm{d}t,
\end{equation}

\noindent{}where $j=1$ for societal wealth 
and $j \in [2,n]$ for corporate wealth, both measured in dollars.
In fact, capital (as represented by energy embodied in infrastructure, $\vec{B}_{K}$) and 
financial resources (as represeted by $X_{2 \ldots n}$) 
are complimentary factors of production for economic processes. 
But, we can go further than linking capital with financial resources.
If capital~($\vec{B}_{K}$) is to be useful, we need financial resources
or currency~($\dot{X}$) to 
\begin{itemize}
	\item{purchase direct energy~($\dot{E}$) to power the capital,
	\item{purchase resources~($\dot{R}$) to feed the capital, and}
	\item{pay workers~(represented 
	by societal energy input to the economy,~$\vec{T}_{1}$) 
	to operate the capital.}
} 
\end{itemize}

Thus, economic growth could be considered a ``fully coupled'' problem:
understanding it requires breadth of knowledge and appreciation for 
interactions among many important factors.
Each of the factors discussed above 
($\dot{X}$, $\vec{B}_{K}$, $\dot{E}$, $\dot{R}$, and $\vec{T}_{1}$)
is necessary, but not sufficient, for economic growth.

Our framework serves highlight to several issues in economic ``growth.'' 
Should it be measured by a stock or a flow? 
Which measure is most appropriate? 
What roles do currency, capital stock, energy, resources, and labor
play in economic processes?
These are overlapping and complementary areas of inquiry.
We encourage further research in all of these areas.


%%%%%%%%%% Implications %%%%%%%%%%
\section{Implications for recycling, reuse, and dematerialization}
\label{sec:recycling}
%%%%%%%%%%

%**** Mik: take a shot a rewriting. **** MCD---shot taken! ****

Dematerialization is the idea that economic activity can be unlinked 
from material or energy demands.\cite{FischerKowalski:2011uo} 
One method for dematerializing an economy 
is reuse and recycling of materials from both
short-lived goods
$\left(\vec{B}_{waste}\right)$
and depreciation of 
capital 
stock~$\left(\hat{\boldsymbol{\gamma}}_{K}\vec{B}_{K}\right)$
that would otherwise have
been expended to the biosphere.\footnote{Another 
method  of dematerialization is 
substitution away from production of material goods 
and toward information and services
in an economy.}
%**** MCD - I removed the following sentence as I don't agree with it. ****
% The impact of recycling can be seen in the I-O formulation 
% only when depreciation and accumulation terms are included. 

In Chapter~\ref{chap:intensity},
we defined the rate of accumulation 
of embodied energy within the economy
$\left(\frac{\mathrm{d}\vec{B}_{K}}{\mathrm{d}t}\right)$
by the following equation:

\begin{equation}
	\frac{\mathrm{d}\vec{B}_{K}}{\mathrm{d}t} 
	= \vec{E}_{0}
	+ \vec{T}_{1}
	+ \hat{\vec{X}} (\vec{A}^{\mathrm{T}} - \vec{I})\boldsymbol{\varepsilon} 
	- \vec{B}_{waste}
	- \hat{\boldsymbol{\gamma}}_{K}\vec{B}_{K}\tag{\ref{eq:matrix_leontief}}
\end{equation}

One effect of recycling is to reduce the magnitude 
of the waste 
$\left(\vec{B}_{waste}\right)$
and depreciation 
$\left(\hat{\boldsymbol{\gamma}}_{K}\right)$ 
terms,
since for a given level of capital 
stock~$\left(\vec{B}_{K}\right)$
the amount of depreciation to the 
biosphere $\left(\hat{\boldsymbol{\gamma}}_{K}\vec{B}_{K}\right)$
will be lower due to the fraction that
is now recycled.
As can be seen by looking at 
Equation~\ref{eq:matrix_leontief},
this indicates that 
recycling of material in an economy, 
by reducing both $\vec{B}_{waste}$ 
and~$\hat{\boldsymbol{\gamma}}_{K}$, 
puts \emph{upward} pressure on the accumulation of 
energy embodied in capital stock
$\left(\frac{\mathrm{d}\vec{B}_{K}}{\mathrm{d}t}\right)$.

Recycling has a mixed effect on energy demand ($\vec{E}_{0}$). 
Because recycled materials can displace newly-produced material 
in the economy and society, 
recycling will tend to reduce energy demand ($\vec{E}_{0}$). 
However, recycling processes require energy to operate, 
thereby putting upward pressure on energy demand ($\vec{E}_{0}$). 
If the energetic cost of recycling is lower than that of obtaining materials
from virgin resources, as is the case for many metals, 
e.g.\ aluminum~\cite{Chapman1975}, 
the result is a net reduction of energy demand 
from the biosphere ($\vec{E}_{0}$). 
Berry and Fels found that recycling of the material in automobiles
would result in 12,640 kW-hr of energy reduction per vehicle.\cite[p. 15]{Berry:1973vo}
Therefore recycling will also put \emph{downward} pressure on 
the growth of embodied energy in the economy
$\left(\frac{\mathrm{d}\vec{B}_{K}}{\mathrm{d}t}\right)$. 

If recycling produces a net reduction in energy demand ($\vec{E}_{0}$), 
% i.e.\ if the effect of displaced production dominates over the effect 
% of energy consumed in recycling processes, 
the upward pressure on growth 
$\left(\frac{\mathrm{d}\vec{B}_{K}}{\mathrm{d}t}\right)$ 
from decrease in depreciation 
$\left(\hat{\boldsymbol{\gamma}}_{K}\right)$ 
and the downward pressure on growth 
from net reduction in energy demand 
$\left(\vec{E}_{0}\right)$ 
can offset each other.
Under those conditions, 
the accumulation rate of energy embodied in capital stock
$\left(\frac{\mathrm{d}\vec{B}_{K}}{\mathrm{d}t}\right)$ 
will remain near zero 
and total embodied energy 
$(\vec{B}_{K})$ will remain constant. 
In that scenario, 
dematerialization can occur: 
reduced material and energy input ($\vec{E}_{0}$) 
can be accompanied by 
no change in
the growth of the economy
$\left(\frac{\mathrm{d}\vec{B}_{K}}{\mathrm{d}t}\right)$.
However,
as will be discussed in Section~\ref{sec:material_quality},
recycled materials can never entirely replace
the need for new materials.

%%%%%%%%%% Implications %%%%%%%%%%
\section{Comparison to a steady-state economy}
\label{sec:SSE}
%%%%%%%%%%

As discussed in Chapter~\ref{chap:intro},
the human economy is a subset of the biosphere which is a
finite, non-growing system,
\emph{open} to flows of solar energy but \emph{closed}
to material transfers.
Because the biosphere is finite in size,
the human economy cannot physically grow indefinitely.
The concept of a non-growing or ``steady-state'' economy
has existed for centuries.
There are a number of different
conditions that may characterize a system as steady-state.
In thermodynamics,
the condition of a zero rate of accumulation
of some stock, $x$
$\left(\frac{\mathrm{d}x}{\mathrm{d}t} = 0\right)$
is normally used to define steady state conditions.\footnote{Or,
more generally,
a change in any property over time
$\left(\frac{\partial p}{\partial t}\right)$.}
Other conditions that might define a steady-state
economy are:
a constant rate of material through-put,
constant GDP, or constant population.
Our framework,
as outlined here,
can address the first three steady-state conditions.
The fourth condition, constant population, could be
accommodated with some adaptation of the 
framework. 
The issue of human population as part of society's
capital stock is addressed in Section~\ref{sec:people_as_stock}.


%+++++++++++++++++++++++++++++++++++++++++++
%+++++++++ Constant capital stock ++++++++++
%+++++++++++++++++++++++++++++++++++++++++++
\subsection{Constant level of capital stock}

In Chapter~\ref{chap:materials}, 
we introduced Equation~\ref{eq:C_CV_all_b}:

\begin{align}\tag{\ref{eq:C_CV_all_b}}
	- \frac{\mathrm{d}R_{0}}{\mathrm{d}t}										&
	=\sum_{j}\frac{\mathrm{d}K_{j}}{\mathrm{d}t}
	+ \sum_{i,j}\dot{S}_{ij}
	+ \sum_{j}\gamma_{K_{j}}K_{j}.
\end{align}

\noindent{}which indicates that 
the depletion of natural resources in the 
biosphere~$\left(- \frac{\mathrm{d}R_{0}}{\mathrm{d}t}\right)$
by the economy
is used for the purposes of:

\begin{itemize}
	\item{increasing man-made capital stocks
	within the economy~$\left(\frac{\mathrm{d}K_{j}}{\mathrm{d}t}\right)$,}
	\item{providing short-lived goods exchanged within the
	economy~$\left(\dot{S}_{ij}\right)$, and}
	\item{overcoming depreciation of man-made
	capital stocks~$\left(\sum_{j}\gamma_{K_{j}}K_{j}\right)$.}
\end{itemize}

Assuming,
first,
that a steady-state economy exists when the level
of capital stock remains constant
$\left(\sum_{j}\frac{\mathrm{d}K_{j}}{\mathrm{d}t}=0\right)$,\footnote{Note
that the steady-state condition does not preclude expansion
of some sectors of the economy, provided that there is 
equal contraction elsewhere.}
we can see that Equation~\ref{eq:C_CV_all_b}
reduces to:

\begin{align}\label{eq:C_CV_all_c}
	- \frac{\mathrm{d}R_{0}}{\mathrm{d}t}										&
	= \sum_{i,j}\dot{S}_{ij}
	+ \sum_{j}\gamma_{K_{j}}K_{j}.
\end{align}

A number of interesting concepts may be understood in
relation to Equation~\ref{eq:C_CV_all_c}. 
Firstly,
if our steady-state economy is to be supported
sustainably,
then withdrawal of natural resources from the biosphere
$\left(\frac{\mathrm{d}R_{0}}{\mathrm{d}t}\right)$
had better be at some rate lower than the biosphere
can replenish the natural capital stocks. 
In reality, 
$\frac{\mathrm{d}R_{0}}{\mathrm{d}t}$
is really the sum of many different resources
(flora and fauna, water) each of which will have
its own natural rate of regeneration.
As such, 
the sustainability criterion  is a vector of values,
one for each natural resource.

Secondly,
the steady state condition 
$\left(\sum_{j}\frac{\mathrm{d}K_{j}}{\mathrm{d}t}=0\right)$
says nothing about the transfer rates of short-lived goods 
within in the economy $\left(\sum_{i,j}\dot{S}_{ij}\right)$
or the depreciation of capital stock back to the biopshere
$\left(\sum_{j}\gamma_{K_{j}}K_{j}\right)$.
Equation~\ref{eq:C_CV_all_c} indicates that 
the higher the rates of these flows,
the greater the rate of depletion of
natural resources, and the more difficult it will be to
meet the sustainability condition 
(that the withdrawal rate of natural resources from the biosphere
is lower than the biosphere
replenishment rate).
Within industrial society,
the flow of short-lived goods 
(packaging,
paper products,
disposable tableware,
cutlery,
and napkins)
is large,
and, presumably, attaining a sustainable steady-state economy
will be difficult.
This definition of steady state 
$\left(\sum_{j}\frac{\mathrm{d}K_{j}}{\mathrm{d}t}=0\right)$
does not necessarily coincide with sustainability.

As discussed in Chapter~\ref{chap:materials},
the rate of depreciation ($\gamma_{K}$) is inversely
proportional to the average lifetime of capital stock---as
the average lifetime of capital stock decreases,
the rate of depreciation of capital stock increases which
increases the draw on natural resources (by Equation~\ref{eq:C_CV_all_c}).
It is likely that the average lifetime of capital stock has
decreased over the last century, 
due to a decrease in durability of capital stock
(the average table built today is not as durable as the average table
built in the early twentieth century)
and also due to increasing proportions of consumer electronics
with short lifetimes (cell phones, laptops, tablets).
Decreasing lifetime causes higher rates of flow for
replacement materials.
In the absence of extreme recycling of materials,
these large replacement flows place large demands on
natural resources.

Thirdly,
the maintenance flows necessary to overcome
depreciation
$\left(\sum_{j}\gamma_{K_{j}}K_{j}\right)$
are proportional to the magnitude of the capital
stock ($K_{j}$).
As such,
a larger stock of capital requires
greater draw on natural resources and is thus harder
to maintain within any sustainability constraint.
These points emphasize that constant capital stock
(or analogously constant population)
is not a sufficient condition for environmental sustainability.


%++++++++++++++++++++++++++++++++++++++++
%+++++++++ Constant throughput ++++++++++
%++++++++++++++++++++++++++++++++++++++++
\subsection{Constant material throughput}

Herman Daly has placed great emphasis on a steady-state
economy as having a constant rate of material throughput 
\cite{Daly1977, Daly1997}
which, as discussed above, should be below biophysical limits
if sustainability is to be achieved.
This is often referred to as the ``scale'' issue---how 
large is the (currently growing) human economy in relation
to the finite, non-growing biosphere of which it is a sub-system?
Growth of the human economy must either displace other natural
ecosystems (replacing old growth forest for crops)
or deplete natural capital stocks, be they 
renewable (fisheries) or 
non-renewable (fossil fuels).
As shown in Figure~\ref{fig:C_materials},
material throughput is composed of two distinct processes:
exchange of material \emph{from} the biosphere 
\emph{into} the economy (extraction)
and exchange of material \emph{from} the economy
\emph{into} biosphere (waste and depreciation).
We may characterizing constant material throughput
as either constant rate of extraction, constant rate of
waste disposal, or both.
In the language of our framework, we could write:

\begin{align}\label{eq:const_throughput}
	\frac{\mathrm{d}}{\mathrm{d}t}\left(\dot{R}_{0}\right)		&
	= 0																							\\
	\frac{\mathrm{d}}{\mathrm{d}t}\left(\dot{S}_{0}\right)		&
	= 0																						\\
	\sum\limits_{i}
			\left[
				\frac{\mathrm{d}}{\mathrm{d}t}\left(\dot{R}_{i0}\right)
				+ \frac{\mathrm{d}}{\mathrm{d}t}\left(\dot{S}_{i0}\right)
				+ \frac{\mathrm{d}}{\mathrm{d}t}\left(\dot{K}_{i0}\right)
			\right]																			&
	= 0.
\end{align}

The above equations say nothing about the level 
of man-made capital stock ($K$)
or the flow rate of short-lived goods ($\dot{S}$).
Thus, within the constant throughput constraint,
increasingly effective use of materials could
theoretically allow increasing accumulation
of man-made capital ($K$) 
and increasing flow of short-lived goods ($\dot{S}$)
as society learns to use resources better.
Eventually,
physical limits would entail that capital
stock could no longer be increased.
Presumably,
society would desire that the throughput of
materials would be within levels that could
be sustained by the biosphere,
both at the input side---natural 
resources extracted at rates lower
than natural regeneration rates---and 
at the output side---wastes emitted 
at rates below which
the biosphere can assimilate.


%++++++++++++++++++++++++++++++++++++++++
%+++++++++ Constant GDP ++++++++++
%++++++++++++++++++++++++++++++++++++++++
\subsection{Constant GDP}

The issue of GDP and economic ``development'' 
has already been discussed in 
Section~\ref{sec:implications_for_development}. 
However,
the case of zero growth in GDP was not.
Within our framework,
a condition of constant GDP would be
characterized by the following equation:

\begin{equation}\label{eq:const_GDP}
	\frac{\mathrm{d}}{\mathrm{d}t}\left(GDP\right)
	=	\sum\limits_{j}\frac{\mathrm{d}}{\mathrm{d}t}\left(\dot{X}_{j}\right)
	= 0
\end{equation}

Since,
under the subjective theory of value,
no value is attributed to the flow of materials
to or from the biosphere,
it is unclear what the impact constant GDP would be
on capital stock ($K$)
or material throughput
(extraction and waste disposal).
If we constrained $\dot{R}_{0}$ and
$\dot{S}_{0}$, 
it is likely that economic growth would 
decrease or even become zero or negative
$\left(\frac{\mathrm{d}}{\mathrm{d}t}\left(GDP\right) \leq 0\right)$.
It is conceivable that constraining
economic growth may act to constrain
material throughput,
though this is certainly not assured.
Many authors argue that increasing GDP
no longer guarantees increasing welfare~\cite{G-R1975a, Wackernagel1996,
Cobb1999, Daly2006, Costanza2014} 
for two main reasons:
\begin{itemize}
	\item firstly, that the costs of growth in GDP, 
	some of which are added as benefits into GDP 
	but also externalities (especially environmental) 
	which are not counted at all, outweigh any benefit 
	that comes from increasing GDP;\@ and
	\item secondly, that increasing GDP serves to
	increase relative income inequality,
	despite increasing absolute income,
	which decreases welfare for both rich and poor 
	alike.~\cite{Daly2006}
\end{itemize}

Furthermore,
the argument for increasing GDP as a means
to alleviate poverty is undermined by this second point.
The world's poor are poor because they cannot afford to
buy food and other goods due to their \emph{relative} poverty,
not because there is an \emph{absolute} lack of these goods.
The rising tide raises all boats,
but the (growing) luxury cruisers soon capsize 
the (shrinking) dinghies.



**** MCD---Becky, can you speak to this? 
This last paragraph is way outside my area of expertise and
is a little bit ``soap boxy''. ****

**** Discuss maintenance vs.\ accumulation flows? 
Is this a better place to discuss this issue? Can refer to
Section 6.5 and Odum B/P vs. P/B. ****

%****** Mik: Finish this section. 
%In terms of what a SSE would look like in the I-O framework, 
%at first blush, I would think that dB/dt = 0 is one aspect.  
%Also, with no growth, inflow rates = depreciation rates.  
%The larger that B is for any society, the larger E must be (to overcome depreciation).  
%To minimize E, hyper-recycling is probably useful.  
%Those are at least a place to start. ******
%
%****** In our discussion, 
%we also addressed the attempts at SSE from point of view of society. 
%In order to achieve this goal \emph{without} recycling, 
%the goods and services sector should have to increase extraction to offset decreasing ore grade, 
%the energy sector should have to increase extraction of energy 
%to allow increasing extraction (unless efficiency could make up the gap: unlikely) 
%in which case the SSE would be violated from these two and from the POV of the earth.
%******


\bibliographystyle{unsrt}
\bibliography{../../EROI_review_v2}


% Always give a unique label
% and use \ref{<label>} for cross-references
% and \cite{<label>} for bibliographic references
% use \sectionmark{}
% to alter or adjust the section heading in the running head
%% Instead of simply listing headings of different levels we recommend to let every heading be followed by at least a short passage of text. Furtheron please use the \LaTeX\ automatism for all your cross-references and citations.

%% Please note that the first line of text that follows a heading is not indented, whereas the first lines of all subsequent paragraphs are.

%% Use the standard \verb|equation| environment to typeset your equations, e.g.
%
%% \begin{equation}
%% a \times b = c\;,
%% \end{equation}
%
%% however, for multiline equations we recommend to use the \verb|eqnarray|
%% environment\footnote{In physics texts please activate the class option \texttt{vecphys} to depict your vectors in \textbf{\itshape boldface-italic} type - as is customary for a wide range of physical subjects.}.
%% \begin{eqnarray}
%% a \times b = c \nonumber\\
%% \vec{a} \cdot \vec{b}=\vec{c}
%% \label{eq:01}
%% \end{eqnarray}

%% \subsection{Subsection Heading}
%% \label{subsec:2}
%% Instead of simply listing headings of different levels we recommend to let every heading be followed by at least a short passage of text. Furtheron please use the \LaTeX\ automatism for all your cross-references\index{cross-references} and citations\index{citations} as has already been described in Sect.~\ref{sec:2}.

%% \begin{quotation}
%% Please do not use quotation marks when quoting texts! Simply use the \verb|quotation| environment -- it will automatically render Springer's preferred layout.
%% \end{quotation}


%% \subsubsection{Subsubsection Heading}
%% Instead of simply listing headings of different levels we recommend to let every heading be followed by at least a short passage of text. Furtheron please use the \LaTeX\ automatism for all your cross-references and citations as has already been described in Sect.~\ref{subsec:2}, see also Fig.~\ref{fig:1}\footnote{If you copy text passages, figures, or tables from other works, you must obtain \textit{permission} from the copyright holder (usually the original publisher). Please enclose the signed permission with the manucript. The sources\index{permission to print} must be acknowledged either in the captions, as footnotes or in a separate section of the book.}

%% Please note that the first line of text that follows a heading is not indented, whereas the first lines of all subsequent paragraphs are.

% For figures use
%
%% \begin{figure}[b]
%% \sidecaption
% Use the relevant command for your figure-insertion program
% to insert the figure file.
% For example, with the option graphics use
%% \includegraphics[scale=.65]{figure}
%
% If not, use
%\picplace{5cm}{2cm} % Give the correct figure height and width in cm
%
%% \caption{If the width of the figure is less than 7.8 cm use the \texttt{sidecapion} command to flush the caption on the left side of the page. If the figure is positioned at the top of the page, align the sidecaption with the top of the figure -- to achieve this you simply need to use the optional argument \texttt{[t]} with the \texttt{sidecaption} command}
%% \label{fig:1}       % Give a unique label
%% \end{figure}


%% \paragraph{Paragraph Heading} %
%% Instead of simply listing headings of different levels we recommend to let every heading be followed by at least a short passage of text. Furtheron please use the \LaTeX\ automatism for all your cross-references and citations as has already been described in Sect.~\ref{sec:2}.

%% Please note that the first line of text that follows a heading is not indented, whereas the first lines of all subsequent paragraphs are.

%% For typesetting numbered lists we recommend to use the \verb|enumerate| environment -- it will automatically render Springer's preferred layout.

%% \begin{enumerate}
%% \item{Livelihood and survival mobility are oftentimes coutcomes of uneven socioeconomic development.}
%% \begin{enumerate}
%% \item{Livelihood and survival mobility are oftentimes coutcomes of uneven socioeconomic development.}
%% \item{Livelihood and survival mobility are oftentimes coutcomes of uneven socioeconomic development.}
%% \end{enumerate}
%% \item{Livelihood and survival mobility are oftentimes coutcomes of uneven socioeconomic development.}
%% \end{enumerate}


%% \subparagraph{Subparagraph Heading} In order to avoid simply listing headings of different levels we recommend to let every heading be followed by at least a short passage of text. Use the \LaTeX\ automatism for all your cross-references and citations as has already been described in Sect.~\ref{sec:2}, see also Fig.~\ref{fig:2}.

%% Please note that the first line of text that follows a heading is not indented, whereas the first lines of all subsequent paragraphs are.

%% For unnumbered list we recommend to use the \verb|itemize| environment -- it will automatically render Springer's preferred layout.

%% \begin{itemize}
%% \item{Livelihood and survival mobility are oftentimes coutcomes of uneven socioeconomic development, cf. Table~\ref{tab:1}.}
%% \begin{itemize}
%% \item{Livelihood and survival mobility are oftentimes coutcomes of uneven socioeconomic development.}
%% \item{Livelihood and survival mobility are oftentimes coutcomes of uneven socioeconomic development.}
%% \end{itemize}
%% \item{Livelihood and survival mobility are oftentimes coutcomes of uneven socioeconomic development.}
%% \end{itemize}

%% \begin{figure}[t]
%% \sidecaption[t]
% Use the relevant command for your figure-insertion program
% to insert the figure file.
% For example, with the option graphics use
%% \includegraphics[scale=.65]{figure}
%
% If not, use
%\picplace{5cm}{2cm} % Give the correct figure height and width in cm
%
%% \caption{Please write your figure caption here}
%% \label{fig:2}       % Give a unique label
%% \end{figure}

%% \runinhead{Run-in Heading Boldface Version} Use the \LaTeX\ automatism for all your cross-references and citations as has already been described in Sect.~\ref{sec:2}.

%% \subruninhead{Run-in Heading Italic Version} Use the \LaTeX\ automatism for all your cross-refer\-ences and citations as has already been described in Sect.~\ref{sec:2}\index{paragraph}.
% Use the \index{} command to code your index words
%
% For tables use
%
%% \begin{table}
%% \caption{Please write your table caption here}
%% \label{tab:1}       % Give a unique label
%
% For LaTeX tables use
%
%% \begin{tabular}{p{2cm}p{2.4cm}p{2cm}p{4.9cm}}
%% \hline\noalign{\smallskip}
%% Classes & Subclass & Length & Action Mechanism  \\
%% \noalign{\smallskip}\svhline\noalign{\smallskip}
%% Translation & mRNA$^a$  & 22 (19--25) & Translation repression, mRNA cleavage\\
%% Translation & mRNA cleavage & 21 & mRNA cleavage\\
%% Translation & mRNA  & 21--22 & mRNA cleavage\\
%%Translation & mRNA  & 24--26 & Histone and DNA Modification\\
%%\noalign{\smallskip}\hline\noalign{\smallskip}
%%\end{tabular}
%%$^a$ Table foot note (with superscript)
%%\end{table}
%
%% \section{Section Heading}
%%\label{sec:3}
% Always give a unique label
% and use \ref{<label>} for cross-references
% and \cite{<label>} for bibliographic references
% use \sectionmark{}
% to alter or adjust the section heading in the running head
%% Instead of simply listing headings of different levels we recommend to let every heading be followed by at least a short passage of text. Furtheron please use the \LaTeX\ automatism for all your cross-references and citations as has already been described in Sect.~\ref{sec:2}.

%% Please note that the first line of text that follows a heading is not indented, whereas the first lines of all subsequent paragraphs are.

%%If you want to list definitions or the like we recommend to use the Springer-enhanced \verb|description| environment -- it will automatically render Springer's preferred layout.

%%\begin{description}[Type 1]
%%\item[Type 1]{That addresses central themes pertainng to migration, health, and disease. In Sect.~\ref{sec:1}, Wilson discusses the role of human migration in infectious disease distributions and patterns.}
%%\item[Type 2]{That addresses central themes pertainng to migration, health, and disease. In Sect.~\ref{subsec:2}, Wilson discusses the role of human migration in infectious disease distributions and patterns.}
%%\end{description}

%%\subsection{Subsection Heading} %
%% In order to avoid simply listing headings of different levels we recommend to let every heading be followed by at least a short passage of text. Use the \LaTeX\ automatism for all your cross-references and citations citations as has already been described in Sect.~\ref{sec:2}.

%% Please note that the first line of text that follows a heading is not indented, whereas the first lines of all subsequent paragraphs are.

%% \begin{svgraybox}
%% If you want to emphasize complete paragraphs of texts we recommend to use the newly defined Springer class option \verb|graybox| and the newly defined environment \verb|svgraybox|. This will produce a 15 percent screened box 'behind' your text.

%% If you want to emphasize complete paragraphs of texts we recommend to use the newly defined Springer class option and environment \verb|svgraybox|. This will produce a 15 percent screened box 'behind' your text.
%% \end{svgraybox}


%% \subsubsection{Subsubsection Heading}
%%Instead of simply listing headings of different levels we recommend to let every heading be followed by at least a short passage of text. Furtheron please use the \LaTeX\ automatism for all your cross-references and citations as has already been described in Sect.~\ref{sec:2}.

%% Please note that the first line of text that follows a heading is not indented, whereas the first lines of all subsequent paragraphs are.

%% \begin{theorem}
%% Theorem text goes here.
%% \end{theorem}
%
% or
%
%% \begin{definition}
%% Definition text goes here.
%% \end{definition}

%% \begin{proof}
%\smartqed
%% Proof text goes here.
%% \qed
%% \end{proof}

%%\paragraph{Paragraph Heading} %
%% Instead of simply listing headings of different levels we recommend to let every heading be followed by at least a short passage of text. Furtheron please use the \LaTeX\ automatism for all your cross-references and citations as has already been described in Sect.~\ref{sec:2}.

%% Note that the first line of text that follows a heading is not indented, whereas the first lines of all subsequent paragraphs are.
%
% For built-in environments use
%
%%\begin{theorem}
%%Theorem text goes here.
%%\end{theorem}
%
%%\begin{definition}
%%Definition text goes here.
%%\end{definition}
%
%%\begin{proof}
%%\smartqed
%% Proof text goes here.
%%\qed
%%\end{proof}
%
%% \begin{acknowledgement}
%% If you want to include acknowledgments of assistance and the like at the end of an individual chapter please use the \verb|acknowledgement| environment -- it will automatically render Springer's preferred layout.
%% \end{acknowledgement}
%
%% \section*{Appendix}
%% \addcontentsline{toc}{section}{Appendix}
%
%% When placed at the end of a chapter or contribution (as opposed to at the end of the book), the numbering of tables, figures, and equations in the appendix section continues on from that in the main text. Hence please \textit{do not} use the \verb|appendix| command when writing an appendix at the end of your chapter or contribution. If there is only one the appendix is designated ``Appendix'', or ``Appendix 1'', or ``Appendix 2'', etc. if there is more than one.

%% \begin{equation}
%% a \times b = c
%% \end{equation}
% Problems or Exercises should be sorted chapterwise
%% \section*{Problems}
%% \addcontentsline{toc}{section}{Problems}
%
% Use the following environment.
% Don't forget to label each problem;
% the label is needed for the solutions' environment
%% \begin{prob}
%% \label{prob1}
%% A given problem or Excercise is described here. The
%% problem is described here. The problem is described here.
%% \end{prob}

%% \begin{prob}
%% \label{prob2}
%% \textbf{Problem Heading}\\
%% (a) The first part of the problem is described here.\\
%% (b) The second part of the problem is described here.
%% \end{prob}


