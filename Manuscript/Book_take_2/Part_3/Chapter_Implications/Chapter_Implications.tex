%!TEX root = ../../Heun_Dale_Haney_A_dynamic_approach_to_input_output_modeling.tex
%%%%%%%%%%%%%%%%%%%%% chapter.tex %%%%%%%%%%%%%%%%%%%%%%%%%%%%%%%%%
%
% sample chapter
%
% Use this file as a template for your own input.
%
%%%%%%%%%%%%%%%%%%%%%%%% Springer-Verlag %%%%%%%%%%%%%%%%%%%%%%%%%%
%\motto{Use the template \emph{chapter.tex} to style the various elements of your chapter content.}

%%%%%%%%%%%%%%%%%%%%%%%%%%%%%%%%%%
%%%%%%%%%% Implications %%%%%%%%%%
%%%%%%%%%%%%%%%%%%%%%%%%%%%%%%%%%%
\chapter{Implications}
% Always give a unique label
\label{chap:implications}
% use \chaptermark{} to alter or adjust the chapter heading in the running head
\chaptermark{Implications}
%%%%%%%%%%%%%%%%%%%%%%%%%%%%%%%%%%
%%%%%%%%%%%%%%%%%%%%%%%%%%%%%%%%%%
%%%%%%%%%%%%%%%%%%%%%%%%%%%%%%%%%%

\abstract*{[NEED TO ADD ABSTRACT HERE]}

%% \abstract{Each chapter should be preceded by an abstract (10--15 lines long) that summarizes the content. The abstract will appear \textit{online} at \url{www.SpringerLink.com} and be available with unrestricted access. This allows unregistered users to read the abstract as a teaser for the complete chapter. As a general rule the abstracts will not appear in the printed version of your book unless it is the style of your particular book or that of the series to which your book belongs.\newline\indent
%% Please use the 'starred' version of the new Springer \texttt{abstract} command for typesetting the text of the online abstracts (cf. source file of this chapter template \texttt{abstract}) and include them with the source files of your manuscript. Use the plain \texttt{abstract} command if the abstract is also to appear in the printed version of the book.}

%% Use the template \emph{chapter.tex} together with the Springer document class SVMono (monograph-type books) or SVMult (edited books) to style the various elements of your chapter content in the Springer layout.


Several implications can be drawn from the detailed development 
of a framework for materials and energy accounting 
(in Chapters~\ref{chap:materials}--\ref{chap:intensity})
that includes energy input from society to the economy,
embodied energy accumulation, waste material flows, and depreciation.


%%%%%%%%%% Implications for the I-O method %%%%%%%%%%
\section{Implications for the I-O method}
\label{sec:Implications_for_IO}
%%%%%%%%%%

The first set of implications are for the I-O method itself;
specifically, for the process of estimating
the energy intensity of economic output ($\bm{\varepsilon}$).


%%%%%%%%%% I-O implications: estimating epsilon %%%%%%%%%%
\subsection{Estimating $\bm{\varepsilon}$}
\label{sec:estimating_epsilon-implications_chapter}
%%%%%%%%%%

Extension of the Leontief\index{Leontief} 
Input-Output method\index{input-output method}
for energy analysis has allowed energy analysts to estimate 
the energy intensity\index{energy intensity}
of economic products ($\bm{\varepsilon}$). 
As discussed in Section~\ref{sec:Value_Methodology},
we do not take this important result as a license
to declare an intrinsic ``energy theory of value.''
\index{theory of value!energy}
Rather, we belive that energy intensity is an 
important and useful metric that can assess 
the energy performance of economies,
even within the prevailing subjective theory of value
\index{theory of value!subjective}
that underlies modern economics. 
Thus, it is important to consider the assumptions behind
the literature's presentation of the I-O method 
for estimating the energy intensity economic output.

In this manuscript, Equation~\ref{eq:epsilon_leontief_depreciation_simplification} 
provides a means of estimating the energy intensity of economic sectors:

\begin{equation}
	\bm{\varepsilon} 
	= {(\vec{I} - \vec{A}^{\mathrm{T}})}^{-1}\hat{\vec{X}}^{-1}
		\left[\vec{E}_{0} 
				+ \vec{T}_{1} 
				- \left. \frac{\mathrm{d}\vec{B}_{K}}{\mathrm{d}t} \right|_{\mathrm{other}}
				- \vec{B}_{waste}
		\right]. \tag{\ref{eq:epsilon_leontief_depreciation_simplification}}
\end{equation}

\noindent{}The I-O literature~\cite{Bullard1975,Casler1984}, 
on the other hand, 
writes Equation~\ref{eq:epsilon_leontief_depreciation_simplification} 
as\footnote{For a discussion of other, more-subtle, differences
between the energy intensity equations in the literature
and this manuscript, see Appendix~\ref{chap:Casler}.}

\begin{equation} \label{eq:epsilon_leontief_with_A_literature}
	\bm{\varepsilon} 
	= {(\vec{I} - \vec{A}^{\mathrm{T}})}^{-1}
	\hat{\vec{X}}^{-1}
	\vec{E}_{0}.
\end{equation}

The differences between Equations~\ref{eq:epsilon_leontief_depreciation_simplification}
and~\ref{eq:epsilon_leontief_with_A_literature} are obvious. 
The literature neglects
\begin{itemize}
	\item{energy input from society ($\vec{T}_{1}$),}
	\item{accumulation of embodied energy in the economy,
			exclusive of physical depreciation 
			$\left( \left. \frac{\mathrm{d}\vec{B}}{\mathrm{d}t} \right|_{\mathrm{other}} \right)$,
			and}
	\item{the embodied energy of scrap resource	and short-lived material flows 
			($\dot{\vec{B}}_{waste}$)}
\end{itemize}

\noindent{}when estimating the energy intensity  
of economic sectors ($\bm{\varepsilon}$).\footnote{To be precise, 
the literature effectively assumes
$
	\vec{T}_{1}
	- \left. \frac{\mathrm{d}\vec{B}_{K}}{\mathrm{d}t} \right|_{\mathrm{other}}
	- \dot{\vec{B}}_{waste}
	= \vec{0}
$}
In other words, energy analysts using the input-output method
have, to date, and perhaps unwittingly, assumed 
a developed-world ($\vec{T}_{1} \ll \vec{E}_{0}$), 
steady state economy\index{steady-state economy}
$\left( \left. \frac{\mathrm{d}\mathrm{\vec{B}}_{K}}{\mathrm{d}t} \right|_{\mathrm{other}} 
= \vec{0} \right)$ with no waste ($\vec{B}_{waste} = \vec{0}$).

The following subsections discuss each of these assumptions in turn.


%+++++++++ Negligible energy input from society ++++++++++
\subsubsection{Negligible energy input from society $\left( \vec{T}_{1} = \vec{0} \right)$}
%+++++++++

Energy input from society to the economy ($\vec{T}_{1}$)
is ``muscle work'' supplied by working humans 
and draught animals.\cite{Ayres:2003ec,Ayres:2010ug,Warr:2012cg} 

**** Mik asks: One question I have is:
Does this value include all of the upstream energy needed to support labor, 
or simply the upstream energy required to produce the food that fuels the labor?
We should spell out exactly what is included via this framework. 
This is important because, in NEA, labor is often used 
as a HUGE lever to produce whatever result is desired.
****

For industrialized economies, muscle work 
is likely to provide only a small fraction
of the energy input from fossil fuels ($\vec{E}_{0}$),
so neglecting $\vec{T}_{1}$ causes negligible error when
estimating energy intensity ($\bm{\varepsilon}$) by
Equation~\ref{eq:epsilon_leontief_with_A_literature}.
However, for some agrarian\index{economy!agrarian} 
and developing economies\index{economy!developing}, 
where $\vec{T}_{1}$ and $\vec{E}_{0}$ 
could be on the same order of magnitude,
neglecting $\vec{T}_{1}$ could cause errors
in estimates of $\bm{\varepsilon}$.
To the extent that $\vec{T}_{1}$ 
is significant relative to $\vec{E}_{0}$,
Equation~\ref{eq:epsilon_leontief_with_A_literature}
will underpredict the energy intensity of the economy.

Accurate estimation of the energy intensity of economic output ($\bm{\varepsilon}$)
requires independent knowledge of the rate at which society supplies
energy to the economy ($\vec{T}_{1}$). 
Ayres and Warr have estimated human and animal muscle work
input to the economy for a few developed countries.\cite{Ayres:2010ug}
We recommend that more of this work be done 
in the future for many more countries.


%+++++++++ Negligible accumulation of embodied energy ++++++++++
\subsubsection{Negligible accumulation of embodied energy
$\left( \left. \frac{\mathrm{d}\vec{B}_{K}}{\mathrm{d}t} \right|_{\mathrm{other}} = \vec{0} \right)$}
%+++++++++

As discussed in detail below (Section~\ref{sec:implications_for_development}),
the accumulation of embodied energy 
$\left( \left. \frac{\mathrm{d}\vec{B}_{K}}{\mathrm{d}t} \right|_{\mathrm{other}} \right)$
in society and the economy
can be considered a marker of economic ``development.''

**** 
Mik says: Daly talks about this in terms 
of Odum's B/P vs P/B ratios for growing vs.\ steady systems.
****

Equation~\ref{eq:epsilon_leontief_depreciation_simplification} 
shows that embodied energy accumulation within economic sectors 
exclusive of depreciation
$\left( \left. \frac{\mathrm{d}\vec{B}_{K}}{\mathrm{d}t} \right|_{\mathrm{other}} \right)$ 
decreases the energy intensity of products~($\bm{\varepsilon}$),
because incoming energy becomes embodied within capital stock rather than products.
Costanza points out that traditional
``input-output methodology does not include capital stocks explicitly, 
since a static equillibrium is assumed.''\cite{Costanza:1980ww}
Comparison with Equation~\ref{eq:epsilon_leontief_with_A_literature}
shows that the method in the literature will overestimate the energy intensity
of economic products ($\bm{\varepsilon}$) when 
$\left. \frac{\mathrm{d}\vec{B}_{K}}{\mathrm{d}t} \right|_{\mathrm{other}} > 0$.

Rapidly growing economies (as measured by GDP), such as China and India today,
are expected to have rather large positive values in the  
$\left. \frac{\mathrm{d}\vec{B}_{K}}{\mathrm{d}t} \right|_{\mathrm{other}}$ vector,
while slowly-growing, industrialized economies, 
such as the United States and the United Kingdom,
are expected to have rather smaller values in the 
$\left. \frac{\mathrm{d}\vec{B}_{K}}{\mathrm{d}t} \right|_{\mathrm{other}}$ vector.
Applying Equation~\ref{eq:epsilon_leontief_with_A_literature} 
(i.e., assuming a steady-state economy\index{economy!steady-state})
will tend to overestimate the energy intensity 
of economic products for rapidly growing economies such as China and India.
The approximate magnitude of mis-estimation remains unknown, because
estimates of the magnitude of
$\left. \frac{\mathrm{d}\vec{B}_{K}}{\mathrm{d}t} \right|_{\mathrm{other}}$
have yet to be made.

Equation~\ref{eq:epsilon_leontief_depreciation_simplification} shows that 
accurate estimation of the energy intensity of economic output ($\bm{\varepsilon}$)
requires \emph{independent} knowledge of the rate 
at which embodied energy accumulates in the capital stock of the economy 
$\left( \left. \frac{\mathrm{d}\vec{B}_{K}}{\mathrm{d}t} \right|_{\mathrm{other}} \right)$,
and all estimates of energy intensity ($\bm{\varepsilon}$) to date have assumed 
$\left. \frac{\mathrm{d}\vec{B}_{K}}{\mathrm{d}t} \right|_{\mathrm{other}} = 0$.\footnote{To 
our knowledge, there are no independent estimates 
of the accumulation rate of embodied energy in the economy
$\left( \left. \frac{\mathrm{d}\vec{B}_{K}}{\mathrm{d}t} \right|_{\mathrm{other}} \right)$.}
Furthermore, Equation~\ref{eq:epsilon_leontief_depreciation_simplification} 
cannot be used to estimate the accumulation rate of energy embodied in 
capital stock by solving for 
$\left( \left. \frac{\mathrm{d}\vec{B}_{K}}{\mathrm{d}t} \right|_{\mathrm{other}} \right)$,
because that approach would require independent knowledge
of the energy intensity of economic sectors ($\bm{\varepsilon}$).
Alternatively, Equation~\ref{eq:C_embodied_energy_accounting_123_with_depreciation} 
could be simplified using the 
methods of Section~\ref{sec:depreciation} to obtain

\begin{equation} \label{eq:C_embodied_energy_accounting_123_with_depreciation_simplified}
	\left. \frac{\mathrm{d}B_{K_{j}}}{\mathrm{d}t} \right|_{\mathrm{other}}
	= \sum\limits_{i=1}^n\dot{B}_{ij} 
	- \dot{B}_{j}
	- \dot{B}_{waste,j0}
	+ \dot{Q}_{j0},
\end{equation}

\noindent{}but this approach would require independent knowledge of the 
embodied energy of all inter-sectoral material transfers in the entire economy.

At present, this situation appears to be a catch-22. 
Clearly, further work focused on estimating the relative magnitudes of 
$\left. \frac{\mathrm{d}\vec{B}_{K}}{\mathrm{d}t} \right|_{\mathrm{other}}$,
$\vec{T}_{1}$, and $\vec{B}_{waste}$ 
will benefit energy analysts who utilize the I-O method.


%+++++++++ Negligible embodied energy in waste ++++++++++
\subsubsection{Negligible energy embodied in waste ($\vec{B}_{waste} = 0$)}
%+++++++++

If the energy embodied in waste resources ($\dot{R}$)
and short-lived materials ($\dot{S}$) are ignored,
Equation~\ref{eq:epsilon_leontief_depreciation_simplification}
shows that estimates of the energy intensity of economic products 
($\bm{\varepsilon}$) will be too high.
We are unaware of any estimates of the energy embodied in wasted
material in an economy.  
But, one might develop a metric for the resource material effiency of an 
economic sector ($\eta_{\dot{R}}$)\nomenclature[h]{$\eta_{\dot{R}}$}{resource efficiency [kg/kg]} 
such that

\begin{equation}
	\eta_{\dot{R},j}
	\equiv \frac{\dot{P}_{j}}{\sum\limits_{i=1}^{n} \dot{R}_{ij}}.
\end{equation}

\noindent{}With the above definition, 
the scrap rate for resources could be expressed as
$(1~-~\eta_{\dot{R}}) \sum\limits_{i=1}^{n} \dot{R}_{ij}$.

Furthermore, one could assume that the rate
of short-lived materials ($\dot{S}$) used by a sector could be given as a 
fraction of the resource ($\dot{R}$) use rate such that:

\begin{equation}
	\rho_{\dot{S},j}
	\equiv \frac{\dot{S}_{j0}}{\sum\limits_{i=1}^{n} \dot{R}_{ij}}
	= \frac{\sum\limits_{i=1}^{n} \dot{S}_{ij}}{\sum\limits_{i=1}^{n} \dot{R}_{ij}}.
\end{equation} \nomenclature[r]{$\rho_{\dot{S}}$}{ratio of short-lived material
flow to resource inputs [kg/kg]}

With the above definitions, the waste resource rate from an economic sector
can be given as

\begin{equation}
	\dot{R}_{j0} + \dot{S}_{j0}
	= (1 - \eta_{\dot{R}_{j}} + \rho_{\dot{S},j}) \sum\limits_{i=1}^{n} \dot{R}_{ij}.
\end{equation}

\noindent{}The embodied energy in the waste materials would need to be estimated
from the embodied energy of the incoming resource and short-lived material flows.

However, it could be argued that the embodied energy content 
of waste from the production process should be 
assigned to the products of a sector.\footnote{This approach 
is similar to assigning waste heat energy from an
economic sector to the embodied energy of its products.
See Section~\ref{subsec:A_first_law_embodied} and
Equation~\ref{eq:A_dB1/dt} for a discussion of this approach.}
If so, Equation~\ref{eq:epsilon_leontief_depreciation_simplification}
could be modified as

\begin{equation} \label{eq:epsilon_leontief_without_waste}
	\bm{\varepsilon} 
	= {(\vec{I} - \vec{A}^{\mathrm{T}})}^{-1}\hat{\vec{X}}^{-1}
		\left[\vec{E}_{0} 
				+ \vec{T}_{1} 
				- \left. \frac{\mathrm{d}\vec{B}_{K}}{\mathrm{d}t} \right|_{\mathrm{other}}
		\right].
\end{equation}

Further work focused on estimating values for 
$\eta_{\dot{S}}$ and $\xi_{\dot{S}}$
and the the energy embodied in waste flows 
will benefit energy analysts who utilize the I-O method.

A second set of implications for the I-O method 
comes from one of the ways that 
the energy intensity vector ($\bm{\varepsilon}$)
is often used: to estimate energy demand from the biosphere ($\vec{E}_{0}$).


%%%%%%%%%% I-O implications: estimating E %%%%%%%%%%
\subsection{Estimating $\vec{E}_{0}$}
%%%%%%%%%%

The assumptions of Equation~\ref{eq:epsilon_leontief_with_A_literature}
may cause another challenge for energy analysts. 
As shown in 
Equation~\ref{eq:epsilon_leontief_depreciation_simplification}
and discussed in Section~\ref{sec:estimating_epsilon-implications_chapter} above, 
the I-O method can be used to estimate energy intensities 
for each sector of the economy ($\bm{\varepsilon}$). 
With $\bm{\varepsilon}$ values in hand,
and assuming that $\bm{\varepsilon}$ is constant with respect to time,
energy analysts have estimated changes in energy demand 
from the biosphere ($\vec{E}_{0}$) 
as the output of economic sectors ($\hat{\vec{X}}$) 
increases or decreases by solving 
Equation~\ref{eq:epsilon_leontief_with_A_literature} 
for $\vec{E}_{0}$:

\begin{equation} \label{eq:Leontief_lit_solved_for_E}
	\vec{E}_{0} 
	= \hat{\vec{X}}(\vec{I} - \vec{A}^{\mathrm{T}})\bm{\varepsilon}.
\end{equation}

When Equation~\ref{eq:epsilon_leontief_with_A_literature}
is modified to account for accumulation of embodied energy 
in the economy 
$\left( \left. \frac{\mathrm{d}\vec{B}_{K}}{\mathrm{d}t} \right|_{\mathrm{other}} \right)$,
energy supplied by society to the economy ($\vec{T}_{1}$), and
waste materials ($\vec{B}_{waste}$),
we see that energy demands ($\vec{E}_{0}$) must be calculated differently. 
Solving Equation~\ref{eq:epsilon_leontief_depreciation_simplification} 
for $\vec{E}_{0}$ gives 

\begin{equation} \label{eq:Leontief_solved_for_E_with_embodied_depreciation}
	\vec{E}_{0} 
	= \hat{\vec{X}}
		(\vec{I} - \vec{A}^{\mathrm{T}})
		\bm{\varepsilon} 
	+ \left. \frac{\mathrm{d}\vec{B}_{K}}{\mathrm{d}t} \right|_{\mathrm{other}}
	+ \vec{B}_{waste}
	- \vec{T}_{1}.
\end{equation}

\noindent{}Comparison of Equations~\ref{eq:Leontief_lit_solved_for_E} 
and~\ref{eq:Leontief_solved_for_E_with_embodied_depreciation}
shows that to the extent that embodied energy accumulation,
exclusive of physical depreciation
$\left( \left. \frac{\mathrm{d}\vec{B}_{K}}{\mathrm{d}t} \right|_{\mathrm{other}} \right)$
is non-zero, estimates of energy demand ($\vec{E}_{0}$) using 
Equation~\ref{eq:Leontief_lit_solved_for_E} are too low. 
If energy input from society to the economy ($\vec{T}_{1}$) is significant,
estimates of energy demand ($\vec{E}_{0}$) using 
Equation~\ref{eq:Leontief_lit_solved_for_E} will be too high. 
And, if wastes ($\vec{B}_{waste}$) are ignored, 
estimates of energy demand ($\vec{E}_{0}$) using 
Equation~\ref{eq:Leontief_lit_solved_for_E} will be too low. 

At this time, the relative magnitudes of $\vec{E}_{0}$,
$\left. \frac{\mathrm{d}\vec{B}_{K}}{\mathrm{d}t} \right|_{\mathrm{other}}$,
$\vec{B}_{waste}$, and $\vec{T}_{1}$ are unknown. 
Further work to clarify these magnitudes will be beneficial
for energy analysts who employ the I-O method.


%%%%%%%%%% Implications %%%%%%%%%%
\section{Implications for economic ``development''}
\label{sec:implications_for_development}
%%%%%%%%%%

**** Becky, write a paragraph here about alternative measures of development. 
Discuss Daly's Genuine Progress Indicator (which include natural capital, Mik to verify)
and Gross National Happiness and other. ****

Economic ``development'' is usually ``measured''
by Gross Domestic Product\index{Gross Domestic Product} (GDP).
With reference to Figure~\ref{fig:C_value}, $GDP$ is calculated by

\begin{equation} \label{eq:GDP_def}
	GDP
	= \sum\limits_{i=2}^{n} \dot{X}_{i}
\end{equation}

\noindent{}where $n$ is the number of sectors in the economy.
Equation~\ref{eq:GDP_def} clearly shows that 
$GDP$ is a \emph{flow} of value.

However, one consequence of increased $GDP$ is often
increased capital stock ($K$) and associated
accumulation of embodied energy ($B_K$)
in economic sectors and society.\footnote{**** Say something about the difference 
between the flow of GDP and the stock of capital. Which is more important to count?
Or should we track both? Admit that we're not comfortable using
the term development or developed. ****} 
If we turn this around, 
accumulation of embodied energy in economic sectors and society 
could be considered a \emph{proxy} for development.\footnote{Embodied energy 
as a proxy for development may be overly focused on capital stock, 
therefore one-dimensional, and reductive, 
but GDP and other measures are open to similar criticism.}
Equation~\ref{eq:Dev_Integral_Economy} indicates how accumulated
embodied energy in the capital stock 
of an economy ($\vec{B}_{K}$) could be calculated:

\begin{equation} \label{eq:Dev_Integral_Economy}
	\vec{B}_{K}(t) 
	= \vec{B}_{K}(0) 
	+ \int_{t=0}^{t=t} \frac{\mathrm{d}\vec{B}_{K}}{\mathrm{d}t}\mathrm{d}t,
\end{equation}

\noindent{}where $\vec{B}_{K}$ is given by Equation~\ref{eq:B_vec_def}.
Equation~\ref{eq:Dev_Integral_Economy} clearly shows
that energy embodied energy in capital stock ($\vec{B}_{K}$) 
is a \emph{stock}, not a flow. 
**** Discuss units here to illuminate the stock vs.\ flow issue. ****

The behavior of $\vec{B}_{K}$ with respect to
$\left. \frac{\mathrm{d}\vec{B}}{\mathrm{d}t} \right|_{\mathrm{other}}$ 
is vitally important. 
An industrialized economy has significantly larger embodied energy ($\vec{B}_{K}$) 
than an agrarian economy, and, thus, 
the outflow rate of embodied energy 
due to depreciation ($\hat{\bm{\gamma}}_{K}\vec{B}_{K}$) will be larger, too. 
**** Discuss this in terms of the development path followed in the West for a given economy. ****
As increasingly large amounts of energy are embodied 
in the capital stock of an economy, 
Equation~\ref{eq:matrix_leontief} shows that
increasingly large energy extraction rates ($\vec{E}_{0}$) 
are required to maintain capital stock 
in the sectors of the economy
by offsetting the effects of depreciation 
$\left( \left. \frac{\mathrm{d}\vec{B}_{K}}{\mathrm{d}t} \right|_{\mathrm{other}} \right)$,
assuming that $\frac{\mathrm{d}\vec{B}_{K}}{\mathrm{d}t} \ge 0$ 
is desired. 

During a period of rapid industrialization and infrastructure build-out, 
we expect both $GDP$ and energy embodied in the economy ($\vec{B}_{K}$)
to increase.
But, there is no guarantee that $GDP$ and $\vec{B}_{K}$ move 
in the same direction at all times.
Industrialized economies may experience $GDP$ growth while 
the stock of embodied energy in the economy ($\vec{B}_{K}$) remains nearly constant,
because the economy is running circles to overcome the effects of depreciation.
In times of economic decline (such as occurred in post-colonial Africa),
$GDP$ may decrease 
while the stock of energy embodied in the economy ($\vec{B}_{K}$) increases 
due to foreign aid focused on infrastructure enhancement.
**** We may want a better example here.
Did this actually occur for any country in post-colonial Africa? 
Are there other examples or GDP and capital stock 
moving in opposite directions? --MKH ****

**** Discuss the above this in terms of a phase lag, 
between value crash, maintenance flows decline, 
then infrastructure will decline. ****

A second possible measure of economic ``development'' is another \emph{stock}, 
wealth:

\begin{equation} \label{eq:Dev_Integral_Wealth}
	X_{i}(t) 
	= X_{i}(0) 
	+ \int_{t=0}^{t=t} \frac{\mathrm{d}X_{i}}{\mathrm{d}t}\mathrm{d}t,
\end{equation}

\noindent{}where $i=1$ for societal wealth 
and $i \in [2,n]$ for corporate wealth.
In fact, capital (as represented by energy embodied in infrastructure, $\vec{B}_{K}$) and 
financial resources (as represeted by $X_{2 \ldots n}$) 
are complimentary factors for economic processes. 
But, we can go further than linking capital with financial resources.
If capital~($\vec{B}_{K}$) is to be useful, we need financial resources
or currency~($\dot{X}$) to 
\begin{itemize}
	\item{purchase direct energy~($\dot{E}$) to power the capital,
	\item{purchase resources~($\dot{R}$) to feed the capital, and}
	\item{pay workers~(represented 
	by societal energy input to the economy,~$\vec{T}_{1}$) 
	to operate the capital.}
} 
\end{itemize}

Thus, economic growth could be considered a ``fully coupled'' problem:
understanding it requires breadth of knowledge and appreciation for 
interactions among many important factors.
Each of the factors discussed above 
($\dot{X}$, $\vec{B}_{K}$, $\dot{E}$, $\dot{R}$, and $\vec{T}_{1}$)
is necessary, but not sufficient, for economic growth.

**** Consider moving this paragraph to the end of the Chapter. ****
As any economist will say, 
there are many pieces of unfinished business here:
whether ``growth'' should be measured by a stock or a flow, 
various alternative measures of economic ``growth,'' 
and the roles of currency, capital stock, energy, resources, and labor
in economic processes
are overlapping and complementary areas of inquiry.
We encourage further research in all of these areas.


%%%%%%%%%% Implications %%%%%%%%%%
\section{Implications for recycling, reuse, and dematerialization}
%%%%%%%%%%

Dematerialization is the idea that economic activity can be unlinked 
from material or energy demands.\cite{FischerKowalski:2011uo} 
One method for dematerializing an economy 
is reuse and recycling of materials.\footnote{**** MKH discuss technological 
substitution as another factor in dematerialzation. 
Substitution toward information and services and away from production
of material goods. ****}
The impact of recycling can be seen in the I-O formulation 
only when depreciation and accumulation terms are included. 

One effect of recycling is to reduce the magnitude 
of the depreciation ($\hat{\bm{\gamma}}_{K}$)
and waste ($\vec{B}_{waste}$) terms. 
Equation~\ref{eq:matrix_leontief}:

\begin{equation}
	\frac{\mathrm{d}\vec{B}_{K}}{\mathrm{d}t} 
	= \vec{E}_{0}
	+ \vec{T}_{1}
	+ \hat{\vec{X}} (\vec{A}^{\mathrm{T}} - \vec{I})\bm{\varepsilon} 
	- \vec{B}_{waste}
	- \hat{\bm{\gamma}}_{K}\vec{B}_{K}\tag{\ref{eq:matrix_leontief}}
\end{equation}

\noindent{}indicates that 
recycling of material in an economy, 
thereby reducing both $\vec{B}_{waste}$ 
and~$\hat{\bm{\gamma}}_{K}$, 
put upward pressure on the accumulation of energy embodied 
in capital stock
$\left(\frac{\mathrm{d}\vec{B}_{K}}{\mathrm{d}t}\right)$. 

**** Mik: take a shot a rewriting. ***** 
Recycling has a mixed effect on energy demand ($\vec{E}_{0}$). 
Because recycled material displaces newly-produced material 
in the economy and society, 
recycling will tend to reduce energy demand ($\vec{E}_{0}$). 
However, recycling processes require energy to operate, 
thereby increasing energy demand ($\vec{E}_{0}$). 
If recycling results in a net reduction of energy demand 
from the biosphere ($\vec{E}_{0}$), 
recycling will put downward pressure on the growth 
of embodied energy in the economy
$\left(\frac{\mathrm{d}\vec{B}_{K}}{\mathrm{d}t}\right)$. 

If recycling produces a net reduction in energy demand ($\vec{E}_{0}$), 
i.e.\ if the effect of displaced production dominates over the effect 
of energy consumed in recycling processes, 
the upward pressure on growth $\left(\frac{\mathrm{d}\vec{B}_{K}}{\mathrm{d}t}\right)$ 
from decrease in $\hat{\bm{\gamma}}_{K}$ and 
the downward pressure on growth from net reduction of $\vec{E}_{0}$ 
can offset each other.
Under those conditions, 
the accumulation rate of embodied energy in capital stock
$\left(\frac{\mathrm{d}\vec{B}_{K}}{\mathrm{d}t}\right)$ 
will remain near zero, 
and total embodied energy $(\vec{B}_{K})$ will remain constant. 
In that scenario, dematerialization can develop: 
reduced material and energy input ($\vec{E}_{0}$) can be accompanied by 
no change in
the growth of the economy
$\left(\frac{\mathrm{d}\vec{B}_{K}}{\mathrm{d}t}\right)$.


%%%%%%%%%% Implications %%%%%%%%%%
\section{Comparison to a steady-state economy}
%%%%%%%%%%

**** Discuss maintenance vs.\ accumulation flows? 
Is this a better place to discuss this issue? ****

****** Mik: Finish this section. 
In terms of what a SSE would look like in the I-O framework, 
at first blush, I would think that dB/dt = 0 is one aspect.  
Also, with no growth, inflow rates = depreciation rates.  
The larger that B is for any society, the larger E must be (to overcome depreciation).  
To minimize E, hyper-recycling is probably useful.  
Those are at least a place to start. ******

****** In our discussion, 
we also addressed the attempts at SSE from point of view of society. 
In order to achieve this goal \emph{without} recycling, 
the goods and services sector should have to increase extraction to offset decreasing ore grade, 
the energy sector should have to increase extraction of energy 
to allow increasing extraction (unless efficiency could make up the gap: unlikely) 
in which case the SSE would be violated from these two and from the POV of the earth.
******


\bibliographystyle{unsrt}
\bibliography{../../EROI_review_v2}


% Always give a unique label
% and use \ref{<label>} for cross-references
% and \cite{<label>} for bibliographic references
% use \sectionmark{}
% to alter or adjust the section heading in the running head
%% Instead of simply listing headings of different levels we recommend to let every heading be followed by at least a short passage of text. Furtheron please use the \LaTeX\ automatism for all your cross-references and citations.

%% Please note that the first line of text that follows a heading is not indented, whereas the first lines of all subsequent paragraphs are.

%% Use the standard \verb|equation| environment to typeset your equations, e.g.
%
%% \begin{equation}
%% a \times b = c\;,
%% \end{equation}
%
%% however, for multiline equations we recommend to use the \verb|eqnarray|
%% environment\footnote{In physics texts please activate the class option \texttt{vecphys} to depict your vectors in \textbf{\itshape boldface-italic} type - as is customary for a wide range of physical subjects.}.
%% \begin{eqnarray}
%% a \times b = c \nonumber\\
%% \vec{a} \cdot \vec{b}=\vec{c}
%% \label{eq:01}
%% \end{eqnarray}

%% \subsection{Subsection Heading}
%% \label{subsec:2}
%% Instead of simply listing headings of different levels we recommend to let every heading be followed by at least a short passage of text. Furtheron please use the \LaTeX\ automatism for all your cross-references\index{cross-references} and citations\index{citations} as has already been described in Sect.~\ref{sec:2}.

%% \begin{quotation}
%% Please do not use quotation marks when quoting texts! Simply use the \verb|quotation| environment -- it will automatically render Springer's preferred layout.
%% \end{quotation}


%% \subsubsection{Subsubsection Heading}
%% Instead of simply listing headings of different levels we recommend to let every heading be followed by at least a short passage of text. Furtheron please use the \LaTeX\ automatism for all your cross-references and citations as has already been described in Sect.~\ref{subsec:2}, see also Fig.~\ref{fig:1}\footnote{If you copy text passages, figures, or tables from other works, you must obtain \textit{permission} from the copyright holder (usually the original publisher). Please enclose the signed permission with the manucript. The sources\index{permission to print} must be acknowledged either in the captions, as footnotes or in a separate section of the book.}

%% Please note that the first line of text that follows a heading is not indented, whereas the first lines of all subsequent paragraphs are.

% For figures use
%
%% \begin{figure}[b]
%% \sidecaption
% Use the relevant command for your figure-insertion program
% to insert the figure file.
% For example, with the option graphics use
%% \includegraphics[scale=.65]{figure}
%
% If not, use
%\picplace{5cm}{2cm} % Give the correct figure height and width in cm
%
%% \caption{If the width of the figure is less than 7.8 cm use the \texttt{sidecapion} command to flush the caption on the left side of the page. If the figure is positioned at the top of the page, align the sidecaption with the top of the figure -- to achieve this you simply need to use the optional argument \texttt{[t]} with the \texttt{sidecaption} command}
%% \label{fig:1}       % Give a unique label
%% \end{figure}


%% \paragraph{Paragraph Heading} %
%% Instead of simply listing headings of different levels we recommend to let every heading be followed by at least a short passage of text. Furtheron please use the \LaTeX\ automatism for all your cross-references and citations as has already been described in Sect.~\ref{sec:2}.

%% Please note that the first line of text that follows a heading is not indented, whereas the first lines of all subsequent paragraphs are.

%% For typesetting numbered lists we recommend to use the \verb|enumerate| environment -- it will automatically render Springer's preferred layout.

%% \begin{enumerate}
%% \item{Livelihood and survival mobility are oftentimes coutcomes of uneven socioeconomic development.}
%% \begin{enumerate}
%% \item{Livelihood and survival mobility are oftentimes coutcomes of uneven socioeconomic development.}
%% \item{Livelihood and survival mobility are oftentimes coutcomes of uneven socioeconomic development.}
%% \end{enumerate}
%% \item{Livelihood and survival mobility are oftentimes coutcomes of uneven socioeconomic development.}
%% \end{enumerate}


%% \subparagraph{Subparagraph Heading} In order to avoid simply listing headings of different levels we recommend to let every heading be followed by at least a short passage of text. Use the \LaTeX\ automatism for all your cross-references and citations as has already been described in Sect.~\ref{sec:2}, see also Fig.~\ref{fig:2}.

%% Please note that the first line of text that follows a heading is not indented, whereas the first lines of all subsequent paragraphs are.

%% For unnumbered list we recommend to use the \verb|itemize| environment -- it will automatically render Springer's preferred layout.

%% \begin{itemize}
%% \item{Livelihood and survival mobility are oftentimes coutcomes of uneven socioeconomic development, cf. Table~\ref{tab:1}.}
%% \begin{itemize}
%% \item{Livelihood and survival mobility are oftentimes coutcomes of uneven socioeconomic development.}
%% \item{Livelihood and survival mobility are oftentimes coutcomes of uneven socioeconomic development.}
%% \end{itemize}
%% \item{Livelihood and survival mobility are oftentimes coutcomes of uneven socioeconomic development.}
%% \end{itemize}

%% \begin{figure}[t]
%% \sidecaption[t]
% Use the relevant command for your figure-insertion program
% to insert the figure file.
% For example, with the option graphics use
%% \includegraphics[scale=.65]{figure}
%
% If not, use
%\picplace{5cm}{2cm} % Give the correct figure height and width in cm
%
%% \caption{Please write your figure caption here}
%% \label{fig:2}       % Give a unique label
%% \end{figure}

%% \runinhead{Run-in Heading Boldface Version} Use the \LaTeX\ automatism for all your cross-references and citations as has already been described in Sect.~\ref{sec:2}.

%% \subruninhead{Run-in Heading Italic Version} Use the \LaTeX\ automatism for all your cross-refer\-ences and citations as has already been described in Sect.~\ref{sec:2}\index{paragraph}.
% Use the \index{} command to code your index words
%
% For tables use
%
%% \begin{table}
%% \caption{Please write your table caption here}
%% \label{tab:1}       % Give a unique label
%
% For LaTeX tables use
%
%% \begin{tabular}{p{2cm}p{2.4cm}p{2cm}p{4.9cm}}
%% \hline\noalign{\smallskip}
%% Classes & Subclass & Length & Action Mechanism  \\
%% \noalign{\smallskip}\svhline\noalign{\smallskip}
%% Translation & mRNA$^a$  & 22 (19--25) & Translation repression, mRNA cleavage\\
%% Translation & mRNA cleavage & 21 & mRNA cleavage\\
%% Translation & mRNA  & 21--22 & mRNA cleavage\\
%%Translation & mRNA  & 24--26 & Histone and DNA Modification\\
%%\noalign{\smallskip}\hline\noalign{\smallskip}
%%\end{tabular}
%%$^a$ Table foot note (with superscript)
%%\end{table}
%
%% \section{Section Heading}
%%\label{sec:3}
% Always give a unique label
% and use \ref{<label>} for cross-references
% and \cite{<label>} for bibliographic references
% use \sectionmark{}
% to alter or adjust the section heading in the running head
%% Instead of simply listing headings of different levels we recommend to let every heading be followed by at least a short passage of text. Furtheron please use the \LaTeX\ automatism for all your cross-references and citations as has already been described in Sect.~\ref{sec:2}.

%% Please note that the first line of text that follows a heading is not indented, whereas the first lines of all subsequent paragraphs are.

%%If you want to list definitions or the like we recommend to use the Springer-enhanced \verb|description| environment -- it will automatically render Springer's preferred layout.

%%\begin{description}[Type 1]
%%\item[Type 1]{That addresses central themes pertainng to migration, health, and disease. In Sect.~\ref{sec:1}, Wilson discusses the role of human migration in infectious disease distributions and patterns.}
%%\item[Type 2]{That addresses central themes pertainng to migration, health, and disease. In Sect.~\ref{subsec:2}, Wilson discusses the role of human migration in infectious disease distributions and patterns.}
%%\end{description}

%%\subsection{Subsection Heading} %
%% In order to avoid simply listing headings of different levels we recommend to let every heading be followed by at least a short passage of text. Use the \LaTeX\ automatism for all your cross-references and citations citations as has already been described in Sect.~\ref{sec:2}.

%% Please note that the first line of text that follows a heading is not indented, whereas the first lines of all subsequent paragraphs are.

%% \begin{svgraybox}
%% If you want to emphasize complete paragraphs of texts we recommend to use the newly defined Springer class option \verb|graybox| and the newly defined environment \verb|svgraybox|. This will produce a 15 percent screened box 'behind' your text.

%% If you want to emphasize complete paragraphs of texts we recommend to use the newly defined Springer class option and environment \verb|svgraybox|. This will produce a 15 percent screened box 'behind' your text.
%% \end{svgraybox}


%% \subsubsection{Subsubsection Heading}
%%Instead of simply listing headings of different levels we recommend to let every heading be followed by at least a short passage of text. Furtheron please use the \LaTeX\ automatism for all your cross-references and citations as has already been described in Sect.~\ref{sec:2}.

%% Please note that the first line of text that follows a heading is not indented, whereas the first lines of all subsequent paragraphs are.

%% \begin{theorem}
%% Theorem text goes here.
%% \end{theorem}
%
% or
%
%% \begin{definition}
%% Definition text goes here.
%% \end{definition}

%% \begin{proof}
%\smartqed
%% Proof text goes here.
%% \qed
%% \end{proof}

%%\paragraph{Paragraph Heading} %
%% Instead of simply listing headings of different levels we recommend to let every heading be followed by at least a short passage of text. Furtheron please use the \LaTeX\ automatism for all your cross-references and citations as has already been described in Sect.~\ref{sec:2}.

%% Note that the first line of text that follows a heading is not indented, whereas the first lines of all subsequent paragraphs are.
%
% For built-in environments use
%
%%\begin{theorem}
%%Theorem text goes here.
%%\end{theorem}
%
%%\begin{definition}
%%Definition text goes here.
%%\end{definition}
%
%%\begin{proof}
%%\smartqed
%% Proof text goes here.
%%\qed
%%\end{proof}
%
%% \begin{acknowledgement}
%% If you want to include acknowledgments of assistance and the like at the end of an individual chapter please use the \verb|acknowledgement| environment -- it will automatically render Springer's preferred layout.
%% \end{acknowledgement}
%
%% \section*{Appendix}
%% \addcontentsline{toc}{section}{Appendix}
%
%% When placed at the end of a chapter or contribution (as opposed to at the end of the book), the numbering of tables, figures, and equations in the appendix section continues on from that in the main text. Hence please \textit{do not} use the \verb|appendix| command when writing an appendix at the end of your chapter or contribution. If there is only one the appendix is designated ``Appendix'', or ``Appendix 1'', or ``Appendix 2'', etc. if there is more than one.

%% \begin{equation}
%% a \times b = c
%% \end{equation}
% Problems or Exercises should be sorted chapterwise
%% \section*{Problems}
%% \addcontentsline{toc}{section}{Problems}
%
% Use the following environment.
% Don't forget to label each problem;
% the label is needed for the solutions' environment
%% \begin{prob}
%% \label{prob1}
%% A given problem or Excercise is described here. The
%% problem is described here. The problem is described here.
%% \end{prob}

%% \begin{prob}
%% \label{prob2}
%% \textbf{Problem Heading}\\
%% (a) The first part of the problem is described here.\\
%% (b) The second part of the problem is described here.
%% \end{prob}


