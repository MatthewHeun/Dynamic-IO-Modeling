%!TEX root = ../../Heun_Dale_Haney_A_dynamic_approach_to_input_output_modeling.tex
%%%%%%%%%%%%%%%%%%%%% chapter.tex %%%%%%%%%%%%%%%%%%%%%%%%%%%%%%%%%
%
% sample chapter
%
% Use this file as a template for your own input.
%
%%%%%%%%%%%%%%%%%%%%%%%% Springer-Verlag %%%%%%%%%%%%%%%%%%%%%%%%%%
%\motto{Use the template \emph{chapter.tex} to style the various elements of your chapter content.}

%%%%%%%%%%%%%%%%%%%%%%%%%%%%%%%%%%%%%%%%%%%%%%%%%%%%%%%
%%%%%%%%%% Conceptual and Theoretical Isseus %%%%%%%%%%
%%%%%%%%%%%%%%%%%%%%%%%%%%%%%%%%%%%%%%%%%%%%%%%%%%%%%%%
\chapter{Practical, Methodological, and Theoretical Issues}
% Always give a unique label
\label{chap:unfinished_business}
% use \chaptermark{} to alter or adjust the chapter heading in the running head
\chaptermark{Issues}
%%%%%%%%%%%%%%%%%%%%%%%%%%%%%%%%%%
%%%%%%%%%%%%%%%%%%%%%%%%%%%%%%%%%%
%%%%%%%%%%%%%%%%%%%%%%%%%%%%%%%%%%

\abstract*{[NEED TO ADD ABSTRACT HERE]}

%% \abstract{Each chapter should be preceded by an abstract (10--15 lines long) that summarizes the content. The abstract will appear \textit{online} at \url{www.SpringerLink.com} and be available with unrestricted access. This allows unregistered users to read the abstract as a teaser for the complete chapter. As a general rule the abstracts will not appear in the printed version of your book unless it is the style of your particular book or that of the series to which your book belongs.\newline\indent
%% Please use the 'starred' version of the new Springer \texttt{abstract} command for typesetting the text of the online abstracts (cf. source file of this chapter template \texttt{abstract}) and include them with the source files of your manuscript. Use the plain \texttt{abstract} command if the abstract is also to appear in the printed version of the book.}

%% Use the template \emph{chapter.tex} together with the Springer document class SVMono (monograph-type books) or SVMult (edited books) to style the various elements of your chapter content in the Springer layout.


With any endeavor of this magnitude, 
namely development and comprehensive presentation 
of a framework for economies of the world,
unfinished business is inevitable. 
This chapter discusses several 
practical, methodological, and theoretical issues
that should be addressed in the future.
On a practical level, additional data are needed 
to fully utilize the framework developed herein.
In terms of methodological issues, we encourage that economists of all types
embrace material, energy, embodied energy, and economic value counting 
as a valid method of inquiry for modern economics.
And, there are issues of co-products and the choice of the energy input vector
that need to be addressed.  
Finally, several theoretical issues, including
material and energy quality, 
model boundaries, and 
the Sun
are addressed.
We begin with the topic of data.


%%%%%%%%%% Data %%%%%%%%%%
\section{Data}
\label{sec:Data}
%%%%%%%%%%

At several points in this manuscript,
we have noted the very practical issue of the
need for additional data, information, and analysis.
In particular, Chapter~\ref{chap:intensity} identified the need
for a method of independently determining of the rate of embodied energy 
accumulation in capital stock for the purposes of 
estimating the energy intensity of economic products,
especially for rapidly industrializing economies.
Chapter~\ref{chap:intensity} also noted the need to collect
data on human and draught animal physical work input to 
sectors of economies, espeically for those economies where 
muscle work is on the same order of magnitude 
as fossil fuel energy input.

Unfortunately, the U.S.\ stopped collecting 
satellite account data in 2000. 
Beyond the U.S., few countries collect and disseminate
the kind of data required as inputs to the framework
developed herein.

In Chapter~\ref{chap:intro}, we indicated that this book was 
about ``counting'' materials, energy, and value 
as each flows through economies.  
Unfortunately, unless data on these flows 
is collected and disseminated routinely, 
such counting is impossible.
The adage ``you can't control what you don't measure''
seems appropriate here. 
As the world confronts significant challenges
of material and energy supplies in the coming years,
it will be impossible to make wise decisions about
which materials to use,
which energy sources to develop, and 
which products to incentivize.

Another adage, ``we count what we value and we value what we count,''
points out that we are not presently valuing highly enough
the important flows of which our economies are comprised.
We add our voices to those encouraging governments and 
institutions to collect high-quality data on material 
and energy flows.
**** Becky, are there references that call for a renewed focus on 
collecting satellite account data? 
We should point to those references here. ****
 
**** Becky: review the above paragraphs.
Can you provide additional context on the decision
to stop collecting this data? ****


%%%%%%%%%% Metabolic Economy Models %%%%%%%%%%
\section{Metaphors and models}
\label{sec:mataphors_and_models}
%%%%%%%%%%

**** Mik and Becky: please review and comment on this section. --Matt ****

Historically, mainstream economists have used the metaphor of \emph{machine}
to describe economies. 
In fact, much of mathematical economic modeling today 
relies upon mechanistic conceptual foundations whose roots 
are in Newtonian physics and the enlightenment.\footnote{The subfield 
of longwave economic growth modeling, in particular, 
uses mechanistic language to describe their equations. 
For example, Jones \cite[pp. 6 \& 18]{Jones:2001wn} 
refers to the ordinary differential equations 
that describe births, deaths, and the production of ideas
as ``laws of motion.''}
If your metahpor is a machine, it makes sense to analyze economies
with equations that describe machines.
In the mainstream, such equations usually, but not exclusively or restrictively,
are developed assuming that the economic machine is a \emph{closed} system.

In this book, we take the strong position that economies are
better-represented as open, organic metabolisms.
Many ecological economists, 
who are predisposed to view the economy as a metabolism,
reject the use of mechanistic equations 
to describe the economy, in part because such equations often assume a 
machine that operates independently from the biosphere.
Other ecological economists dismiss mainstream economics 
and its mechanistic models out of hand. 
For example, S{\"o}llner pejoratively says 
``The prime example of physics envy and the desire 
to emulate the natural sciences is, of course, 
neoclassical economics which was explicitly and purposefully 
copied from classical mechanics.'' \cite[p. 178]{Sollner:1997wx}

But, it doesn't have to be this way.
We claim that it is not the use of equations 
and mathematical models themselves that is the problem.
Rather, it is the application of such equations 
to an assumed closed system
that is is worthy of criticism.
Thus, our approach herein includes rigorous application of the 
admittedly mechanistic accounting equations 
for materials, energy, embodied energy, and economic value
\emph{under the assumption of} and \emph{including} 
vigorous and necessary interactions between
the economy and the biosphere.

It is our hope that this book will allow

\begin{itemize}
	\item{mainstream economists to see that}
	\begin{itemize}
		\item{the metabolism metaphor is apt because 
		interaction between the economy and the biosphere is 
		what drives economic growth and}
		\item{rigorous mechanistic models can and should include 
		interactions between the biosphere and the economy, and} 
	\end{itemize}
	\item{ecological economists to see that rigorous mathematical modeling
	is an important tool for understanding interactions
	between the economy and the biosphere.}
\end{itemize}

In summary, we encourage both mainstream and ecological economists
to embrace mathematical modeling of 
material, 
energy, 
embodied energy, 
and economic value flows
both within the economy and between the economy and the biosphere
as a valid method of inquiry for modern economics.


%%%%%%%%%% Make-Use %%%%%%%%%%
\section{Co-products}
\label{sec:make-use}
%%%%%%%%%%

The intersection between our materials, energy, and value accounting framework 
and the I-O literature has been developed under the assumption 
that each economic sector makes a single product ($\dot{P}$).
In particular, the matrix mathematics of Chapter~\ref{chap:intensity}
relies heavily on this assumption 
from the early days of the I-O method.\cite{Bullard:1978vd}
However, the I-O method has been extended 
in the literature to include
co-products for each economic sector.\cite{Costanza:1984tq,Casler1984} 
To do so, both \emph{make} and \emph{use} data are employed.

We decided to leverage the older,
single-product formulation of the I-O method
for the purposes of simplicity. 
The materials, energy, and value accounting framework
presented herein is more easily understood 
without the complexity of the make-use formulation 
of the I-O method.
However, work remains to adapt the materials and energy framework
developed herein to the make-use formulation of the I-O method.


%%%%%%%%%% Energy Input Vector %%%%%%%%%%
\section{Choice of energy input vector}
\label{sec:energy_input_vector}
%%%%%%%%%%

Consistent with traditional I-O methods, 
the derivation of the materials, energy, and value framework
presented herein counts energy at the point of inflow to the economy. 
That is, elements of the energy input vector to the economy
($\vec{E}_{0}$) are zero except for those sectors 
that receive energy directly from the Earth. 
So, in Figure~\ref{fig:C_energy} from Example~C, 
$\dot{E}_{03} = 0$ and $\dot{E}_{02} \ge 0$. 

Costanza (1984) suggests an alternative approach, 
namely to count energy input to the economy 
at the point of conversion to useful work. 
Theoretical justification for Costanza's Direct Energy Conversion~(DEC) 
approach comes from both thermodynamic and economic considerations. 
The thermodynamic justification derives from the purpose 
of energy consumption in an economy, 
namely to produce useful work. 
If energy flows \emph{through} a sector, 
it should not be counted ``against'' that sector: 
only energy that is converted to useful work \emph{within} 
a sector should be counted against that sector.
The economic justification derives 
from the typical treatment of transportation sectors of the economy.
Costanza notes:

\begin{quote}
	The primary energy sectors functions are like the transportation sectors, 
	which alse require special treatment in I-O analysis 
	based on the difference between the services they provide 
	and their physical inputs and outputs. 
	If a strictly physical interpretation were applied to the transportation sectors, 
	they would receive almost all goods produced in the whole economy as inputs 
	and redistribute them as output, 
	masking information on transfers of goods between sectors. 
	For this reason, the transportation sectors in I-O analysis 
	are thought of as providing transportation services 
	that are purchased by the producing sector, 
	preserving the connection between the producing and consuming sector 
	but adding a `transportation margin.' 
	For analogous reasons, 
	the primary energy sectors should be thought of as providing a 
	`transportation service' in moving primary energy 
	from nature to the consuming sectors. 
	The DEC energy input vector incorporates this interpretation.\cite{Costanza:1984tq}
\end{quote}

The DEC approach implicitly redefines energy intensity ($\bm{\varepsilon}$)
to be the consumed amount of fossil fuel energy to produce a unit of economic output. 

****** Equation redefining $\varepsilon$ here.

In the DEC approach, 
electricity consumption is converted to its fossil energy equivalent~(coal) 
before being ``applied'' to an economic sector. 
And, refined petroleum is converted to its fossil energy equivalent~(crude) 
before being ``applied'' to an economic sector.

****** The DEC option is akin to my idea of substituting the 
1st Law into the total energy equation. 
Show this derivation after redefining $\varepsilon$ 
to be embodied energy per dollar, not total energy per dollar. 
So, there is a second implicit assumption in Costanza (1984), 
namely that we have a re-derivation of energy intensity. ******

****** Show that re-derivation results in 
only counting the energy burned by each sector 
(or the waste heat off of each sector). 
Costanza (1984) shows that distributing energy input 
at the point of consumption reduces the variance 
of energy intensity across all sectors of the economy. ******


%%%%%%%%%% Second Law %%%%%%%%%%
\section{Material and energy quality}
\label{sec:material_and_energy_quality}
%%%%%%%%%%

The quality of both materials and energy play a role in the 
efficiency with which economies process materials into products with energy.
This section touches briefly on both.

%+++++++++ Material Quality ++++++++++
\subsection{Quality of materials}
\label{sec:material_quality}
%+++++++++

**** Mik: Please review and add to this draft section. --Matt ****

Material quality has an effect on our framework in two ways. 
First, in a so-called ``Fourth Law of Thermodynamics''
posited by Georgescu-Roegen, and second, 
in terms of biosphere resource quality.

Georgescu-Roegen proposed a Fourth Law of Thermodynamics 
which states that ``In a closed system, 
the material entropy must ultimately reach a maximum:'' 
material degrades, just as energy.\cite{GeorgescuRoegen:1977tf}
For example, consumption of coal to produce electricity
results in CO$_2$ and ash that cannot
be reprocessed into coal.
Although material mass is conserved through processes, 
CO$_2$ and ash are significantly less useful than coal,
and Georgescu-Roegen claims that we need a way to account
for the loss of material quality.\footnote{Note that 
Georgescu-Roegen's approach establishes analogies for
\begin{itemize}
	\item{the Law of Conservation of Mass, in which mass can be neither created nor destroyed and} 
	\item{the First Law of Thermodynamics, in which energy can be neither created nor destroyed,}
\end{itemize}
\noindent{}and
\begin{itemize}
	\item{the Fourth Law of Thermodynamics, in which mass quality degrades with use and}
	\item{the Second Law of Thermodynamics, in which energy quality degrades with use.}
\end{itemize}}

Furthermore, initial extraction of easy-to-reach ores 
and fossil fuel resources (coal and oil)
leads, over time, to a reduction in the quality 
of material resources (ore gredes decrease over time) 
**** Mik: do you have a reference for this? ****
and increased energy resources needed to perform extraction.

Further work needs to be done to enfold material quality issues
into the materials accounting framework in Chapter~\ref{chap:materials}.


%+++++++++ Energy Quality ++++++++++
\subsection{Quality of energy}
\label{sec:energy_quality}
%+++++++++

There are several ways to assess the quality of energy.
The Second Law of Thermodynamics provides, among other things,
a framework for discussing the \emph{quality} of energy.
Hammond and Winnett~\cite{Hammond:2009tu} reviewed
the influence of thermodynamics on ecological economics 
and noted the importance of the concept of \emph{exergy},
which combines the First and Second Laws of thermodynamics
to describe the maximum physical work obtainable from an energy source:

\begin{quote}
	[i]n a sense [exergy] represents the thermodynamic `quality' 
	of an energy carrier, 
	and that of the waste heat or energy lost in the reject stream. 
	Electricity, for instance, may be regarded as an energy carrier having a high quality, 
	or exergy, because it can undertake work. 
	In contrast, low temperature hot water, although also an energy source, 
	can only be used for heating purposes. 
	This distinction between energy (strictly enthalpy) and exergy is very important 
	when considering a switch, for example, 
	from traditional internal combustion engines to electric, hybrid, or fuel cell vehicles. 
	Thus, \ldots{} it is important to employ exergy analysis 
	alongside a traditional First Law energy analysis in order to illuminate these issues.
\end{quote}

The quality of energy can be assessed in terms of its economic value, too.
Some energy sources have been shown to be more economically valuable in their use
than others.
And, ``accounting for energy quality reveals a relatively strong relationship 
between energy use and economic output.''\cite[p. 313]{Cleveland2000}

We recommend that energy quality issues be investigated for inclusion
in the framework developed herein.


%%%%%%%%%% Endogenous %%%%%%%%%%
\section{What is Endogenous?}
%%%%%%%%%%

Are government and households endogenous? 
Costanza (1980) was the first to endogenize government and households, 
because households provide services to the economy (labor) 
in exchange for wages and government provides services 
to the economy in exchange for taxes, both of which require energy. 
Costanza (1980) showed that by including government and households 
as sectors in the economy, 
the variation of energy intensity is significantly reduced 
across all sectors of the economy. 

**** Idea: Re-derive the energy intensity equation,
Equation~\ref{eq:epsilon_leontief_depreciation_simplification},
including Final Demand (1). ****


%%%%%%%%%% The Sun %%%%%%%%%%
\section{What About the Sun?}
%%%%%%%%%%

Costanza (1980) includes an option to consider 
the sun as an input to the economy, 
thereby significantly increasing the energy intensity 
of agricultural sectors and other sectors 
that depend upon agricultural outputs, 
however Costanza (1984) did not include 
the sun.~\cite{Costanza:1980ww,Costanza:1984tq}
Whether solar input to the economy 
should be considered is probably dependent upon 
the objectives of the analysis. 
In this framework, we are primarily interested 
in the effects of declining energy resource quality in industrial economies, 
due to depletion of fossil fuels. 
As such, inclusion of solar flows is unnecessary. 
However, expanding the framework to include non-industrial 
or more agrarian societies would probably require accounting for these flows. 
Additionally, similar concerns might be raised 
in dealing with a society that is largely reliant on solar or wind energy.   
[EARLIER FOOTNOTE ON INDUSTRIAL VS.\ 
NON-INDUSTRIAL ECONOMIES COULD BE BROUGHT IN HERE---MD]

There are a number of means by which solar flows can be accounted. 
Short-term solar flows could be accounted 
in the output of agricultural and forestry sectors, 
as well as some of the renewable energy producers, 
such as solar thermal and PV, 
wind, 
ocean thermal, 
hydro-power, and 
biomass. 
This method does not account for longer-term flows 
of solar energy used to form fossil fuels. 
The \emph{emergy} accounting method puts all flows 
in terms of \emph{em}bodied en\emph{ergy} flows.~\cite{Odum1975, Odum1996}
The basic unit of measure is 
the \emph{em}joule which is often given in terms 
of flows of solar energy embodied in 
the energy (or material)---the solar emjoules---per unit of resource, 
abbreviated to seJ/J for energy resources, 
or seJ/g for materials. 
As such, even fossil fuels, e.g.\ coal, 
extracted from the earth have an embodied energy 
of around 67,000~seJ/J.~\cite{Brown2004} 


\bibliographystyle{unsrt}
\bibliography{../../EROI_review_v2}


% Always give a unique label
% and use \ref{<label>} for cross-references
% and \cite{<label>} for bibliographic references
% use \sectionmark{}
% to alter or adjust the section heading in the running head
%% Instead of simply listing headings of different levels we recommend to let every heading be followed by at least a short passage of text. Furtheron please use the \LaTeX\ automatism for all your cross-references and citations.

%% Please note that the first line of text that follows a heading is not indented, whereas the first lines of all subsequent paragraphs are.

%% Use the standard \verb|equation| environment to typeset your equations, e.g.
%
%% \begin{equation}
%% a \times b = c\;,
%% \end{equation}
%
%% however, for multiline equations we recommend to use the \verb|eqnarray|
%% environment\footnote{In physics texts please activate the class option \texttt{vecphys} to depict your vectors in \textbf{\itshape boldface-italic} type - as is customary for a wide range of physical subjects.}.
%% \begin{eqnarray}
%% a \times b = c \nonumber\\
%% \vec{a} \cdot \vec{b}=\vec{c}
%% \label{eq:01}
%% \end{eqnarray}

%% \subsection{Subsection Heading}
%% \label{subsec:2}
%% Instead of simply listing headings of different levels we recommend to let every heading be followed by at least a short passage of text. Furtheron please use the \LaTeX\ automatism for all your cross-references\index{cross-references} and citations\index{citations} as has already been described in Sect.~\ref{sec:2}.

%% \begin{quotation}
%% Please do not use quotation marks when quoting texts! Simply use the \verb|quotation| environment -- it will automatically render Springer's preferred layout.
%% \end{quotation}


%% \subsubsection{Subsubsection Heading}
%% Instead of simply listing headings of different levels we recommend to let every heading be followed by at least a short passage of text. Furtheron please use the \LaTeX\ automatism for all your cross-references and citations as has already been described in Sect.~\ref{subsec:2}, see also Fig.~\ref{fig:1}\footnote{If you copy text passages, figures, or tables from other works, you must obtain \textit{permission} from the copyright holder (usually the original publisher). Please enclose the signed permission with the manucript. The sources\index{permission to print} must be acknowledged either in the captions, as footnotes or in a separate section of the book.}

%% Please note that the first line of text that follows a heading is not indented, whereas the first lines of all subsequent paragraphs are.

% For figures use
%
%% \begin{figure}[b]
%% \sidecaption
% Use the relevant command for your figure-insertion program
% to insert the figure file.
% For example, with the option graphics use
%% \includegraphics[scale=.65]{figure}
%
% If not, use
%\picplace{5cm}{2cm} % Give the correct figure height and width in cm
%
%% \caption{If the width of the figure is less than 7.8 cm use the \texttt{sidecapion} command to flush the caption on the left side of the page. If the figure is positioned at the top of the page, align the sidecaption with the top of the figure -- to achieve this you simply need to use the optional argument \texttt{[t]} with the \texttt{sidecaption} command}
%% \label{fig:1}       % Give a unique label
%% \end{figure}


%% \paragraph{Paragraph Heading} %
%% Instead of simply listing headings of different levels we recommend to let every heading be followed by at least a short passage of text. Furtheron please use the \LaTeX\ automatism for all your cross-references and citations as has already been described in Sect.~\ref{sec:2}.

%% Please note that the first line of text that follows a heading is not indented, whereas the first lines of all subsequent paragraphs are.

%% For typesetting numbered lists we recommend to use the \verb|enumerate| environment -- it will automatically render Springer's preferred layout.

%% \begin{enumerate}
%% \item{Livelihood and survival mobility are oftentimes coutcomes of uneven socioeconomic development.}
%% \begin{enumerate}
%% \item{Livelihood and survival mobility are oftentimes coutcomes of uneven socioeconomic development.}
%% \item{Livelihood and survival mobility are oftentimes coutcomes of uneven socioeconomic development.}
%% \end{enumerate}
%% \item{Livelihood and survival mobility are oftentimes coutcomes of uneven socioeconomic development.}
%% \end{enumerate}


%% \subparagraph{Subparagraph Heading} In order to avoid simply listing headings of different levels we recommend to let every heading be followed by at least a short passage of text. Use the \LaTeX\ automatism for all your cross-references and citations as has already been described in Sect.~\ref{sec:2}, see also Fig.~\ref{fig:2}.

%% Please note that the first line of text that follows a heading is not indented, whereas the first lines of all subsequent paragraphs are.

%% For unnumbered list we recommend to use the \verb|itemize| environment -- it will automatically render Springer's preferred layout.

%% \begin{itemize}
%% \item{Livelihood and survival mobility are oftentimes coutcomes of uneven socioeconomic development, cf. Table~\ref{tab:1}.}
%% \begin{itemize}
%% \item{Livelihood and survival mobility are oftentimes coutcomes of uneven socioeconomic development.}
%% \item{Livelihood and survival mobility are oftentimes coutcomes of uneven socioeconomic development.}
%% \end{itemize}
%% \item{Livelihood and survival mobility are oftentimes coutcomes of uneven socioeconomic development.}
%% \end{itemize}

%% \begin{figure}[t]
%% \sidecaption[t]
% Use the relevant command for your figure-insertion program
% to insert the figure file.
% For example, with the option graphics use
%% \includegraphics[scale=.65]{figure}
%
% If not, use
%\picplace{5cm}{2cm} % Give the correct figure height and width in cm
%
%% \caption{Please write your figure caption here}
%% \label{fig:2}       % Give a unique label
%% \end{figure}

%% \runinhead{Run-in Heading Boldface Version} Use the \LaTeX\ automatism for all your cross-references and citations as has already been described in Sect.~\ref{sec:2}.

%% \subruninhead{Run-in Heading Italic Version} Use the \LaTeX\ automatism for all your cross-refer\-ences and citations as has already been described in Sect.~\ref{sec:2}\index{paragraph}.
% Use the \index{} command to code your index words
%
% For tables use
%
%% \begin{table}
%% \caption{Please write your table caption here}
%% \label{tab:1}       % Give a unique label
%
% For LaTeX tables use
%
%% \begin{tabular}{p{2cm}p{2.4cm}p{2cm}p{4.9cm}}
%% \hline\noalign{\smallskip}
%% Classes & Subclass & Length & Action Mechanism  \\
%% \noalign{\smallskip}\svhline\noalign{\smallskip}
%% Translation & mRNA$^a$  & 22 (19--25) & Translation repression, mRNA cleavage\\
%% Translation & mRNA cleavage & 21 & mRNA cleavage\\
%% Translation & mRNA  & 21--22 & mRNA cleavage\\
%%Translation & mRNA  & 24--26 & Histone and DNA Modification\\
%%\noalign{\smallskip}\hline\noalign{\smallskip}
%%\end{tabular}
%%$^a$ Table foot note (with superscript)
%%\end{table}
%
%% \section{Section Heading}
%%\label{sec:3}
% Always give a unique label
% and use \ref{<label>} for cross-references
% and \cite{<label>} for bibliographic references
% use \sectionmark{}
% to alter or adjust the section heading in the running head
%% Instead of simply listing headings of different levels we recommend to let every heading be followed by at least a short passage of text. Furtheron please use the \LaTeX\ automatism for all your cross-references and citations as has already been described in Sect.~\ref{sec:2}.

%% Please note that the first line of text that follows a heading is not indented, whereas the first lines of all subsequent paragraphs are.

%%If you want to list definitions or the like we recommend to use the Springer-enhanced \verb|description| environment -- it will automatically render Springer's preferred layout.

%%\begin{description}[Type 1]
%%\item[Type 1]{That addresses central themes pertainng to migration, health, and disease. In Sect.~\ref{sec:1}, Wilson discusses the role of human migration in infectious disease distributions and patterns.}
%%\item[Type 2]{That addresses central themes pertainng to migration, health, and disease. In Sect.~\ref{subsec:2}, Wilson discusses the role of human migration in infectious disease distributions and patterns.}
%%\end{description}

%%\subsection{Subsection Heading} %
%% In order to avoid simply listing headings of different levels we recommend to let every heading be followed by at least a short passage of text. Use the \LaTeX\ automatism for all your cross-references and citations citations as has already been described in Sect.~\ref{sec:2}.

%% Please note that the first line of text that follows a heading is not indented, whereas the first lines of all subsequent paragraphs are.

%% \begin{svgraybox}
%% If you want to emphasize complete paragraphs of texts we recommend to use the newly defined Springer class option \verb|graybox| and the newly defined environment \verb|svgraybox|. This will produce a 15 percent screened box 'behind' your text.

%% If you want to emphasize complete paragraphs of texts we recommend to use the newly defined Springer class option and environment \verb|svgraybox|. This will produce a 15 percent screened box 'behind' your text.
%% \end{svgraybox}


%% \subsubsection{Subsubsection Heading}
%%Instead of simply listing headings of different levels we recommend to let every heading be followed by at least a short passage of text. Furtheron please use the \LaTeX\ automatism for all your cross-references and citations as has already been described in Sect.~\ref{sec:2}.

%% Please note that the first line of text that follows a heading is not indented, whereas the first lines of all subsequent paragraphs are.

%% \begin{theorem}
%% Theorem text goes here.
%% \end{theorem}
%
% or
%
%% \begin{definition}
%% Definition text goes here.
%% \end{definition}

%% \begin{proof}
%\smartqed
%% Proof text goes here.
%% \qed
%% \end{proof}

%%\paragraph{Paragraph Heading} %
%% Instead of simply listing headings of different levels we recommend to let every heading be followed by at least a short passage of text. Furtheron please use the \LaTeX\ automatism for all your cross-references and citations as has already been described in Sect.~\ref{sec:2}.

%% Note that the first line of text that follows a heading is not indented, whereas the first lines of all subsequent paragraphs are.
%
% For built-in environments use
%
%%\begin{theorem}
%%Theorem text goes here.
%%\end{theorem}
%
%%\begin{definition}
%%Definition text goes here.
%%\end{definition}
%
%%\begin{proof}
%%\smartqed
%% Proof text goes here.
%%\qed
%%\end{proof}
%
%% \begin{acknowledgement}
%% If you want to include acknowledgments of assistance and the like at the end of an individual chapter please use the \verb|acknowledgement| environment -- it will automatically render Springer's preferred layout.
%% \end{acknowledgement}
%
%% \section*{Appendix}
%% \addcontentsline{toc}{section}{Appendix}
%
%% When placed at the end of a chapter or contribution (as opposed to at the end of the book), the numbering of tables, figures, and equations in the appendix section continues on from that in the main text. Hence please \textit{do not} use the \verb|appendix| command when writing an appendix at the end of your chapter or contribution. If there is only one the appendix is designated ``Appendix'', or ``Appendix 1'', or ``Appendix 2'', etc. if there is more than one.

%% \begin{equation}
%% a \times b = c
%% \end{equation}
% Problems or Exercises should be sorted chapterwise
%% \section*{Problems}
%% \addcontentsline{toc}{section}{Problems}
%
% Use the following environment.
% Don't forget to label each problem;
% the label is needed for the solutions' environment
%% \begin{prob}
%% \label{prob1}
%% A given problem or Excercise is described here. The
%% problem is described here. The problem is described here.
%% \end{prob}

%% \begin{prob}
%% \label{prob2}
%% \textbf{Problem Heading}\\
%% (a) The first part of the problem is described here.\\
%% (b) The second part of the problem is described here.
%% \end{prob}


