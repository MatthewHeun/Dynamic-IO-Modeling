%!TEX root = ../Heun_Dale_Haney_A_dynamic_approach_to_input_output_modeling.tex

%%%%%%%%%%%%%%%%%%%%%%%%%%%%%%%%%%%%%%%%%%%%%%%%%%%%%%%%%%%%%%%%%%%
%%%%%%%%%%%%%%%%%%%%%%%%%%%%%%%%%%%%%%%%%%%%%%%%%%%%%%%%%%%%%%%%%%%
% This file contins information for the index.
% The Index is built with the glossaries package.
%%%%%%%%%%%%%%%%%%%%%%%%%%%%%%%%%%%%%%%%%%%%%%%%%%%%%%%%%%%%%%%%%%%
%%%%%%%%%%%%%%%%%%%%%%%%%%%%%%%%%%%%%%%%%%%%%%%%%%%%%%%%%%%%%%%%%%%

% Define the index database for the nomenclature section
\newglossary{index}{gidx}{gind}{Index}

% Define the two-column style for the index
\newglossarystyle{indexstyle}{
    \glossarystyle{indexgroup}
    \renewcommand*{\glossarypreamble}{\begin{multicols}{2}} 
    \renewcommand*{\glossarypostamble}{\end{multicols}}
}
	
%%%%%%%%%%%%%%%%%%%%%%%%%%%%%%%%%%%%%
% Provide index entries.
%
% Index entries need to be defined in this file before they are used in the document.
%
% Fields of the index entries are:
%    key: the unnamed argument right after \newglossaryentry{_key_}{params}.
%         Convention: Let's adopt the convention that ALL keys are ALL lower-case.
%    type: which glossary will contain this term. For all entries in this index file, use "index"
%    name: how the entry will appear in the Index
%    text: how the entry will appear in the body of the book when referenced by \gls{key}
%    first: how the entry will appear the FIRST time it is used in the book
%    plural: how the entry will appear when referenced by \glspl{key}
%    firstplural: how the entry will appear the FIRST time when referenced by \glspl{key}
%    sort: how the entry will be sorted in the Index. If omitted, name is used.
%    description: set to {\nopostdesc}, because we want to suppress descriptions in the Index
%
% Index entries are used in the document as follows:
% 
%    These energy flows originate from the natural environment, 
%    recognition of which provoked researchers from fields of 
%    \gls{nea} to extend the traditional~(\gls{leontief}) 
%    \gls{i-o} framework to include important 
%    energy flows from the environment, 
%    developing an \gls{ei-o}
%    framework as depicted in Figure~\ref{fig:basic_unit}B.
%
%    At the beginning of each chapter, we may wish to say
%    \glsresetall[index] to show the "first" version of each entry again.
%%%%%%%%%%%%%%%%%%%%%%%%%%%%%%%%%%%%%

\newglossaryentry{ei-o}{type = index, 
						name = energy input-output,
						text = EI-O, 
						first = Energy Input-Output (EI-O),
						description = {\nopostdesc}}

\newglossaryentry{i-o}{type = index, 
						name = input-output,
						text = I-O, 
						first = Input-Output (I-O),
						description = {\nopostdesc}}

\newglossaryentry{leontief}{type = index, 
						name = Leontief, 
						text = Leontief,
						description = {\nopostdesc}}
						
\newglossaryentry{nea}{type = index, 
						name = net energy analysis, 
						text = net energy analysis,
						description = {\nopostdesc}}
