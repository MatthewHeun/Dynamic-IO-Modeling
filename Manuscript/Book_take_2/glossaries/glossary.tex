%!TEX root = ../Heun_Dale_Haney_A_dynamic_approach_to_input_output_modeling.tex

%%%%%%%%%%%%%%%%%%%%%%%%%%%%%%%%%%%%%%%%%%%%%%%%%%%%%%%%%%%%%%%%%%%
%%%%%%%%%%%%%%%%%%%%%%%%%%%%%%%%%%%%%%%%%%%%%%%%%%%%%%%%%%%%%%%%%%%
% This file contins information for the acronym section.
% Acronyms is built with the glossaries package.
%%%%%%%%%%%%%%%%%%%%%%%%%%%%%%%%%%%%%%%%%%%%%%%%%%%%%%%%%%%%%%%%%%%
%%%%%%%%%%%%%%%%%%%%%%%%%%%%%%%%%%%%%%%%%%%%%%%%%%%%%%%%%%%%%%%%%%%

% Define the glossary database for the nomenclature section
\newglossary{glossary}{gloin}{gloout}{Glossary}

% Define the style for the glossary
\newglossarystyle{glossarystyle}{
	\glossarystyle{long}
}

%%%%%%%%%%%%%%%%%%%%%%%%%%%%%%%%%%%%%
% Provide Glossary entries.
%
% Fields of the glossary entries are:
%    key: the unnamed argument right after \newglossaryentry{_key_}{params}.
%         Convention: Let's adopt the convention that ALL keys are ALL lower-case 
%                     and prefixed by "glo:". This prevents conflicts with the Index.
%    name: how the symbol will appear in the Glossary
%    description: the description of the term.
%    sort: how the entry will be sorted in the Index.
%         Convention: Always include a sort field.
%                     Use ALL lowercase letters in the sort field.
%%%%%%%%%%%%%%%%%%%%%%%%%%%%%%%%%%%%%

\newglossaryentry{glo:bea}{type = glossary, 
								name = BEA,
								description = Bureau of Economic Analysis{,}
												US Department of commerce 
												(\url{http://www.bea.gov}),
								sort = bea,
								}

\newglossaryentry{glo:nea}{type = glossary, 
								name = NEA,
								description = Net Energy Analysis,
								sort = nea,
								}

